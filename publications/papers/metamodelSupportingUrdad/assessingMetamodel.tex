\section{Assessing the URDAD metamodel \label{sec:metamodelAssessment}}



%-------------------------------------------------------------------------------------

\subsection{Core metamodel integrity assessment}


\begin{itemize}
  \item Check that no classes unsatisfiable
  \item Check consistency of OCL constraints
  \item Redundency checks (redundent classes, redundent attributes and properties) redundend constraints
\end{itemize}

%-------------------------------------------------------------------------------------

\subsection{Assessing model qualities}

\begin{itemize}
  \item Comparing complexity of URDAD-DSL and UML encodings of URDAD model
    \begin{itemize}
     \item UML model requires much more model elements to encode URDAD semantics.
    \end{itemize}
  \item Assessing traceability (\cite{dick_design_2005} - see notes on Zotero entry)
  \item Completeness checks
(What does completeness mean - That only technical information needs to be provided and that the full requirements
from a business perspective are specified across levels of granularity.)
    \begin{itemize}
     \item all OCL constraints for decision conditions specified
     \item all OCL constraints for pre and post-conditions specified
     \item All required request and result fields all specified either via OCL constraints 
	or by default values.
    \end{itemize}
  \item Check that process addresses all functional requirements and nothing but the functional requirements.
\end{itemize}

%-------------------------------------------------------------------------------------

\subsection{Sufficiency}

An URDAD model is meant to be able to absorb the information required for performing implementation mappings, generating service tests, and
requirements traceability.


Jeremy Dick \cite{dick_design_2005} emphasizes the importance of traceability need for traceability across levels of granularity

- traceability generally important in design (not only software)

- traceability links can be enriched through rationale sematics (URDAD: requiredBy, requiredService, ...)

- include sufficiency and necessity, i.e. are all requirements met and are all design elements necessary/required to realize requirements

- identify immediate impact point, calculate potential impact tree, prune impact tree

Ramesh et al\. \cite{ramesh_toward_2001} identify 4 core traceability link types including (i) {\em satisfaction links} (ensuring that requirements are satisfied by system), (ii) \emph{evolutional links (document inputs resulting in changes to objects and resulting changes)

  3. Rationale links (why is this required, use instead requires by)

  4. Dependency links (from model reflection: dependency between services and so on)
%-------------------------------------------------------------------------------------

\subsection{Usability assessment (potentially an empirical study)}

\begin{itemize}
 \item Have 2 groups doing URDAD modeling, one using UML encoding and UML tools and one using URDAD encoding.
 \item Assess relative productivity, error densities and completeness of the two resultant models 
\end{itemize}

