\section{Assessing the URDAD metamodel \label{sec:metamodelAssessment}}



%-------------------------------------------------------------------------------------

\subsection{Core metamodel integrity assessment}


\begin{itemize}
  \item Check that no classes unsatisfiable
  \item Check consistency of OCL constraints
  \item Redundency checks (redundent classes, redundent attributes and properties) redundend constraints
\end{itemize}

%-------------------------------------------------------------------------------------

\subsection{Assessing model qualities}

\begin{itemize}
  \item Comparing complexity of URDAD-DSL and UML encodings of URDAD model
    \begin{itemize}
     \item UML model requires much more model elements to encode URDAD semantics.
    \end{itemize}
  \item Completeness checks
(What does completeness mean - That only technical information needs to be provided and that the full requirements
from a business perspective are specified across levels of granularity.)
    \begin{itemize}
     \item all OCL constraints for decision conditions specified
     \item all OCL constraints for pre and post-conditions specified
     \item All required request and result fields all specified either via OCL constraints 
	or by default values.
    \end{itemize}
  \item Check that process addresses all functional requirements and nothing but the functional requirements.
\end{itemize}

%-------------------------------------------------------------------------------------

\subsection{Sufficiency}

An URDAD model is meant to be able to absorb the information required for requirements traceability, implementation mappings and test generation.

Requirements traceability is important for design validation and estimation. Validation includes assessing sufficiency and neccessity, i.e.\ assessing whether all requirements are met and whether all model elements are required. This has to be done across levels of granularity \cite{dick_design_2005}. Ramesh et al\. \cite{ramesh_toward_2001} identify four core traceability link types including (i) \emph{satisfaction links} which represent links between requirements and activities which satisfy them, (ii) \emph{evolutional links} which link change requests and the resulting changes, (iii) \emph{rationale links} which link requirements to the rationale (e.g. goals) which drive them and (iv) \emph{dependency links} which represent links between model elements.

In the URDAD metamodel satisfaction links are represented by \verb+usedToAddress+ links between services and pre- and post-conditions. Evolutionary links are not addressed within the URDAD metamodel and are left to the version control environment. Higher level purpose, goals and rationale links are also not addressed in URDAD. The metamodel does, however, include the \verb+requiredBy+ linkage between requirements and the stakeholders which require them. A requirement around a service may be required by a responsibility domain (i.e.\ a roles) or by another service. Finally, dependency links are included in the model. The dependency is formost between services and service contracts, i.e.\ a particular process design for a service has dependencies on services contracts (not on particular service implementations). Note that the URDAD metamodel does not differentiate between dependency and satisfaction links. Requirements result in dependencies and if the dependencies are available, the requirements are satisfied.




%-------------------------------------------------------------------------------------

\subsection{Usability assessment (potentially an empirical study)}

\begin{itemize}
 \item Have 2 groups doing URDAD modeling, one using UML encoding and UML tools and one using URDAD encoding.
 \item Assess relative productivity, error densities and completeness of the two resultant models 
\end{itemize}

