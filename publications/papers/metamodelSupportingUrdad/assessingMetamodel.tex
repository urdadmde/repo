\section{Assessing the URDAD metamodel}



%-------------------------------------------------------------------------------------

\subsection{Core metamodel integrity assessment}


\begin{itemize}
  \item Check that no classes unsatisfiable
  \item Check consistency of OCL constraints
  \item Redundency checks (redundent classes, redundent attributes and properties) redundend constraints
\end{itemize}

%-------------------------------------------------------------------------------------

\subsection{Assessing model qualities}

\begin{itemize}
  \item Comparing complexity of URDAD-DSL and UML encodings of URDAD model
    \begin{itemize}
     \item UML model requires much more model elements to encode URDAD semantics.
    \end{itemize}
  \item Assessing traceability (\cite[dick_design_2005] - see notes on Zotero entry)
  \item Completeness checks
What does completeness mean - That only technical information needs to be provided and that the full requirements
from a business perspective are specified across levels of granularity.
    \begin{itemize}
     \item all OCL constraints for decision conditions specified
     \item all OCL constraints for pre and post-conditions specified
     \item All required request and result fields all specified either via OCL constraints 
	or by default values.
    \end{itemize}
  \item Check that process addresses all functional requirements and nothing but the functional requirements.
  
\end{itemize}

%-------------------------------------------------------------------------------------

\subsection{Usability assessment (potentially an empirical study)}

\begin{itemize}
 \item Have 2 groups doing URDAD modeling, one using UML encoding and UML tools and one using URDAD encoding.
 \item Assess relative productivity, error densities and completeness of the two resultant models 
\end{itemize}

