\section{Introduction}

Requirements quality is still the core cause of system errors \cite{heck_experiences_2008}. Besides intrinsic uncertainty and misunderstandings of the requirements by business, some of the core factors which contribute to defects in the captures requirements specification include using specification technologies with informal or vagually defined requirements semantics \cite{ferguson_empirical_2006}, specifying requirements only at a single or insufficient levels of granularity with a lot of the lower level detailed business requirements being left to the technical team and the inability to validate requirements for consistency and completeness \cite{}, the inability to test whether a solution fulfills the requirements (non-testable requirements) \cite{bashardoust-tajali_extracting_2008}.

Model-driven engineering (MDE) aims to address some of the above by using modeling languages with defined semantics which enforce a level of consistency across model elements, by requiring that the primary model should be a domain model and not a technical model and by providing a set of tools which support MDE processes. The technology neutral domain model is to be developed by domain experts and not by technical experts. In the case of enterprise system development domain experts are represented by business analysts.

The {\em Object Management Group} (OMG) is a industry standards body which has, in the context of its {\em Model Driven Architecture} (MDA), introduced and standardized a number of technologies which aim to practically facilitate {\em model-driven development} (MDD). These include the {\em Meta-Object Facility} (MOF) for specifying modeling languages, the {\em Unified Modeling Language} (UML) as a candidate for a generic modeling language, the {\em Object Constraint Language} (OCL) used to specify constraints at the metamodel or model levels, {\em Query View Transformations} for model-to-model transformations and (\em Model-To-Text} (M2T) for generation of text artifacts like code.

The {\em Use-Case, Responsibility Driven Analysis and Design} methodology (URDAD) provides is used by 
service-oriented analysis and design methodology which is used by business analysts in industry
 - algorithimic process to analysis and domain model design
 - requirements decomposition across levels of granularity
 - resultant domain model testable and traceable
 
  \cite{bashardoust-tajali_extracting_2008} stress the need for the domain model to be testable.

Requirements quality determined by
  - semantics
  - level of granularity
  - contracts (testability)
  - traceability
  - validatability


UML enconding
  - UML = established std
  - strong tool support
  - widely embraced in industry
but
  - large
  - semantics not clear
  - requires discipline
  - 

To tighten up
 - profile
 - dsl
 - concrete syntax feeding into UML 

DSL adv
  - smaller language
  - lower model complexity
  - well defined semantics
Disadvantages
  - not an established std
  - requires learning
  - tool support must be developed
  - 

In this paper we
  - discuss the metamodel
  - verify the internal integrity of the metamodel
  - discuss the validatability of model instances (consistency, address all functional requirements and nothing but the FR)
  - assess certain metamodel qualities like complexity, traceability, validatability, ...
  - present an example concrete syntax


In the context of OMG's vision of MDE, the {\em Model-Driven Architecture} (MDA), the first model is the 
{\em Computation Independent Model} (CIM) which contains ... 

CIM = BMM + BPM + RM
The first level refinement is the {\em Platform Independent Mode} (PIM)
is the

PIM = interfaces, inputs

URDAD, the {\em  is a methodology used by business analysts for technology neutral requirements analysis and 
business process design\cite{solms_technology_2007} is a methodology which assists domain experts to capture
and validate requirements. The output of the methodology is a requirements model which can be viewed as the 
{\em Platform Independent Model} (PIM) \cite{solms_urdad_2010} of the {\em Model Driven Architecture} (MDA).

