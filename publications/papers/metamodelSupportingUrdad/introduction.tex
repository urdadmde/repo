\section{Introduction}

Requirements quality is still the core cause of system errors \cite{heck_experiences_2008}. Factors which contribute to requirements defects include using specification technologies with informal or vagually defined requirements semantics \cite{ferguson_empirical_2006} show that there is a direct relationship between the expressiveness defects in the language within which requirements are provided and the resultant density of test failures).
  \item the inability to validate requirements for consistency and completeness \cite{}
  \item non-testable requirements \cite{bashardoust-tajali_extracting_2008}
  \item incomplete requirements due to requirements being largely specified at a single level of granularity. 
\end{itemize}

OMG's {\em Model Driven Architecture} (MDA) aim to address this by requiring that the primary model is a technology neutral domain
model. \cite{bashardoust-tajali_extracting_2008} stress the need for the domain model to be testable


MDD aims to address this by moving the modeling into the problem domain .... Discuss CIM/PIM

URDAD is a service-oriented analysis and design methodology which is used by business analysts in industry
 - algorithimic process to analysis and domain model design
 - requirements decomposition across levels of granularity
 - resultant domain model testable and traceable
 

Requirements quality determined by
  - semantics
  - level of granularity
  - contracts (testability)
  - traceability
  - validatability


UML enconding
  - UML = established std
  - strong tool support
  - widely embraced in industry
but
  - large
  - semantics not clear
  - requires discipline
  - 

To tighten up
 - profile
 - dsl
 - concrete syntax feeding into UML 

DSL adv
  - smaller language
  - lower model complexity
  - well defined semantics
Disadvantages
  - not an established std
  - requires learning
  - tool support must be developed
  - 

In this paper we
  - discuss the metamodel
  - verify the internal integrity of the metamodel
  - discuss the validatability of model instances (consistency, address all functional requirements and nothing but the FR)
  - assess certain metamodel qualities like complexity, traceability, validatability, ...
  - present an example concrete syntax


Requirements are still the main cause of system defects (see, for example, \cite{heck_experiences_2008}).

Model-Driven Engineering (MDE) \cite{} aims to address this by shifting the emphasis to the requirements and
requiring that the modeling be done in the problem domain. In the case of enterprise system development
the problem domain is the business domain and the modeling needs to be done by business analysts.

In the context of OMG's vision of MDE, the {\em Model-Driven Architecture} (MDA), the first model is the 
{\em Computation Independent Model} (CIM) which contains ... 

CIM = BMM + BPM + RM
The first level refinement is the {\em Platform Independent Mode} (PIM)
is the

PIM = interfaces, inputs

URDAD, the {\em  is a methodology used by business analysts for technology neutral requirements analysis and 
business process design\cite{solms_technology_2007} is a methodology which assists domain experts to capture
and validate requirements. The output of the methodology is a requirements model which can be viewed as the 
{\em Platform Independent Model} (PIM) \cite{solms_urdad_2010} of the {\em Model Driven Architecture} (MDA).

