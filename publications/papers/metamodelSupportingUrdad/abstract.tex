\abstract{
URDAD is a methodology which is used by domain experts \commentt{There must always be a comma after ``e.g.''}{} (e.g., business analysts) 
to develop requirements models. Historically these requirements models were 
encoded in UML. However, in order to encode an URDAD model in UML, one has to
enforce a strong discipline on the usage of UML. Using UML
requires that business analysts go through the mundane tasks of adding
a range of semantic relationships in order to fully capture the URDAD semantics.
The resultant process is inefficient and error prone. This paper introduces a domain-specific language
supporting the URDAD methodology. We introduce and assess an URDAD metamodel 
which supports URDAD model encodings with significantly lower language and model 
complexity than an URDAD UML profile. Also, model validation for completeness and consistency as well as
URDAD tool development are simplified. We also present simple concrete
text and diagrammatic syntaxes for URDAD, which capture the URDAD semantics in a normalized way. These syntaxes can automatically be derived from the URDAD meta-model.
}
 