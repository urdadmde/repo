\abstract{

URDAD is a methodology which s used by domain experts (e.g. business analysts) 
to develop requirements models. Historically these requirements models were 
encoded in UML. However, in order to encode an URDAD model in UML, one has to
enforce a strong discipline on the usage of UML. Furthermore, using UML
requires that business analysts go through the mundane tasks of adding
a range of semantic relationships in order to fully capture the URDAD semantics.
The resultant process is inefficient and errror prone. This paper presents and assesses
a metamodel supporting the URDAD methodology together with a simple concrete
text syntax for URDAD which captures the URDAD semantics in a normalized way. 
The metamodel enforces the URDAD model structure and requires substantially
less model validation. Furthermore, the resultant model has significantly lower
model complexity than the semantically equivalent UML model simplifying 
URDAD tool development.
}
 