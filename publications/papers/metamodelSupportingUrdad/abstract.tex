\abstract{
URDAD is a methodology which is used by requirements engineering to develop semi-formal requirements models. Historically these requirements models were encoded in UML. However, capturing the requirements model coming out of the URDAD process in UML requires a high level of skill and discipline and includes tedious tasks around inserting certain semantic relationships required by the URDAD method. The resultant process is inefficient and error prone. To enforce model quality for a UML encoding of an URDAD model one needs to specify an URDAD UML profile introducing certain URDAD-specific semantics and extensive OCL constraints which constrain the UML model to a valid, consistent and ultimately complete URDAD model. This paper presents and assesses the alternative approach of introducing a domain-specific language (DSL) supporting the URDAD method and defines a concrete text syntax which can be used to encode an URDAD requirements model. The URDAD-DSL provides simpler language to encode an URDAD model, results in models with lower model complexity, enforces the URDAD model structure, facilitates faster model encoding and results in higher quality models. Model validation for completeness and consistency and URDAD tool development are thus simplified.
}
 