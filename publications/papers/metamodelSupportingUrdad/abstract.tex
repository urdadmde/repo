\abstract{
URDAD is a methodology which is used by requirements engineering to develop semi-formal requirements models. Historically these requirements models were encoded in UML. However, capturing the requirements model coming out of the URDAD process in UML requires a high level of skill and discipline and includes tedious tasks around inserting certain semantic relationships required by the URDAD method. The resultant process is inefficient and error prone. This paper introduces a domain-specific language supporting the URDAD methodology. We introduce and assess an URDAD metamodel which supports URDAD model encodings with significantly lower language and model complexity than an URDAD UML profile. Also, model validation for completeness and consistency as well as URDAD tool development are simplified. We also present simple concrete text and diagrammatic syntaxes for URDAD, which capture the URDAD semantics in a normalized way. These syntaxes can be derived from the URDAD meta-model.
}
 

