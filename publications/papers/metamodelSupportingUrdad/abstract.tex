\abstract{
URDAD is a methodology which is used by domain experts (e.g. business analysts) 
to develop requirements models. Historically these requirements models were 
encoded in UML. However, in order to encode an URDAD model in UML, one has to
enforce a strong discipline on the usage of UML. Using UML
requires that business analysts go through the mundane tasks of adding
a range of semantic relationships in order to fully capture the URDAD semantics.
The resultant process is inefficient and error prone. To enforce model quality
for a UML encoding of an URDAD model one needs to specify an URDAD UML profile introducing
certain URDAD-specific semantics and extensive OCL constraints which constrain the
UML model to a valid, consistent and ultimately complete URDAD model.
This paper presents the alternative approach of introducing a domain-specific language
supporting the URDAD methodology. We introduce and assess an URDAD metamodel 
which supports URDAD model encodings with significantly lower language and model 
complexity. Model validation for completeness and consistency as well as
URDAD tool development are simplified. We also present a simple concrete
text syntax for URDAD which captures the URDAD semantics in a normalized way. 
}
 