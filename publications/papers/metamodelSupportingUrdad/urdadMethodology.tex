\section{Overview of the URDAD methodology \label{sec:urdadMethodology}}

URDAD as a methodology supporting MDA

URDAD is an algorithmic, semi-formal methodology for eliciting and capturing service/use case requirements and technology neutral business process designs\cite{solms_urdad_2010}. The methodology recursively decomposes cohesive units of functional requirements into lower level functional requirements until all requirements are decomposed into functional components which can be sourced from the environment. The core aim is to provide a repeatable engineering process for eliciting and capturing requirements. 

The process is an iterative process, iterating across levels of granularity. At each level of granularity the required functionality is decomposed into cohesive lower level functional units called services. The required logic for assembling the higher level functionality/service from lower level services is specified in a process.

The steps at any level of granularity include
\begin{enumerate}
 \item lower level {functional requirements elicitation}
 
Decompose the higher level functional requirement (service) into lower level functional requirements required by different stake holders.
 \item Specify the services contract for the higher level service including the pre- and post-conditions, data structures for the result and the quality requirements
 \item 
\end{enumerate}

Discuss recursive nature of URDAD, i.e. how URDAD can be used to design itself but refer to quality paper - feed that into quality paper.

