\section{Overview of the URDAD methodology \label{sec:urdadMethodology}}

URDAD is an algorithmic, semi-formal method for eliciting and capturing service/use case requirements and technology neutral business process designs\cite{solms_urdad_2010}. Services are recursively decomposed until the lowest level services which are services sourced from the environment (e.g. operating system, frameworks, off-the shelf solutions and services sourced from external service providers). These low level provided by the environment are treated as atomic services.

URDAD aims to provide a repeatable engineering process for performing the technology neutral requirements analysis and process design around a use case or service. To this end it must be an algorithmic with defined process steps and defined inputs and outputs for each process step. The URDAD process steps include service contract specification, functional/service requirement identification, responsibility allocation and process specification.

The process of performing the requirements elicitation and process design for a level of granularity can be itself be seen as a service which is provided by requirements engineers. It is a recursive service which is repeated for any lower level services for which the requirements elicitation and process specification has not yet been done.

Since higher level services are commonly assembled from lower level services sourced from different domains of responsibility, the requirements elicitation and process design across services of different levels of granularity is not commonly done by a single requirements specialist. Instead URDAD aims to enable requirements specialists across domains of responsibility to contribute to a single requirements model. For example, the business process of processing an insurance claim might be assembled from lower level services like providing the status of the policy, assessing the claim coverage and value, checking funds availability for a policy, settling the claim and that of recuperating losses for a claim. These lower level services touch most of the responsibility domains of an insurer including that of policy contracts, policy accounts, valuation, legal and finance. Thus, whilst the requirements specialist (business analyst) from claims would most probably design the claims process itself, the requirements elicitation and process design for the lower level services would be done by business analysts from departments addressing these different responsibility domains. 

\subsection{Service contract specification}
This step includes the identification of stake holders, the pre- and post-conditions and quality requirements they have for the service. A stake holder is a role player which has an interest in the use case/service and hence may have requirements around it. 

A pre-condition is a condition under which the service may be refused. For each condition one specifies the exception which can be used to communicate that the condition does not hold true. The exception is thus used to notify the service requester that the requested service is not going to be provided because a particular pre-condition for the service is not met. If all preconditions are met, the service must be provided as per contract, otherwise there is an error. 

A post-condition is a condition which must hold true after the service has been provided. Functional testing will test that the service is not refused (no exception thrown) if all pre-conditions are met and that all post-conditions hold true once the service has been provided.

The quality requirements are the non-functional requirements like scalability, performance, reliability, accessibility/integrability, security, auditability and cost requirements. 

\subsection{Functional requirements elicitation}
The functional requirements step determines, for each pre- and post-condition, the required functions/services. For example, to be able to assess the {\em policy is active} pre-condition, one requires a \verb+providePolicyStatus+ service. 

The service requirements refer to service contracts and not specific service implementations, i.e. the requirement is to have any service available which realizes a particular service contract.

Part of the functional requirements step is to see whether multiple service requirements can be abstracted into a single cohesive service (with a single purpose) which is assembled from any of the required services. For example, before one can enroll a candidate for a course one needs Combine services to fix level of granularity in repeatable way.

\subsection{Responsibility allocation}

URDAD has its roots in concepts from Responsibility Driven Design (RDD) \cite{wirfs-brock_object-oriented_1989,wirfs-brock_object_2002}. Services contracts and service designs are assigned to the responsibility domains into which the services fall. This provides the step which prompts the search for service reuse, and, if a required service does not exist

\subsection{Process design}
In the previous steps the lower level services to be used to adress the pre- and post-conditions were identified. In this step one orchestrates a process which, if all pre-conditions are met, yields the result and, if not, aborts the service delivery raising the exception which has been associated with the pre-condition which has not been met. The process will incorporate all services







