\section{Guidelines for the Implementation Mapping}

The following are some guidelines for the implementation mapping of specific URDAD model artifacts. The guidelines have a bias towards a typical modern object-oriented language like \verb+Java+ or \verb+C#+, but the concepts can be easily mapped onto other technologies.

\subsection{Responsibility domains}

Responsibility domains are meant to be mapped onto packages or name spaces as well as onto an interface with the name of the package with the capitalization converted to the convention of the target technology. 

\subsection{Service contracts}

Service contracts are meant to be mapped onto service/method declarations on interfaces or pure abstract classes which represent the services contract for a responsibility domain. The mapping of a contract for a specific service is meant to be done as follows:
\begin{itemize}
  \item The request object is an instance of a request class mapping onto a single input parameter is an instance of the request class,
  \item Both, the request and response/result classes are service specific and should, if the target technology supports it, be mapped onto static nested classes of the interface.
  \item If the target technology supports notifiable exceptions (e.g. Java), the exceptions raised for the preconditions of the service should be included in the throws clause of the service.
  \item If the target includes unit test generation, then the unit test for a service should test that
    \begin{itemize}
      \item the service is provided and not refused (no exception is thrown) if all pre-conditions are met,
      \item all post-conditions are satisfied after the service has been provided.
    \end{itemize}
\end{itemize}

\subsection{Service and process specification}

The concrete services in the metamodel contain the functional requirements (the lower level services from which the sevice is to be assembled) as well as the technology-neutral process specification which specifies how the process is orchestrated across these lower level services. There are service request activities, control structures, activities which create and manipulate local process variables and termination activities which either returning the result or notify the user that the requested service is not provided because a particular precondition for the service is not met by raising the exception associated with that precondition. 

The technology neutral design does not specify how references to the service providers of the lower level services are obtained. For modern implementation technologies this would typically be via dependency injection. The implementation mapping for other technologies might need to introduce a service provider registry and generate the approapriate service provider lookup code.

The process specification includes process variables which are accessible within the process and are maintained for the duration of the process. The inputs and outputs of the process (the request ond result objects for the core service itself) ae also process variables which can be queried and manipulated within the process.

\subsection{Data structures}

The URDAD data structure specification is essentially the same as in other object-oriented technologies supporting classes with attributes, association to other classes, inheritence and abstract classes. The different types of association relationships are meant to be implemented as follows:
\begin{itemize}
  \item Plain \textit{association} relationships are mapped onto simple references or pointers.
  \item \textit{Aggregation} relationships are implemented the same as plain association relationships except in the case where state change notification is to be supported. In these cases one also needs to ensure that the aggregate object listens to state change events from its components and that it issues the appropriate state change event to its observers.
  \item \textit{Composition} should be implemented as exclusive ownership, i.e. that the component is part of the owner and not part of any other object and should not survive the owner. Optionally the implementation mapping can implement full encapsulation for composition, i.e. that the component can only be accessed through the owner.
\end{itemize}


\subsection{Unit tests}

Unit tests should be developed for service contracts, not for concrete service definitions. They should only test the post-conditions for the current level of granularity, assuming that the lower level service providers fulfill their respective service contracts. In particular, they should test that
\begin{itemize}
  \item that the service is not refused if any of the preconditions do not hold true, i.e.\ that no exception is thrown, and
  \item that, when the service is provided (no exception thrown), all postconditions hold true after the service has been provided.
\end{itemize}

