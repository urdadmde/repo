\section{Evaluation guidelines}



\subsection{Mapping preferences}

When evaluating implementation mappings, we would recommend the following preferences:
\begin{itemize}
  \item Declarative bi-directional mappings should be preferred over operational/algorithmic uni-directional mappings. 
  \item Standard mapping technologies like QVT relational should be preferred over custom mapping technologies.
  \item More simple, decoupled mapping rules should be preferred over fewer, complex rules.
  \item Standard target technologies commonly used for enterprise systems (e.g.\ JavaEE, SOA, Microsoft.Net) should be preferred over technologies seldom used for enterprise systems.
\end{itemize}

\subsection{Completeness measure}
The implementation should be tested against the tests developed or generated for the service contracts. 

Completeness checks should includes
\begin{itemize}
  \item whether the interfaces for the service contracts have been generated,
 
\end{itemize}


   * What are the criteria for evaluating the submitted solutions to the case?
    
      Correctness test: which are the reference input/ouput documents (models/graphs) and how should they be used? Ideally, a case description includes a testsuite, as well as a test driver
            (The test driver can be an online web service, or a local script that can be deployed in Online demo in SHARE.)

          Which transformation tool-related features are important and how can they be classified?
            (e.g., formal analysis of the transformation program, rule debugging support, ...)

          How to measure the quality of submitted solutions, at the design level?
            (e.g., measure the number of rules, the conciseness of rules, ...)

\subsection{Test compliance}


\subsection{Systematic Evaluation Guidelines}
    * How can the solutions be evaluated systematically using information technology?
      Please provide one of the following:
          o a simple spreadsheet (an evaluation form that can be aggregated easily into an overview similar to this example table from 2010),
          o a so-called ?classification scheme? in ResearchR.org (or a similar web 2.0 platform.)
