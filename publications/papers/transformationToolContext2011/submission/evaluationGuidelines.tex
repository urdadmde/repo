\section{Evaluation guidelines}

This section discusses some guidelines which we recommend to evaluate different transformation solutions.

\subsection{Mapping preferences}

When evaluating implementation mappings, we would recommend the following preferences, i.e. that transformations which use the preferred mapping strategies are rated higher than those which do not.
\begin{itemize}
  \item Declarative bi-directional mappings should be preferred over operational/algorithmic uni-directional mappings. 
  \item Standard mapping technologies like QVT relational should be preferred over custom mapping technologies.
  \item More simple, decoupled mapping rules should be preferred over fewer, complex rules.
  \item Standard target technologies commonly used for enterprise systems (e.g.\ JavaEE, SOA, Microsoft.Net) should be preferred over technologies seldom used for enterprise systems.
\end{itemize}

\subsection{Completeness measure}
The implementation should be tested against the tests developed or generated for the service contracts. 

Completeness checks should include
\begin{itemize}
  \item whether the interfaces for the service contracts have been generated and to what extend they comply with the implementation guidelines,
  \item the completeness of the data structure specifications and the degree to which they confirm to the implementation guidelines,
  \item the completeness of the service implementations including whether
    \begin{itemize}
     \item the processes have been mapped onto method bodies,
     \item whether the request and result data structures are populated in the generated process specification,
     \item whether the conditionals for conditional flows and while loops are generated,
    \end{itemize}
  \item whether unit tests are generated from the service contracts and the completeness of the unit tests in the form of degree coverage of pre- and post-conditions.
\end{itemize}

\subsection{Correctness tests}

The correctness can be tested by assessing whether the service is realized in such a way that
\begin{itemize}
  \item if a particular precondition is not met, that the service is aborted raising that exception which was assigned in the model to that particular precondition.
  \item if all preconditions are met that the result object is obtained and all post-conditions evaluate to true after the service has been provided.
\end{itemize}

%TODO: Add details
%       Correctness test: which are the reference input/ouput documents (models/graphs) and how should they be used? Ideally, a case description includes a testsuite, as well as a test driver
%             (The test driver can be an online web service, or a local script that can be deployed in Online demo in SHARE.)
% 
%           How to measure the quality of submitted solutions, at the design level?
%             (e.g., measure the number of rules, the conciseness of rules, ...)


%\subsection{Systematic Evaluation Guidelines}
%TODO: Add details

%     * How can the solutions be evaluated systematically using information technology?
%       Please provide one of the following:
%           a simple spreadsheet (an evaluation form that can be aggregated easily into an overview similar to this example table from 2010),
%           a so-called ?classification scheme? in ResearchR.org (or a similar web 2.0 platform.)
