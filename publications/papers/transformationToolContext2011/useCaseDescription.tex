\section{Use case description}

The aim was to take a typical business service for which the requirements and h

\subsection{Subject modeled}

The input modeling language is the URDAD DSL. The metamodel specification and the example model are provided.

Since the model is meant to be a technology neutral model, the output modeling language is left up to the team and would correspond to the target technology chosen. This could be, for example, a model for either a JavaEE, Spring or SOA based implementation.

\subsection{Purpose of model}

The purpose of the model is to provide the requirements and technology neutral process design of a typical business use case. 
technology neutral. Teams which aim to perform the model transformation should select the target model for corresponding to their chosen

typical business use case
            (what are they typically used for from a larger perspective than the proposed case study?)

scope of model

\subsection{Use case variation points}
      (divide up your case in core characteristics and extensions)

diff technologies

diagram generation
\begin{itemize}
      \item common quality requirements like security, auditability, reliability and scalability may be applied to this model.
      \item it is suited to be implemented across different architectures and technologies (e.g., JavaEE, SOA, ...)
\end{itemize}