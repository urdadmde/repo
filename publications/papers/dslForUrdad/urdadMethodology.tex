\section{Overview of the URDAD methodology \label{sec:urdadMethodology}}

URDAD is a semi-formal methodology for eliciting and capturing requirements in the form of technology neutral business process designs \cite{solms_urdad_2010}. Business processes are specified as services with corresponding service contracts and process designs. Services are recursively decomposed across levels of granularity, where the lowest level (atomic) services are sourced from the environment (e.g., operating system, frameworks, off-the shelf solutions, and external service providers). 

The URDAD methodology stipulates a repeatable engineering process, which comprises of the following iterative steps: (1) service contract specification, (2) responsibility allocation, and (3) process design. Following a top-down approach, higher level services are assembled from lower level services, which can be sourced from different domains of responsibility. URDAD enables requirements specialists across different domains of responsibility to contribute to a common requirements model. For example, a service responsible for processing an insurance claim might be assembled from lower level services providing policy status, claim coverage assessment, and settlement value. These services may be sourced from different responsibility domains such as policy contracts, policy accounts, valuation, legal, and finance. 

\subsection{Service contract specification}

The URDAD methodology facilitates the incremental refinement of service requirements across different levels of granularity. Service requirements are encapsulated within a services contract. The specification of service contract includes the identification of stakeholders, the functional and quality requirements that they have for the service, and the data structures required for the service's request and result objects. A stakeholder may be a role or another service.

Functional service requirements can be pre- or post-conditions. If all preconditions of a service are met then the service must be provided. The specification of each pre-condition includes the exception that is raised to notify service requesters that the service will not be executed as a result of the prerequisite condition not holding true. A post-condition is a condition which must hold true after the service has been provided. Non-functional service requirements include, for example, scalability, performance, reliability, accessibility, security, auditability, and cost requirements. 

\subsection{Responsibility allocation}

During the responsibility allocation step the lower level services that are used to assemble a higher level service are identified by their ability to address the higher level service's individual functional requirements. Many of URDAD's concepts originate from Responsibility Driven Design (RDD) \cite{wirfs-brock_object-oriented_1989,wirfs-brock_object_2002}. Each service contract and its corresponding service realization are assigned to a responsibility domain. This prompts domain experts to search within an appropriate responsibility domain for existing services that can be used to address functional requirements, i.e.\ to search for \emph{reuse}. The URDAD methodology does not prescribe an order in which services are to be specified. Thus, higher level services may initially depend on lower level services that are not fully specified yet.

Service contracts represent requirements on a specific level of granularity. The complete requirements for a service are accrued by the accumulation of its requirements and those of its required lower level services. The hierarchical organization and specification of service contracts has proved to make requirements more manageable: each lower level service has its own pre-conditions, post-conditions, and process specification, linking it to the requirements at the next lower level of granularity. Additional levels of granularity can be established by coalescing several service requirements into a single cohesive service to improve the understandability and reusability of processes and their requirements.

\subsection{Process design}

Processes are designed using lower level services identified in the responsibility allocation step. A process specification includes 1) lower level service requests to address pre- and post-conditions, 2) activities to assemble the request objects for these services, 3) activities for handling exceptions raised by lower level services, 4) activities for raising exceptions, and 5) the activity of constructing and returning the result. Each path through a process graph must either end with returning the result or with raising an exception.