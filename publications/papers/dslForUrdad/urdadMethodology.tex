\section{Overview of URDAD 
\label{sec:urdadMethodology}}

URDAD is currently a \emph{semi}-formal methodology for requirements elicitation in the form of layered service contracts and processes  \cite{solms_urdad_2010}; its formalisation is the topic of ongoing research. \emph{Business processes} are specified as services with corresponding service contracts and process flows. The URDAD methodology stipulates a repeatable engineering workflow on the basis of the following iterated steps: service contract specification, responsibility allocation, and process design. The methodology envisages requirements specialists across responsibility domains to contribute to a single requirements model.

\paragraph{Service Contract Specification.} The URDAD methodology facilitates the incremental refinement of service requirements across different levels of granularity. Service requirements are encapsulated within a service contract. The specification of a service contract includes the identification of stake holders, the functional and quality requirements, and the data structures required for the service's request and result objects. A stake holder may be a role, or another service.

Functional service requirements can be expressed in ter\-ms of pre- or post-conditions. If all preconditions of a service are met, the service must be provided. The specification of each pre-condition includes an exception type which must be raised to notify the service requester that the requested service is refused due to the associated pre-condition not being met. A specified post-condition must hold true after the service has been provided. Non-functio\-nal service requirements stipulate qualities such as scalability, efficiency, reliability, accessibility, security, etc.

\paragraph{Responsibility Allocation.} During the responsibility allocation step the lower level services that are used to assemble a higher level service are identified by their ability to address the higher level service's functional requirements. Many of URDAD's concepts originate from Responsibility Driven Design (RDD) \cite{wirfs-brock_object-oriented_1989,wirfs-brock_object_2002}. Each service contract and its corresponding service are assigned to a responsibility domain. This prompts domain experts to search within an appropriate responsibility domain for existing services that can be re-used to implement functional requirements.

Service contracts represent requirements on a specific level of granularity. The complete requirements for a service are accrued by the accumulation of its requirements and those of its required lower level services. The hierarchical decomposition of service contracts has made requirements engineering better manageable. Additional levels of granularity can be opened by coalescing several service requirements into a single cohesive super-service. Such abstraction from details (principle of information hiding) reduces the intellectual complexity, improves the understandability of requirements elicitation tasks and op\-ens further reuse opportunities. 

\paragraph{Process Design.} Computational processes are specified using services id\-entified in the responsibility allocation step, i.e.\ processes are `orchestrated' across lower level services used to realize the pre- and post-conditions of the service. A process specification is assembled from standard control logic for sequential, concurrent and conditional activities with activities either  constructing and request objects or the object representing the computational output, requesting lower level services, handling exceptions raised by lower level services or raising an exception or returning a result to the user of the service. Each path through a process graph must either end with an output, or with raising an exception associated with a precondition of the service.

