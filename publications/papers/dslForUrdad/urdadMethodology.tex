\section{Overview of the URDAD methodology \label{sec:urdadMethodology}}

URDAD is a semi-formal methodology for eliciting and capturing requirements in the form of technology neutral business process designs\cite{solms_urdad_2010}. Business processes are specified as services with corresponding service contracts and process designs. Services are recursively decomposed across levels of granularity, where the lowest level (atomic) services are sourced from the environment (e.g. operating system, frameworks, off-the shelf solutions and from external service providers). 

URDAD aims to provide a repeatable engineering process. To this end it must be algorithmic with defined process steps and defined inputs and outputs for each step. URDAD's steps include (1) service contract specification, (2) responsibility allocation and (3) process specification. These steps are repeated for any lower level services for which the requirements elicitation and process specification has not yet been done.

Higher level services are commonly assembled from lower level services sourced from different domains of responsibility. URDAD enables requirements specialists across domains of responsibility to contribute to a single requirements model. For example, a service responsible for processing an insurance claim might be assembled from lower level services providing policy status, claim coverage assessment and settlement value. These services may be sourced from different responsibility domains such as policy contracts, policy accounts, valuation, legal and finance. 

\subsection{Service contract specification}

Within URDAD one incrementally refines service requirements across levels of granularity with the service requirements at any particular level of granularity encapsulated within a services contract. 

This step includes the identification of stakeholders, the functional and quality requirements they have for the service, and the data structures required for the service's request and result objects. A stakeholder may be a role or another service. Functional requirements may either be pre-conditions or  post-conditions. A pre-condition is a condition under which the service may be refused. For each pre-condition one specifies an exception. The exception is used to notify the service requester that the requested service is not going to be provided because a particular pre-condition for the service has not been met. If all preconditions are met, the service must be provided, otherwise it constitutes an error. A post-condition is a condition which must hold true after the service has been provided. Functional testing will test that the service is not refused (no exception thrown) if all pre-conditions are met and that all post-conditions hold true, once the service has been provided. Quality requirements are the non-functional requirements like scalability, performance, reliability, accessibility/integrability, security, auditability and cost requirements. 

\subsection{Responsibility allocation}

During the responsibility allocation step the lower level services that will be used to assemble a higher level service are identified by their ability to address the higher level service's individual pre-conditions and post-conditions. Many of URDAD's concepts originate from Responsibility Driven Design (RDD) \cite{wirfs-brock_object-oriented_1989,wirfs-brock_object_2002}. Each service contract and its corresponding service realisation are assigned to a responsibility domain. This prompts domain experts to search within an appropriate responsibility domain for existing services that can be used to address functional requirements, i.e.\ to search for \emph{reuse}.

Note that there may be a dependency on service contracts for which requirements have not yet been specified. The responsibility allocation step thus purely identifies the functional units required to address the pre-conditions and post-conditions. The detailed functional requirements for these lower level services will be elicited and documented during the analysis phase for the next level of granularity.  

Service contracts represent requirements for a specific level of granularity. The complete requirements for a service is ultimately provided by the accumulation of its requirements and the requirements associated with the lower level services from which this service is assembled. Instead of having service requirements at a single level of granularity with complex pre-conditions and post-conditions, requirements are made more manageable, since each lower level service has its own pre-conditions, post-conditions and process specification, linking it to the requirements at the next lower level of granularity. By identifying services contracts, one fixes levels of granularity, Part of the responsibility allocation step is to identify whether multiple service requirements can be abstracted into a single cohesive service (with a single purpose) which is assembled from any of the required services. Introducing additional levels of granularity improves both, the understandability of requirements and processes and the reusability of services.

%========================================

\subsection{Process design}

During the process design step of URDAD, one assembles a process using the lower level services identified in the responsibility allocation step. The process includes (1) the service request activities requesting the lower level services used to address the pre-conditions and post-conditions, (2) activities to assemble the request objects for these services, (3) activities for handling exceptions raised by lower level services, (4) activities for raising an exception and (5) the activity of constructing and returning the result. Each path through the process graph must end with either returning the result or with raising an exception.