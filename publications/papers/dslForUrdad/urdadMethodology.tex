\section{Overview of the URDAD methodology \label{sec:urdadMethodology}}

URDAD is a semi-formal method for eliciting and capturing service/use case requirements and technology neutral business process designs\cite{solms_urdad_2010}. Service requirements represented by services contracts and process designs are recursively decomposed across levels of granularity with the lowest level services sourced from the environment (e.g. operating system, frameworks, off-the shelf solutions and services sourced from external service providers). These low level services provided by the environment are treated as atomic services.

URDAD aims to provide a repeatable engineering process for performing the technology neutral requirements analysis and process design around a service. To this end it must be an algorithmic with defined process steps and defined inputs and outputs for each process step. The URDAD process steps include (1) service contract specification, (2) responsibility allocation and (3) process specification. These steps are repeated across levels of granularity, i.e.\ for the lower level services required for the higher level service.

The process of performing the requirements elicitation and process design for a level of granularity can be itself be seen as a service which is provided by requirements engineers. It is a recursive service which is repeated for any lower level services for which the requirements elicitation and process specification has not yet been done.

Since higher level services are commonly assembled from lower level services sourced from different domains of responsibility, the requirements elicitation and process design across services of different levels of granularity is not commonly done by a single requirements specialist. Instead URDAD aims to enable requirements specialists across domains of responsibility to contribute to a single requirements model. For example, the business process of processing an insurance claim might be assembled from lower level services like providing the status of the policy, assessing the claim coverage and value, settling the claim. These lower level services touch most of the responsibility domains of an insurer including that of policy contracts, policy accounts, valuation, legal and finance. Thus, whilst the requirements specialist (business analyst) from claims would most probably do the requirements elicitation and process design design for the claims process service, the requirements elicitation and process design for the lower level services would be done by business analysts from departments addressing these different responsibility domains like policies, valuations, and finance.

\subsection{Service contract specification}

Within URDAD one incrementally refines service requirements across levels of granularity with the service requirements at any particular level of granularity encapsulated within a services contract.

This step includes the identification of stake holders, the pre- and post-conditions and quality requirements they have for the service, and the  data structure requirements for the request and result classes of the service. A stake holder may be a role or another service. A pre-condition is a condition under which the service may be refused. For each condition one specifies the exception which can be used to communicate that the condition does not hold true. The exception is used to notify the service requester that the requested service is not going to be provided because a particular pre-condition for the service is not met. If all preconditions are met, the service must be provided as per contract, otherwise there is an error. A post-condition is a condition which must hold true after the service has been provided. Functional testing will test that the service is not refused (no exception thrown) if all pre-conditions are met and that all post-conditions hold true once the service has been provided. The quality requirements are the non-functional requirements like scalability, performance, reliability, accessibility/integrability, security, auditability and cost requirements. 

\subsection{Responsibility allocation}

During the responsibility allocation step one assigns the responsibilities of addressing the individual post-conditions and those of either checking a precondition or obtaining the required information in order to check a pre-condition to lower level services across. This identifies the services from which the the higher level service is assembled.

URDAD has its roots in concepts from Responsibility Driven Design (RDD) \cite{wirfs-brock_object-oriented_1989,wirfs-brock_object_2002}. Each services contract and ultimately also the service which realize the services contract is assigned to a responsibility domains. This prompts the domain expert to search within the appropriate responsibility domain for any existing services which can be used to assess the functional requirements, i.e.\ to search for \emph{reuse}.

Note that the dependency is on services contracts for which the requirements details (pre-conditions, post-conditions, quality requirements and data structures for the inputs and outputs) may not yet been specified. The responsibility allocation step thus purely identifies the functional units required to address the pre- and post-conditions. The detailed functional requirements for these lower level services will be elicited and documented during the analysis phase for the next level of granularity.  

Services contracts represent represent service requirements for a specific level of granularity or abstraction. The complete, fully detailed functional requirements for a service is ultimately provided by the sum total of the functional requirements of the service itself and the recursively lower level services from which the higher level service is assembled. By identifying services contracts, one fixes levels of granularity, Part of the responsibility allocation step is to inspect whether multiple service requirements can be abstracted into a single cohesive service (with a single purpose) which is assembled from any of the required services. Introducing additional levels of granularity improves both, the understandability of requirements and processes and the reusability of services.

%========================================

\subsection{Process design}

During the process design step of URDAD, one assembles a process across the lower level services identified in the responsibility allocation step. The process includes (1) the service request activities requesting the lower level services used to address the pre- and post-conditions, (2) activities to assemble the service requests for these services as well as the result object, (3) activities for handling an exception raised by a lower level service, (4) activities for raising an exception and (5) the activity of returning the result. Each path through the process graph must end with either returning the result or with raising an exception associated with one of the pre-conditions of the service.

URDAD enforces the decoupling of services via services contracts, i.e.\ a service may not have a dependency on another concrete service with defined process, but only on a services contract specifying the service requirements.


