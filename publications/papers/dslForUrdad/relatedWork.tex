\section{Related work \label{sec:relatedWork}}

The URDAD methodology provides a services-oriented methodology for generating a semi-formal analysis and design model representing MDA's PIM and supporting test and implementation generation. \cite{iacob_model-driven_2008} discuss an alternative approach. Business Rules are specified using OMG's {\em Semantics for Business Vocabulary and Rules} (SBVR) to service specification and orchestration and BPEL process specifications are generated using MDA tools. \todo{re-read paper}

The URDAD DSL allows for the specification of textual and graphical grammars through which the URDAD model is populated. An alternative approach is to define a separate metamodel for the use case narrative and to transform the narrative model requirements model \cite{hoffmann_towards_2009,osis_transforming_2010}. This approach introduces the complexities of having to transform from the narrative to the UML model and requires extensive consistency checks between the narrative and the UML models.

\cite{asnina_computation_2010} stress the need of modelling in the problem domain as well the benefits of accumulating requirements within a single model. Services are grouped into feature sets which are related to responsibility domains. Functional requirements are decomposed across levels of granularity and higher level processes are orchestrated across lower level services. They define the notion of functionals with cause and effect which can be related to the concept of a services contract. In addition they provide a {\em topological functional model} (TFM) for mapping technology neutral service requirements onto available concrete services pool. The TFM is independent of the modelling technique and can be applied to an URDAD model.

In the {\em Requirements Driven Design Automation} methodology (RDDA) \cite{cardei_model_2008} one encodes requirements specifications in SYSML diagrams. The SYSML model is enriched with semantic descriptions after which the model is transformed to the {\em One Pass to Production} (OPP) design language, the ODL. ODL is an OWL based ontology from which the requirements are validated for consistency and completeness. The approach is, however, structure focused with little emphasis on services contracts and recursive orchestration of higher level services from lower level services.

\todo{Dawid Loubser: Please write OWL-S paragraph}
