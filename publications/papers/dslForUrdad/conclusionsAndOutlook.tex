\section{Conclusions and outlook \label{sec:conclusionsAndOutlook}}


URDAD = methodology for services oriented analysis and design - this introduces limitations around events, flexibility, choreography, ...., cooperative agents



URDAD DSL provides simpler language for URDAD - improves accuracy, clarifies semantics, supports URDAD process.

Whilst UML+OCL not sufficient for testable service contract specification within services-oriented approach, URDAD DSL is

We expect that the URDAD DSL models provide a simpler and more complete framework for implementation and test generation, but this needs to be verified

Whilst a textual syntax has been specified its suitability for requirements engineers needs to be verified and the usability benefits of an URDAD DSL approach above a URDAD UML approach need to be empirically studied.


There is a lot of future work to be done. A critical aspect for the success of the URDAD DSL is the definition of a usable graphical syntax and the development of corresponding modelling tools which makes the language accessible to requirements engineers. Once this has been done one can perform an empirical usability assessment of the URDAD-DSL by having two groups of domain experts with similar skills levels doing URDAD modelling, one using UML encoding and UML tools and one using URDAD encoding. One can then assess the relative productivity of the two groups and the error densities and completeness of the UML and URDAD-DSL encoded URDAD models.

What have we
  - metamodel supports methodlogy and contains required semantics/model constructs
  - solid testable contract specification
  - usable textual syntax
  - consistent metamodel
  - clean, direct semantics
  - simpler
  - traceability, complexity
  - language aware editor
  - validatable models

short-comings
  - constraints no longer purely declarative
  - not suited for events centric and autochoregraphy type systems
  - textual syntax still too technical

Outstanding work
  - graphical syntax 
  - usabiltiy assessment
  - model transformation assess
  - complete metamodel constraints specifications
  - tool support
  