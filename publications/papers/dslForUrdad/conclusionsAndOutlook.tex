\section{Conclusions and Future Work \label{sec:conclusionsAndOutlook}}

We presented in this paper a DSL supporting the model constructs required by the URDAD methodology. The URDAD DSL is substantially simpler than UML and provides better support for service contract specification. Its metamodel has been shown to be consistent. URDAD DSL models support traceability and have notably reduced design complexity. We were able to specify a textual grammar and generate useful language-aware editors and parsers for the practitioner.

The URDAD methodology and DSL represent a service-oriented approach that is not very suitable for event-centric systems. Moreover, the current textual syntax is still too technical for many industrial practitioners. A critical aspect for the success of the URDAD DSL is the definition of a usable graphical syntax and the development of corresponding modelling tools which makes the language more accessible for industrial practitioners. Such tool-support is currently in development. Once this has been done, we plan to conduct an empirical usability and productivity assessment of the URDAD DSL (compared to an URDAD UML) approach. The derivation of test cases from URDAD DSL specifications is also planned.
