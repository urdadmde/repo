\begin{abstract}
Use-Case Responsibility-Driven Analysis and Design (URDAD) is a services-oriented software process methodology. It is 
used by requirements engineers to develop technology-neutral, semi-formal platform-independent models (PIM) within the 
OMG's MDA. In the past, URDAD models were denoted in UML. However, that was tedious and error-prone. The resulting 
models were often of rather poor quality. In this paper we introduce and discuss a new Domain-Specific Language 
(DSL) for URDAD. Its meta model is consistent and satisfiable. We show that URDAD DSL specifications are simpler and provide allow for more complete services contract specification than their corresponding UML expressions. They also enable traceability and test case generation. 
\end{abstract}
 
 