\begin{abstract}
Use-Case, Responsibility Driven Analysis and Design (URDAD) is a services-oriented methodology. It is used by requirements engineers to develop technology-neutral, semi-formal analysis and design models representing the PIM within OMG's MDA. Historically, these models were encoded in UML. However, capturing the concepts required by URDAD in UML demands a high level of effort, skill and discipline and results in high model inconsistency risks. This paper introduces and assesses a Domain-Specific Language (DSL) for URDAD. We define a metamodel which formalizes the URDAD modelling constructs and a textual syntax to populate URDAD models.
The URDAD metamodel is shown to be consistent and satisfiable. We demonstrate that models represented in the URDAD-DSL have lower language and model complexity than corresponding UML models and that they facilitate traceability, implementation mapping and test generation. 
\end{abstract}
 
 