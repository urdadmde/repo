\section{Introduction}

Poor requirements quality is still the main cause of system errors \cite{heck_experiences_2008,_strategies_2011}. There are a number of factors, besides the intrinsic requirements uncertainty of domain experts or requirements specialists which potentially contribute to requirements defects. These include specification techniques with only informal or vaguely defined requirements semantics \cite{ferguson_empirical_2006}. Specifying requirements at insufficient levels of granularity leaves developers to make presumptions about lower level requirements \cite{espana_evaluating_2009, getBetterReference}. Also, there is only very limited support for the validation of requirements for consistency and completeness. Moreover, support for the automatic extraction of executable test cases from requirements specifications is spotty \cite{bashardoust-tajali_extracting_2008}.

Model-Driven Engineering (MDE) \cite{schmidt_model_2006} aims to address some of these issues by 1) using modeling languages with well defined semantics, 2) by requiring primary models to be domain models and not technical models \cite{asnina_computation_2010}, and 3) by providing tool support for MDE processes. As a result, technology-neutral domain models are developed by requirements specialists and not by technical experts \cite{asnina_computation_2010}. In the case of enterprise system development, requirements specialists are represented by business analysts.

URDAD, the {\em Use-Case, Responsibility-Driven Analysis and Design} \cite{fritz_solms_technology_2007} methodology supports the MDE approach \cite{solms_urdad_2010}. It is used by requirements specialists (i.e.\ business analysts) to develop and validate technology neutral requirements models. URDAD models are {\em Platform Independent Models} (PIM) in the {\em Model Driven Architecture} (MDA) \cite{solms_urdad_2010}.

URDAD is a semi-formal services-oriented requirements analysis and tech\-no\-logy-neutral process design methodology. For each level of granularity the method generates a testable services contract and a service which realizes the services contract with a particular orchestration of lower level services used to realize the functional requirements of the services contract. The result is a functional requirements tree with the levels in the requirements tree decoupled through services contracts.

The URDAD methodology is independent of the technology used to encode the modeling constructs of an URDAD model. There are several OMG/MDA standards-based options for encoding URDAD models.

Traditionally, domain experts used the Unified Model Language (UML) to encode URDAD models. UML is a natural choice due to its wide adoption in industry and its comprehensive tool support. In this approach the burden to only use the URDAD subset of UML and to obey URDAD model semantics is put on the domain experts. One can develop an URDAD profile containing a set OCL-based metamodel constraints constraining the UML model to an URDAD model, but the bulk and complexity of these constraints impose significant challenges. In practice, the quality of UML encoded URDAD models have been generally so low that model transformations were not feasible, requiring manual interpretation by developers.

In this paper, we will present a domain-specific language (DSL) for the domain of technology-neutral, services-oriented requirements modeling. The URDAD DSL is based on a MOF/EMOF metamodel, which is amenable to the comprehensive MDA tool suite for model-to-model and model-to-text transformations and the specification and generation of concrete textual and diagrammatic syntaxes and tools supporting these \cite{gronback_model_2008}. We will analyze the modeling constructs required by URDAD, elucidate and critically assess the URDAD metamodel and propose a concrete textual syntax for an URDAD DSL. A Description Logics (DL) based representation of the URDAD metamodel will be derived from the MOF/EMOF metamodel in order to show its consistency and satisfiability.

The URDAD DSL has two main advantages compared to the use of an URDAD UML profile. The language is significantly simpler than the UML and, with appropriate tools support, is expected to simplify the process through which domain experts construct analysis and technology neutral design models. The DSL largely enforces the structure of an URDAD model, requiring a much smaller and simpler set of metamodel constraints facilitating model validation.

The paper is structured as follows. In Section \ref{sec:urdadMethodology}, we give a compact overview of the URDAD methodology. Section \ref{sec:metamodel} introduces the URDAD metamodel, its core concepts and the rationale behind it. A concrete textual syntax for the URDAD DSL is presented and exemplified in Section \ref{}. In Section \ref{sec:metamodelAssessment} we discuss the verification of the internal integrity of the metamodel, the ability to validate URDAD models for completeness and consistency, and assess certain metamodel qualities such as complexity, traceability and validatability. Our approach is put into context with related work in Section \ref{sec:relatedWork}. Section \ref{sec:conclusionsAndOutlook} concludes and presents an outlook on future work.
