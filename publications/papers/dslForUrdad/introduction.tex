\section{Introduction}
Insufficiency in requirements engineering is still regarded as a root cause of poor software quality. This is due to various factors, both human and technological, including vague specification languages with only informally defined semantics. Insufficient language support for \emph{layered} specifications (i.e., decompositional system descriptions at different levels of granularity), leads software developers to making wrong presumptions about lower level requirements \cite{espana_evaluating_2009}. Tool support for the validation of requirements specifications, or for the automatic extraction of test cases from them, is also still weak \cite{bashardoust-tajali_extracting_2008}.

Model-Driven Engineering (MDE) \cite{schmidt_model_2006} aims at solving some of those problems by using modelling languages with well defined semantics, by requiring primary models to be domain models, not technical models \cite{asnina_computation_2010} and by providing tool support for MDE processes. Consequently, technology-neutral domain models are developed by requirements specialists, not by technical experts \cite{asnina_computation_2010}.

\emph{URDAD}, the Use-Case Responsibility-Driven Analysis and Design methodology \cite{fritz_solms_technology_2007} supports MDE in a service-oriented way \cite{solms_urdad_2010}. It is used by requirements specialists to develop and validate technology-neutral requirements models. URDAD models are thus Platform-Independent Models (PIM) in the Model-Driven Architecture (MDA) context \cite{solms_urdad_2010}. For each level of granularity the method leads to testable service contracts and for non-leaf services a technology neutral process realizing the service contract through the use of lower level services. Higher-level services are thus a functional composition of lower-level services, similar to the classical DFD technique \cite{demarco_tom_structured_1978}, with the levels of granularity decoupled through service contracts.

Requirements engineers have traditionally used the Unified Model Language (UML) to encode URDAD models. UML was a reasonable choice for this purpose because of its tool-supported use in the software industry. However, UML is an object oriented modeling language which is not conceptually aligned with a service oriented approach where stateless services are assembled form lower level stateless services. On the one UML allows for the specification of requirements models which would not satisfy the constraints of an URDAD model or a services oriented requirements specification in general. On the other sidethe UML does not explicitly support many of the concepts required by URDAD. For example, the concept of a responsibility domain, a stakeholder, and even that of a service contract are not explicitly supported. A specific UML \emph{profile} could be used to restrict a UML model to the structure and constraints of an URDAD model and could also introduce explicitly concepts required by URDAD. In practice, however, such a UML profile would require an excessive number of metamodel constraints enforcing that the model complies structurally to the constraints of an URDAD model.

An alternative approach would be to have a object-oriented requirements specification with a mapping to a services oriented design model. This is certainly a feasible approach which has been

In this paper we present a new domain-specific language (DSL) for the domain of technology-neutral service-orien\-ted requirements modelling. Our new URDAD DSL is described in terms of a MOF/EMOF meta model. This makes it amenable to MDA tool suites for model transformations, as well as the generation of concrete textual and diagrammatic syntaxes with tool support \cite{gronback_model_2008}. To this end we analyse theoretically the modelling constructs required by URDAD. We elucidate and critically assess the URDAD meta model, and we propose a concrete textual syntax for an URDAD DSL. A Description Logics (DL)-based representation of the URDAD meta model is derived from the MOF/EMOF meta model in order to show its consistency and satisfiability.

Consequently we argue (also w.r.t. related work) that the URDAD DSL has two main advantages over the use of an URDAD UML profile. The language is considerably simpler than the UML and, with appropriate tool support, is expected to simplify the process through which requirements engineers can build high-level, technology-neutral models. Our new DSL enforces the structure required for a valid URDAD model, thereby requiring only a rather small and simple set of meta model constraints at the basis of tool-supported model validation. In addition the URDAD DSL provides better support for specifying service contracts within a service-oriented approach.

