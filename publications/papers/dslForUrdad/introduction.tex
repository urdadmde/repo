\section{Introduction}
Insufficiency in requirements engineering is still regarded 
as a main root cause of poor software quality. This is due 
to various factors, both human and technological, including 
vague specification languages with only informally defined 
semantics. In particular, insufficient language support for 
\emph{layered} specifications (i.e., decompositional system 
descriptions at different levels of granularity), leads
software developers to making wrong presumptions about 
lower level requirements \cite{espana_evaluating_2009}. 
Tool support for the validation of requirements specifications, 
or for the automatic extraction of test cases from them, 
is also still weak \cite{bashardoust-tajali_extracting_2008}.

Model-Driven Engineering (MDE) \cite{schmidt_model_2006} aims
at solving some of those problems by using modeling languages 
with well defined semantics as well as by requiring primary 
models to be domain models, not technical models 
\cite{asnina_computation_2010}. Tool support for MDE is also 
encouraged. Consequently, technology-neutral domain models 
are developed by requirements specialists, not by technical 
experts \cite{asnina_computation_2010}.

URDAD, the Use-Case, Responsibility-Driven Analysis and 
Design methodology \cite{fritz_solms_technology_2007} supports 
the MDE approach in a service-oriented way \cite{solms_urdad_2010}. 
URDAD is used by requirements specialists to develop and validate 
technology neutral requirements models. URDAD models of systems
are thus technology-neutral, Platform-Independent Models (PIM) 
in the Model-Driven Architecture (MDA) context \cite{solms_urdad_2010}.
Moreover: URDAD supports a layered, decompositional approach at 
different levels of granularity. At each level of granularity, 
testable service contracts are established together with services 
that fulfill those contracts. Thereby, the functions of higher 
level services can be realized as compositions of lower level 
services.

In the past, requirements engineers had used the Unified 
Model Language (UML) to denote URDAD models. Initially the 
UML had been a reasonable choice for that purpose because 
of its widespread acceptance in the software industry. In 
the UML-based approach to URDAD-modelling, however, the 
responsibility to only use an URDAD-appropriate subset 
of the UML (and thus to stick to the intended URDAD modelling 
rules) was entirely on the side of the requirements engineers. 
One can indeed develop for URDAD a UML profile with help of a 
set OCL-based metamodel constraints. However, the number and 
complexity of those OCL constraints confronted requirements 
practitioners, who wanted to work with URDAD, with unnecessary 
difficulties. Consequently the quality of UML-denoted URDAD 
models was generally so low that MDE-style model transformations 
were not feasible. Those problems motivated the design of a 
Domain-Specific Language (DSL), to be used instead of UML, 
in support of the URDAD methodology.

In this paper we present a new domain-specific language 
(DSL) for the domain of technology-neutral, services-oriented 
requirements modeling. Our new URDAD DSL is described in terms 
of a MOF/EMOF meta model. This makes it amenable to MDA tool 
suites for model transformations, as well as the generation 
of concrete textual and diagrammatic representations with 
tool support \cite{gronback_model_2008}. For this purpose
we analyze theoretically the modeling constructs required 
by URDAD; we elucidate and critically assess the URDAD meta 
model, and we propose a concrete textual syntax for an URDAD 
DSL. A Description Logics (DL)-based representation of the 
URDAD meta model is derived from the MOF/EMOF meta model 
in order to show its consistency and satisfiability.

Consequently we argue that the URDAD DSL has two main 
advantages compared to the use of an URDAD UML profile. 
The formalism is considerably simpler than the UML and, 
with appropriate tool support, is expected to simplify 
the process through which requirements engineers can 
build high-level, technology-neutral system models.
Our new DSL enforces the mandatory structures of 
valid URDAD models, thereby requiring only a rather 
small and simple set of meta model constraints at the
basis of tool-supported model validation. Further details 
will be described and discussed in the following sections.
