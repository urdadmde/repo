%\documentclass[11pt]{article}
\documentclass{llncs}
\usepackage{graphicx}
\usepackage{epsfig}
\usepackage{listings}
\usepackage{color}

\definecolor{lightgray}{gray}{0.85}
\lstset{backgroundcolor=\color{lightgray}}

%\usepackage{a4wide} %Smaller margins = more text per page.
%TODO commands by Henrik Skov Midtiby 
%\newcommand{\todo}[1]{\marginpar{#1}}
\newcommand{\todo}[1]{
    \addcontentsline{tdo}{todo}{\protect{#1}}
    \marginpar{#1}
}

% Additionally a command to generate the list of todos.
\makeatletter \newcommand \listoftodos{\section*{Todo list} \@starttoc{tdo}}
\newcommand\l@todo[2]
  {\par\noindent \textit{#2}, \parbox{10cm}{#1}\par} \makeatother

%Now the list of todo's could be inserted by using the command
%  \listoftodos

\begin{document}

\frontmatter          % for the preliminaries

\pagestyle{headings}  % switches on printing of running heads

\title{A Domain-Specific Language for URDAD}
\titlerunning{DSL for URDAD}  % abbreviated title (for running head)
\author{F. Solms\inst{1}, C. Edwards\inst{1}, S. Gruner\inst{1}, A. Paar\inst{1}, 
        D. Loubser\inst{2}}
\institute{Dept. of Comp. Sc., Univ. of Pretoria, South Africa,
\email{fritz@solms.co.za} \and 
Solms Training and Consulting CC, 2109 Johannesburg, South Africa}

\maketitle

\begin{abstract}
This is the amazing abstract. 

\end{abstract}


\keywords{model-driven development, domain-specific language, meta model,
 service-orientation, requirements engineering, platform-independent model}
 
\section{Introduction}\label{sec:Introduction}

Inferior requirements quality remains a core contributor to software quality\cite{heck_experiences_2008,_strategies_2011}. Furthermore, early defect detection is critical for cost containment\cite{betterRefernceThanBoehm1981WhichReliesOnWaterfall}. Model-Driven Engineering (MDE) and the Model-Driven Architecture (MDA) in particular aim to address this by making the models the primary artifacts which are to be developerd by requirements engineers who have solid domain knowledge and not by technical experts. The platform independent model is enriched with architecture and technology information an the system artifacts (e.g.\ code) are generated either manually by developers or in an automated way using model-transformation and code generation technologies.

URDAD (Use-Case, Responsibility Dirven Analysis and Design) \cite{solms_technology_2007} is a methodology which generates MDA's PIM \cite{solms_generating_2009}. It is a semi-formal \cite{solms_urdad_2010} approach to technology-neutral analysis and design which we are in the process of making incrementally more formal. Historically URDAD models were encoded in UML, but in order to reduce the complexity, tighten up the semantics and enforce certain model qualities through a rigid model structure, we have developed a domain specific language (DSL) for URDAD with a defined metamodel with a concrete text syntax \cite{solmsfritz_domain-specific_????}..

\emph{This might be a little controversial:} URDAD provides an environment (an architecture) to perform the analysis and design. Like any architecture, it needs to provide a suitable infrastructure to support the quality attributes for the processes which are deployed in the architecture \cite{}, i.e.\ the architecture needs to be designed in order to address the quality requirements.

In this paper we identify the stake holders who have an interest in the process and its outputs, their quality requirements and

This can be viewed as an architecture within which one can realize the quality requirements for the analysis and design process can be realized. The infrastructure includes the process definition for services oriented analysis and design as well as a metamodel enforcing the structure of the resultant analysis and design model.


In this paper we aim to identify quality requirements for the requirments model itself, identify quality drivers for the requirements model and analyze how these quality drivers are embedded in the URDAD methodology.

Nodel driven development aims to address this by making the requirements model

URDAD collapses CIM and PIM

URDAD is a process for architecture neutral analysis and design resulting in a model which does not address any architectural concerns, yet it is itself an architecture for doing the requirements process

The concept of quality implies some form of either absolute or relative measurement, i.e.\ something is of high quality yielding good results against some quality measures or it proofs to be of better quality when compared with an alternative. 
 The concept of quality is intrinsically subjective - hence we have the saying that `quality is in the eye of the beholder'. For the purpose of this paper we define quality is a measure to which a solution fulfills the stakeholder functional and non-functional (quality) requirements\cite{}. However, the degree to which a system fulfills its functional requirements is also an aspect of quality.  \cite{lange_managing_2005,lange_improving_2006}.


In the context of an analysis and design methodology one needs to consider both, the quality of the methodology or process as well as the quality of the resulting analysis and design model. It is questionable whether a process which yields low model quality can itself be viewed of high quality as the purpose of the process is to generate the model. On the other hand, one may have different processes which yield similar model qualities but with very different process qualities. 

In order to assess quality, we need to measure it. This paper will introduce quality measures for the design

Refer to \cite{wirfs-brock_object-oriented_1989}

One of the core challanges is to have requirements specialists make the paradigm shift to specify requirements in terms of services contracts for reusable services\cite{haines_impact_2007}. In practice we have found that URDAD assists significantly on that front.



Quality requirements for process  by infrastructure within which process executed, model qualities driven out by model infrastructure (metamodel) and system qualities ultimately by system architecture

process assists in improving quality of requirements by forcing certain questions

because requirements and design are arch and techn neutral, stakeholders (business) can understand them and validate them, improving quality

adapters to existing legacy

urdad as a semi formal method aims for full testability and partial validatability.

Formal methods complex, expensive, high skills, inflexible. Could be applied to certain critical services.

 
Yes, URDAD provides an architecture (infrastructure) for doing the analysis and design work - hence it addresses quality requirements. Any architecture needs to be quality driven

\section{Overview of URDAD 
\label{sec:urdadMethodology}}

URDAD is currently a \emph{semi}-formal methodology for requirements elicitation in the form of layered service contracts and processes  \cite{solms_urdad_2010}; its formalisation is the topic of ongoing research. \emph{Business processes} are specified as services with corresponding service contracts and process flows. The URDAD methodology stipulates a repeatable engineering workflow on the basis of the following iterated steps: service contract specification, responsibility allocation, and process design. The methodology envisages requirements specialists across responsibility domains to contribute to a single requirements model.

\paragraph{Service Contract Specification.} The URDAD methodology facilitates the incremental refinement of service requirements across different levels of granularity. Service requirements are encapsulated within a service contract. The specification of a service contract includes the identification of stake holders, the functional and quality requirements, and the data structures required for the service's request and result objects. A stake holder may be a role, or another service.

Functional service requirements can be expressed in ter\-ms of pre- or post-conditions. If all preconditions of a service are met, the service must be provided. The specification of each pre-condition includes an exception type which must be raised to notify the service requester that the requested service is refused due to the associated pre-condition not being met. A specified post-condition must hold true after the service has been provided. Non-functio\-nal service requirements stipulate qualities such as scalability, efficiency, reliability, accessibility, security, etc.

\paragraph{Responsibility Allocation.} During the responsibility allocation step the lower level services that are used to assemble a higher level service are identified by their ability to address the higher level service's functional requirements. Many of URDAD's concepts originate from Responsibility Driven Design (RDD) \cite{wirfs-brock_object-oriented_1989,wirfs-brock_object_2002}. Each service contract and its corresponding service are assigned to a responsibility domain. This prompts domain experts to search within an appropriate responsibility domain for existing services that can be re-used to implement functional requirements.

Service contracts represent requirements on a specific level of granularity. The complete requirements for a service are accrued by the accumulation of its requirements and those of its required lower level services. The hierarchical decomposition of service contracts has made requirements engineering better manageable. Additional levels of granularity can be opened by coalescing several service requirements into a single cohesive super-service. Such abstraction from details (principle of information hiding) reduces the intellectual complexity, improves the understandability of requirements elicitation tasks and op\-ens further reuse opportunities. 

\paragraph{Process Design.} Computational processes are specified using services id\-entified in the responsibility allocation step, i.e.\ processes are `orchestrated' across lower level services used to realize the pre- and post-conditions of the service. A process specification is assembled from standard control logic for sequential, concurrent and conditional activities with activities either  constructing and request objects or the object representing the computational output, requesting lower level services, handling exceptions raised by lower level services or raising an exception or returning a result to the user of the service. Each path through a process graph must either end with an output, or with raising an exception associated with a precondition of the service.



\section{The URDAD metamodel}

Proposed structure
Ideas for issues to be covered.
Feel free to shuffle, remove, adjust, reword, mock, critique, hide, destroy etc
References/related work still needs to be added
Statements will need to be sanity checked

{\em Fritz: Need to check overlap between this section and introduction}

\begin{itemize}

	\item URDAD META MODEL JUSTIFICATION
	\begin{itemize}
		\item Currently URDAD is considered to be semi-formal requirements modeling methodology. (REF URDAD paper)
		\item In an effort to formalise URDAD, there exists a need for the formal specification of URDAD's semantics.
		\item The concepts and rules associated with the use of the URDAD methodology must be clearly defined.
		\item The creation of an URDAD metamodel will assist with the formal definition of URDAD's semantics.
		\item Define Metamodeling
		\item Models versus metamodels. Matter of perception. Discuss levels of abstraction. (One particular metamodel may be considered to be an instance model of another more abstract metamodel.) 
		\item A metamodel represents a level of abstraction.
		\item	Metamodels are closely related to ontologies. (Concept definition, depiction, including relationships between concepts) DISCUSS
		\item While the concepts associated with URDAD are independent of any particular technology or organisation, URDAD is currently used to produce Platform Independent Models (PIMs) as envisaged in Model Driven Architecture (MDA), the Object Management Group's (OMG) approach to Model Driven Engineering. 
		\item PIMs are independent representations of processes, which may or may not be implemented as a software system. They are used to depict the functional requirements of a system. Once all non-functional requirements have been taken into consideration and an appropriate implementation environment has been selected, PIMs can be used to create Platform Specific Models (PSM) through a process of model transformation and refinement. Unlike PIMs, PSMs are inseparable from the actual technology platforms that will be used to implement the system.
		\item	Historically UML has been used to encode URDAD models. 
		\item	UML was a natural selection considering that it represents a standards-based modelling language that is managed by the OMG.
		\item	Unfortunately encoding the URDAD model in UML is not a trivial exercise.
		\item	It is difficult to ensure that the model is encoded in a consistent manner and it requires a strong discipline in the usage of UML.
		\item For example, actors are used to represent the stakeholders in UML use case diagrams. URDAD prescribes the use of interfaces to depict the roles fulfilled by stakeholders.
		\item For each service URDAD requires that there is a clear linkage between the service's stakeholders and their functional and quality requirements. (REF URDAD paper)
		\item An URDAD UML profile was created in order to help ensure that URDAD UML models are consistently encoded.
		\item For example, the URDAD profile introduced a <<requires>> dependency which facilitated the linkage between a service's stakeholders and their requirements.
		\item However, while this profile has helped improve the quality and consistency of URDAD models encoded in UML, the process is still error prone and unintuitive.
		\item Domain experts tasked with creating URDAD UML models are forced to add a range of mundane relationships and stereotypes in order to accurately capture URDAD's semantics. 
		\item It is also not trivial to capture certain aspects of URDAD's semantics in UML, even with the assistance for a UML profile. (Examples?)
		\item The creation of an URDAD metamodel attempts to address these issues.
		\item Depending on the manner in which the metamodel is encoded, URDAD models that conform to the metamodel will require substantially less manual validation. These models will also hopefully be less complex, especially if tools are developed to support the creation of these models.
	\end{itemize}

	\item CONCEPTS THAT DEFINE URDAD'S METAMODEL
		\begin{itemize}
			\item The URDAD metamodel and the semantics it seeks to define, are independent of its physical encoding.
			\item URDAD's metamodel can be encoded in more than one format.
			\item When selecting a mechanism to encode the metamodel, it is critical that URDAD semantics are represented in their entirety. If this is not the case one could argue that the encoded metamodel does not truly represent URDAD.
{\em Fritz: What is meant by this?}
			\item The fundamental concepts that define the URDAD metamodel are as follows:
			\begin{itemize}

				\item THE MODEL
					\begin{itemize}
						\item Represents the requirements model instance and it constituents
						\item A model represents the formal definition of requirements for a problem domain, modeled as services within responsibility domains, where each service is constrained by the functional and quality requirements of its stakeholders. The stakeholders are themselves represented as responsibility domains.
					\end{itemize}
					
				\item RESPONSIBILITY DOMAINS
					\begin{itemize}
						\item Represents a logical grouping of elements within a model, where each element belongs to the responsibility domain.
						\item Similar to the concept of packaging, but conceptually more consistent with the responsibility driven nature of URDAD.
						\item Responsibility domains also provide a consistent mechanism of depicting the roles/stakeholders within the model.
						\item Responsibilities domains may only be composed of data structures, services, requirements and other responsibility domains.
						\item Each role is associated with a cohesive list of responsibilities.
						\item The introduction of the concept of a responsibility domain eliminates the need for the separate definition of a services contract, which is traditionally used to group logical related services. Services are now grouped by the responsibility domain within which they reside.
						\item In the tradition of namespaces each element within a responsibility domain is uniquely identified by its name appended to the fully qualified name of the domain itself.
					\end{itemize}
					
				\item SERVICES
					\item Each service exists to realise a use case 
{\em Fritz: at some level of granularity the user might be a higher level service, maybe we want to avoid the use case vs service mess and focus on using services} and fulfil the requirements of its stakeholders.
					\item Users of a service are also represented as stakeholders.
					\item Services represent a level of granularity. Each level of granularity can be regarded as a level of abstraction.
					\item Requirements exist at a particular level of granularity, and are themselves decomposed further across subsequent lower levels of granularity.
					\item There are two forms of requirements that a service seeks to address namely functional and quality requirements.
					\item It is important to note that both functional and quality requirements are not restricted to a single service. There exists the reality that more than one service may have requirements in common.
					\item URDAD does not address any non-functional requirements other than the non-functional requirements of the requirements themselves. For example, requirements should be easy to maintain. Requirements should exhibit principles of good design, such as decomposition across layers of granularity, single responsibility, loose coupling and high cohesiveness.
					\item Each service is represented as a formalised contract, with explicit pre and post conditions that seek to address the functional requirements of the service.
					\item Pre-conditions represent the conditions under which the service may legally be refused. A pre-condition's existence must be justified by a tangible functional requirement. 
					\item A pre-condition is contractually obligated to raise an exception if it is not fulfilled. This exception must be clearly specified on the service contract.
					\item Post-conditions represent the conditions that must hold true if all the pre-conditions have be fulfilled and the service is complete.
					\item Unless the service has been sourced from the environment, it must be possible to verify whether each post-condition has been fulfilled. (VERIFY THIS STATEMENT)
					\item The fulfilment of a post condition may have a lasting impact on the state of the environment. Inverse services may be associated with each post-condition. These services are responsible for reversing the effects the post-condition had on the environment and returning the environment to its original state, in the event of an error occurring during the execution of the service.
					\item Associated with each service there exists a process definition that is composed of all the activities that are required to provide the service. Each activity must either directly or directly be associated with:
						\begin{itemize}
							\item The validation of a pre-condition through the execution of a lower level service.
							\item The fulfilment of a post-condition through the execution of a lower level service.
							\item The simultaneous validation of a pre-condition and fulfilment of a post-condition; through the execution of a common lower level service.
							\item The construction of the service result
						\end{itemize}
					\item The process defines the logical orchestration of the activities that are required to provide the service.
					\item The lower level services utilised by each activity constitute the next level of granularity, when considered in the context of the particular service in question.
					\item All activities within a process must directly address one or more of the functional requirements associated with the service.
					\item No activities should exist that do not address functional requirements.
					\item An important non-functional requirement of all activities that constitute a service, is that each activity must be able to be traced back to the fulfilment of a functional requirement. There should be no redundancy.
					\item Each service has a consistent service signature. The signature takes the form of an appropriately named service, a single service request object that contains detailed information pertaining to the request and a single result object, which is composed of the information associated with the result of the service execution.
					\item Traceability is one of the essential characteristics of the service contract and its process definition. Traceability is realised as follows:
						\begin{itemize}
							\item Ensuring that activities only exist to address a pre-condition, a post-condition or both a pre-condition and a post-condition simultaneously.
							\item Associating each functional requirement with one or more stakeholder's represented as responsibility domains.
							\item Ensuring that a service addresses all functional requirements and only the relevant functional requirements.
						\end{itemize}
			\end{itemize}
		\end{itemize}
		
	
	\item METAMODEL ENCODING OPTIONS
		\begin{itemize}
				\item There are many languages and approaches that can be used to encode the URDAD metamodel. 
				\item It is important that the advantages and disadvantages of each encoding option are carefully taken into consideration.
				\item Regardless of the encoding option that is ultimately selected, URDAD's semantics must be able to be represented in their entirety.
				\item Each encoding may differ in the way in which URDAD's semantics are depicted.				
				\item (INSERT REF - Quality in Model-Driven Engineering) argues that model quality is determined by five aspects. The modeling language and tools used to define the model, the modeling process itself, techniques used to assure quality and the relative experience of the individuals tasked with building the model.
				\item An encoding option should be assessed according to these five aspects.
				\item Three encoding options have been considered. Encoding options are not mutually exclusive and several may be concurrently utilised if each offers its own unique advantages. For example, one particular encoding option may make it easier to represent semantics and reason about the model, while another encoding option may be aligned with standards and offer extensive tool support for model tool smiths and practitioners. 
				\item Currently three encoding options have been taken into consideration.
				\item The first option is to attempt to improve the UML encoding of URDAD. While this will always be an option worth considering, it was quickly discarded for the reasons already mentioned in this paper.
				\item Another encoding option that has been considered is the encoding of the URDAD metamodel in a knowledge representation language such as the Web Ontology Langauge (OWL) or more specifically OWL-DL one of OWL's sub languages. There are many benefits associated with this option... (ELABORATE)
				\item The third opinion and the encoding which represents the focus of this paper is to encode the URDAD metamodel using the Eclipse Modeling Framework's (EMF) Ecore metamodel. 
				\item It is possible to capture URDAD's metamodel by extending the Ecore metamodel and introducing URDAD's semantics.
				\item The Eclipse 
				
		TODO COMPLETE...		

		Discuss the tool support
		Discuss Ecore and in particular the Eclipse Modeling Project's alignment with the OMG standards.
			Natural selection considering URDAD current alignment with the OMG's MDA
			Ecore <-> EMOF
			OCL
			QVT
		M2T
		Textual and graphical concrete syntaxes (XText and EMF Text)
		Model Validation
		IDE environment - services offered (syntax checking etc)
		
			\item One may consider the formalisation and encoding of URDAD's metamodel to be an attempt to establish a Domain Specific Language (DSL) for URDAD by defining its abstract syntax.
			
		\end{itemize}
	
	\item THE ECORE METAMODEL ENCODING 
	
	TODO COMPLETE...
	
	Discuss how the URDAD metamodel is realised in Ecore
		
	\item OUTSTANDING ISSUES AND POSSIBLE IMPROVEMENTS TO THE ECORE METAMODEL
	
	TODO COMPLETE...
		
\end{itemize}

\section{Assessing the URDAD DSL \label{sec:assessment}}

In this section we analyze the URDAD metamodel for unsatisfiable and redundent model elements, compare the complexity of the URDAD and UML metamodels and the URDAD-DSL and UML encoded models, and assess the sufficiency of the URDAD-DSL for code and test generation.

%-------------------------------------------------------------------------------------

\subsection{Core metamodel integrity assessment}

\emph{Alex's section}


\begin{itemize}
  \item Check that no classes unsatisfiable
  \item Redundency checks (redundent classes, redundent attributes and properties) redundend constraints
  \item Check consistency of OCL constraints
\end{itemize}

%-------------------------------------------------------------------------------------

\subsection{Complexity}

UML being a generic language is substantially more complex than the domain specific language for URDAD.
Language complexity and encoded model complexity. The UML metamodel is orders of magnitude larger than the URDAD metamodel.

Additional artifacts required by an UML encoding of an URDAD model
\begin{itemize}
  \item Representation of service as both use case and service in services contract
  \item Realization link linking service and use case
  \item Package and interface duplication, each package representing a responsibility domain and defining an services contract (interface) for the responsibility domain.
  \item
\end{itemize}


%-------------------------------------------------------------------------------------

\subsection{Sufficiency}

This section assesses the sufficiency of the URDAD-DSL with respect to requirements traceability, implementation mappings and test generation. 

Requirements traceability is important for design validation and estimation. Validation includes assessing sufficiency and neccessity, i.e.\ assessing whether all requirements are met and whether all model elements are required. This has to be done across levels of granularity \cite{dick_design_2005}. Ramesh et al.\ \cite{ramesh_toward_2001} identify four core traceability link types including (i) \emph{satisfaction links} which represent links between requirements and activities which satisfy them, (ii) \emph{evolutional links} which link change requests and the resulting changes, (iii) \emph{rationale links} which link requirements to the rationale (e.g. goals) which drive them and (iv) \emph{dependency links} which represent links between model elements.

In the URDAD metamodel satisfaction links are represented by \verb+usedToAddress+ links between services and pre- and post-conditions. Evolutionary links are not addressed within the URDAD metamodel and are left to the version control environment. Higher level purpose, goals and rationale links are also not addressed in URDAD. The metamodel does, however, include the \verb+requiredBy+ linkage between requirements and the stakeholders which require them. A requirement around a service may be required by a responsibility domain (i.e.\ a roles) or by another service. The metamodel enforces that the traceability of a requirement and the responsibility domains/services for which require them is specified. Finally, dependency links are included in the model. The dependency is formost between services and service contracts, i.e.\ a particular process design for a service has dependencies on services contracts (not on particular service implementations). Note that the URDAD metamodel does not differentiate between dependency and satisfaction links. Requirements result in dependencies and if the dependencies are available, the requirements are satisfied.

Code generation in a services oriented approach requires for each service the specification of the process through which this higher level service is assembled from lower level services. This includes the standard process logic around sequential, concurrent and conditional activities, specification of service requests, the maintenance of process state, the construction of the request object and the concepts of returning a result or raising an exception. In URDAD service requests are enforced to be made against services contracts, i.e. URDAD enforces decoupling between service providers. In this context URDAD assumes that the concrete service providers used to realize the services contracts are either injected by the run-time environment like in Spring or Java-EE based dependency injection, or are determined during the implementation mapping phase.

The services contract specification within the URDAD-DSL contains the specification of service inputs and outputs, pre- and post-conditions and quality requirements. A pre-condition for a service is the requirement that a condition holds. If the pre-condition does not hold true, the service provider is entitled to refuse the service. A post-condition is a condition which must hold true after the service has been provided. The same condition may be a pre-condition for several services as well as a post condition for other services. For example, the condition \emph{studentIsRegistered} can be for the \verb+enrollStudentForCoursePresentation+ service as well as the post-condition for the \verb+registerStudent+ service. 

In order for conditions to be reusable across several pre- and post-conditions, a condition needs to receive information. For example the \emph{studentIsRegistered} condition needs to receive the information about the student we are referring to. In addition conditions need to be specified with respect to the result object and information obtained from the environment. In a services oriented approach, environmental information is obtained through other services. For example, the \emph{studentIsRegistered} condition could be assessed by using services like \verb+isRegistered+ or \verb+provideRegistrationDetails+. The URDAD-DSL supports conditions with inputs specified in the pre and post-concitions of services and which are us, for functional testability one needs to specify a process assembled across services through which information from the environment is retrieved together with a set of state constraints applied across the obtained data objects containing the environmental information\footnote{Note that the test specification is not based on the design of a particular process (e.g. state chart), but independent of service realizations within a process independent services contract.}. Also in testability the decoupling of the test process from actual service implementations is enforced by requiring that the service requests tie into services contracts, not service implementations. 
 

%-------------------------------------------------------------------------------------

\subsection{Assessing model qualities}



\begin{itemize}
  \item Completeness checks
(What does completeness mean - That only technical information needs to be provided and that the full requirements
from a business perspective are specified across levels of granularity.)
    \begin{itemize}
     \item all OCL constraints for decision conditions specified
     \item all OCL constraints for pre and post-conditions specified
     \item All required request and result fields all specified either via OCL constraints 
	or by default values.
    \end{itemize}
  \item Check that process addresses all functional requirements and nothing but the functional requirements.
\end{itemize}



\section{Related work \label{sec:relatedWork}}

NOTE: See the Zotero reference collection for this paper. Please add any additional references to that collection.

\begin{itemize}
 \item \cite{iacob_model-driven_2008} discuss the mapping of business rules specified using OMG's {\em Semantics for Business Vocabulary and Rules} (SBVR) to service specification and orchestration, mapping onto BPEL process specifications via MDA tools.

  \item \cite{asnina_computation_2010} stress the need for performing the modeling in the problem domain as well as the need to accumulate the requirements within a single model. They effectively also group services into responsibility domains represented by their notion of feature sets, decompose functional requirements across levels of granularity, orchestrate higher level processes across lower level services and define the notion of functional with cause and effect which can be viewed as a way of specifying a services contract. In addition they provide a {\em topological functional model} (TFM) for mapping technology neutral service requirements onto available concrete services pool. The TFM is independent of the modeling technique and can be applied to an URDAD model






\end{itemize}


\section{Conclusions and outlook}

In this paper we identified the stakeholders in the analysis and design methodology and the resultant requirements and technology neutral process design model and their quality requirements. We related these quality requirements to quality-drivers and measures which can be applied within a services-oriented approach. We then identified that subset of quality-drivers which has been embedded within the URDAD process and pointed out how some of these are enforced by the URDAD metamodel.

One of the practical benefits of the URDAD methodology is that it assists requirements engineers to make the paradigm shift\cite{haines_impact_2007} to defining stand-alone services contracts and to assemble processes from abstract, reusable, stateless services with the concrete service providers either selected during the implementation mapping phase or alternatively provided by the execution environment through mechanisms like real-time service provider selection and dependency injection.

Future work includes the specification of a graphical grammar making the domain specific language for URDAD accessible to requirements engineers like business analysts and the specification and development of quality assessment tools which can be used to either report quality measures or provide real time quality guidelines to modellers.

\bibliographystyle{plain} 
\bibliography{../../bibliography}

\end{document}
