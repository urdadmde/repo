\section{Discussion of the URDAD DSL \label{sec:assessment}}
In this section we analyse our meta model's consistency, assess the sufficiency of the URDAD DSL for traceability, code and test generation, and compare the complexities of URDAD DSL and URDAD UML based approaches.

\subsection{Consistency}
The URDAD meta model describes and relates the concepts of the URDAD domain of discourse. Similarly, \emph{Description Logics} (DL) are a family of knowledge representation languages that can be used to represent the knowledge of an application domain in a well-structured and formally sound manner. In DL, the general terminology of a domain is contained in the TBox. The contingent knowledge about particular individuals is contained in the ABox. In DL, the fundamental inference relation on concept expressions is the \emph{subsumption} relation. A special case of subsumption is \emph{satisfiability} which poses the problem of checking whether a concept description does not necessarily denote the empty concept. For empty concepts 
the set of individuals of this concept is always empty. We have transformed our URDAD meta model $\mathcal{M}1$ into an $\mathcal{ALC}(\textbf{D})$ ontology using the \emph{TwoUse} \cite{parreiras_using_2010} Eclipse plugin. The collection of modelling concepts defined by the URDAD meta model was successfully validated to be \emph{satisfiable} $\surd$.

URDAD $\mathcal{M}0$ instance models were transformed into ontological instance knowledge based on the $\mathcal{ALC}(\textbf{D})$ representation of the URDAD meta model. The $\mathcal{M}0$ assertional knowledge and the terminological axioms representing the $\mathcal{M}1$ model were automatically checked by a 
Description Logics reasoner to be \emph{non-contradictory} $\surd$. In particular, qualified cardinality restrictions were used to check minimum-cardinality constraints defined for URDAD modelling constructs like that a requirement is required by one or more stakeholders (the unique name assumption is enforced for ontological individuals to prevent the reasoner from inferring their identity). 

\subsection{Sufficiency for MDE}
Requirements traceability is important for design validation and estimation. Validation includes assessing sufficiency and necessity, i.e., assessing whether all requirements are met and whether all model elements are required. This has to be done across levels of granularity \cite{dick_design_2005},whereby four types of traceability should be taken into account \cite{ramesh_toward_2001}: \emph{satisfaction} links, \emph{evolutional} links, \emph{rationale} links, and \emph{dependency}.

In our meta model, \emph{satisfaction links} are represented by \verb+usedToAddress+ links between services and functional requirements. They can be used to trace that all functional requirements are addressed and that one does not use services which do not address functional requirements. \emph{Evolutionary links} are not addressed within the URDAD meta model as they are provided by the version control environment. \emph{Rationale links} are also not represented in URDAD as URDAD does not currently include the concept of a goal. The meta model does, however, include  \verb+requiredBy+ links between requirements and their stakeholders, i.e.\ between a requirement and its source. \emph{Dependencies links} between model elements are explicitly represented. 

The meta model was found to be sufficient to generate or specify a concrete textual grammar capturing the concepts required by the URDAD methodology. The generated editor and parser as well as the standard model validators provide basic validation against the meta model, including compliance to the meta model constraints. We populated an example model and encoded it in both, the URDAD DSL using our textual syntax, and in UML.

The example model has been analyzed by developers and was found to be sufficient for implementation mapping (code generation) and test generation. In addition, a requirements model specified in the URDAD DSL can be mapped onto a process specified in URDAD UML. It does, however, require the mapping of functional requirements onto functional test processes. Further URDAD DSL tool support needs to be developed to facilitate industrial adoption of the DSL. The map-ability from our DSL into UML can facilitate further tool-support.

The concept of an `event' is not reflected in URDAD's meta model. However, events do not fit naturally into a service-oriented approach where services are regarded as stateless. Even though events can be `simulated' by mapping them onto either the receipt of a service request or a response, URDAD and its meta model are not particularly suited to modelling event-centric systems.

\subsection{Complexity}
The term `complexity' is here understood intuitively as `difficulty of application' from a practical perspective. When looking at a modelling task, we are confronted with the complexity of the model under construction, the complexity of the language used to this end, and the complexity of the workflow through which the task can be accomplished.

The \emph{conceptual model complexity} is the same, irrespective  of whether the models are denoted in the UML or the URDAD DSL. This is so because we are specifying in both cases the same information about a system under construction.

The \emph{language complexity} can be assessed by assessing the complexity of its meta model \cite{mohagheghi_evaluating_2007}. The language complexity affects its learnability and the complexity  of the tools developed around the language including model editors, transformation components and validation tools. Even though a more complex language generally entails a steeper learning curve, it need not result in more complex model encodings. Often the converse is true. A more complex language may have more expressive power and can thus yield smaller model sizes. 

The aim of a domain specific language, on the other hand, is to provide a simpler language which introduces only the concepts required for a particular domain, whilst providing the expressive power to effectively make the statements required for that domain, i.e. to be both a simpler language and to result in simpler models. In the case of our meta model, its UML representation contains about 16 times as many classes and 7 times more relationships than its equivalent in URDAD DSL. 

Comparing the \emph{model complexity} of the URDAD UML model for our example to an equivalent URDAD DSL model we find that the URDAD DSL model has notably lower complexity. Moreover, the URDAD DSL can elegantly describe additional model constructs, which cannot elegantly be represented in URDAD-UML. This is due to the URDAD DSL directly supporting the required semantics for the URDAD methodology, whilst in UML some of the concepts are not directly represented and need to be assembled from more basic model constructs.

The \emph{usage complexity} will depend largely on the diagrammatic syntax tool support. However, it is expected the lower language complexity, the direct representation of core concepts like service contracts and the much lower level of inconsistency risks will contribute to a lower usage complexity.