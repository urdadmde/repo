
\subsection{Process stakeholders and their quality requirements}

According to both Crosby and Demig quality is defined in terms of custumer satisfaction. Looking at the URDAD process and trying to answer the 
question: To what extend does the URDAD process fullfil the requirements of the stake holders that have an interest in it. The URDAD process can
exist as sub process within a software developemnt methodology such as XP or Agile, focusing sole on requirements and formulating design 
artifacts for the project.

\todo[inline,color=blue!40]{Sifiso: Have to formulate this paragraph to be more understable}

The higher level of requirements process yields greater level of system process maturaity \cite  {Boehm_20Viev_Century_SE}. 
The quality of the value it provides is dependant on the interaction between the process and various stake holders across phases.
From the analaysis phase the requirements engieering will drive the excution of the process, the interested stakeholders can start contributing and extrating value from the process.  




\begin {itemize}
 \item Give a scenairo where URDAD is bieng applied
 \item Look at the stake holders who have interest in the URDAD process 
 \item Identify their requirements requirements
 \item See and disscuss how those requirements are best address by the processg
\end {itemize}

\begin {itemize}
 \item Discuss the URDAD process incoparated in to a Software Development Methodology
 \item How URDAD can be used with other to solve specific problem(Service Oriented),
  while a different methodology is bieng used fro the rest of the project. How can this contribute to quality ?
  Compare URDAD to other methodologies, from a quality point of view.
\end {itemize}


\cite{berard_what_1995}

Stake holders:
\begin{itemize}
  \item Project management (measurability, repeatability, estimatability)
  \item Requirements specialists (ease of use, simple process, defined process activities, defined inputs and outputs, tool support)
  \item Business (low cost, trainable, grow in islands)
\end{itemize}


\begin{itemize}
  \item Process measurability
  \item Repeatability
  \item Defined inputs and outputs
  \item Clearly specified tasks with defined activities
  \item Process consistency (URDAD generates itself)
\end{itemize}

Can apply process to a sub-world, decouples from higher and lower level granularities via contracts

Could introduce more abstract qualities and things the process must have to realize these, e.g. \emph{usability} affected by many of these

CMM requires process definition


\subsection{Internal process consistency}

\todo[inline,color=blue!40]{Fritz: Reuse sounds to me like both, a process and a model quality}

Here show that if you use URDAD to design an analysis and design methodlogy, you will get URDAD. Feed additional concepts into URDAD.

Here identify the stakeholders in both, the process and the outputs of the process and their quality requirements.


\subsection{Internal process consistency}

Here show that if you use URDAD to design an analysis and design methodlogy, you will get URDAD. Feed additional concepts into URDAD.

process assists in improving quality of requirements by forcing certain questions
