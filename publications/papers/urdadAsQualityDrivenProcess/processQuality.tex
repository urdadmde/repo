\section{Process quality}

The URDAD process is right at the begining of the SDLC dealing with both requirements engineering and design. The stake holders
involved in the project they have quality requirements around the process which they extract value from. These speaks to different
aspects of the process and their degree of importance vary across the stake holders.

\cite{berard_what_1995}does 

Stake holders:
\begin{itemize}
  \item Project management - (measurability, repeatability, estimatability)
  \item Requirements specialists (ease of use, simple process, defined process activities, defined inputs and outputs, tool support)
  \item Business (low cost, trainable, grow in islands)
\end{itemize}


\begin{itemize}
  \item Process measurability
  \item Repeatability
  \item Defined inputs and outputs
  \item Clearly specified tasks with defined activities
  \item Process consistency (URDAD generates itself)
\end{itemize}

Can apply process to a sub-world, decouples from higher and lower level granularities via contracts

Could introduce more abstract qualities and things the process must have to realize these, e.g. \emph{usability} affected by many of these

CMM requires process definition


\subsection{Internal process consistency}

Here show that if you use URDAD to design an analysis and design methodlogy, you will get URDAD. Feed additional concepts into URDAD.

process assists in improving quality of requirements by forcing certain questions
