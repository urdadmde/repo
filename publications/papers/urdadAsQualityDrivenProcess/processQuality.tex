
\subsection{Process stakeholders and their quality requirements}

According to both Crosby and Demig quality is defined in terms of customer satisfaction. Looking at the URDAD process and trying to answer the 
question: To what extend does the URDAD process fulfill the requirements of the stake holders that have an interest in it. The URDAD process can
exist as sub process within a software development methodology such as SCRUM, focusing solely on requirements and formulating design 
artifacts for the project.

\todo[inline,color=blue!40]{Sifiso: Have to formulate this paragraph to be more understable}

The requirements process of high quality will yield high process maturity \cite  {boehm_view_2006}. The URDAD process has
less dependence on the organization culture, also to what communication channel is used during the the requirements 
engineering phase. The process provides a formal logical way for requirement engineers to identify, analyze and document 
the requirements in a way produces a requirements specification is of high quality. A requirement specification in form of a model will 
have high level of quality than the one documented using a natural language. The methods used to feed information into the model along the process 
ensures they are no duplications of requirements and most importantly concepts,therefore allowing requirement specialist to be more productive. 


The design phase can run concurrently with the analysis phase as the process is iterative, as you have have sufficient requirements for a 
particular use-case you can take the use case to the design phase.A complete and traceable requirement ensures that at any point in time
changes made to the requirement can be tracked, the stakeholder who had the requirement is identifiable while ensuring at any stage along 
the process the value of the requirements can be relatively measured making it more easy to prioritize.\cite {gilb_paper:_2010}


The first one, URDAD has less dependency on the organization culture. Most software development methodologies are more oriented to the culture of an organization. Such 


\begin {itemize}
 \item Give a scenairo where URDAD is bieng applied
 \item Look at the stake holders who have interest in the URDAD process 
 \item Identify their requirements requirements
 \item See and disscuss how those requirements are best address by the processg
\end {itemize}

\begin {itemize}
 \item Discuss the URDAD process incoparated in to a Software Development Methodology
 \item How URDAD can be used with other to solve specific problem(Service Oriented),
  while a different methodology is bieng used fro the rest of the project. How can this contribute to quality ?
  Compare URDAD to other methodologies, from a quality point of view.
\end {itemize}


\cite{berard_what_1995}

Stake holders:
\begin{itemize}
  \item Project management (measurability, repeatability, estimatability)
  \item Requirements specialists (ease of use, simple process, defined process activities, defined inputs and outputs, tool support)
  \item Business (low cost, trainable, grow in islands)
\end{itemize}


\begin{itemize}
  \item Process measurability
  \item Repeatability
  \item Defined inputs and outputs
  \item Clearly specified tasks with defined activities
  \item Process consistency (URDAD generates itself)
\end{itemize}

Can apply process to a sub-world, decouples from higher and lower level granularities via contracts

Could introduce more abstract qualities and things the process must have to realize these, e.g. \emph{usability} affected by many of these

CMM requires process definition


\subsection{Internal process consistency}

\todo[inline,color=blue!40]{Fritz: Reuse sounds to me like both, a process and a model quality}

Here show that if you use URDAD to design an analysis and design methodlogy, you will get URDAD. Feed additional concepts into URDAD.

Here identify the stakeholders in both, the process and the outputs of the process and their quality requirements.


\subsection{Internal process consistency}

Here show that if you use URDAD to design an analysis and design methodlogy, you will get URDAD. Feed additional concepts into URDAD.

process assists in improving quality of requirements by forcing certain questions
