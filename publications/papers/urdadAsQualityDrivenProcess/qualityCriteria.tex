\section{Quality criteria}
\label{sec:qualityCriteria}

We use the ISO 9000:2000 \cite{hoyle_iso_2000} definition of quality stating that quality is \emph{``the degree to which a set of inherent characteristics fulfills requirements``}. For each quality requirement one can specify a quality criterion which is an observable quality characteristic of the solution and a quality measure providing a quantitative metric for the quality criterion\cite{firesmith_quality_2005}. This section identifies the stakeholders, who have an interest in the model and the process which produces the model, and discusses their specific quality criteria.

%-------------------------------------------------

\subsection{Model stakeholders and their quality criteria}
\label{sec:modelStakeholdersAndQualityRequirements}

A lot of work has already been done on defining and measuring program code quality (see, for example, \cite{boehm_barry_w._characteristics_1978}). Recently the focus has shifted onto understanding and measuring model quality \cite{lange_managing_2005,lange_improving_2006,shim_design_2008,qi_yu-dong_analysis_2010}. Though most of the work is applied to UML models, the concepts are generally applicable across analysis and design models, i.e.\ they are generally not UML specific. 

Lindland\cite{lindland_understanding_1994} provides a widely used quality categorization into \emph{semantic quality}, which assesses whether a model confers the meaning it was meant to confer, \emph{syntactic quality}, which is the degree to which the language used to specify the model conforms to the syntax rules of the language, and \emph{pragmatic quality}, which is the extent to which the model can satisfy its intended use. Christiaan Lange  \cite{lange_christiaan_assessing_2007} related model quality back to model purpose and identified a range of pragmatic model quality characteristics including complexity, traceability, modularity, completeness, consistency and communicativeness. Models that exhibit such qualities have been shown to commonly lead to systems with similar quality attributes \cite{podgorelec_estimating_2007}. In this paper we do not consider communicativeness and social quality as we feel these will be strongly influenced by both, the semantics and the model user interface(s).

The core stakeholders in the model include {\bf requirements engineering} (e.g.\ business analysis), {\bf architecture}, {\bf implementation} (e.g.\ developers), {\bf quality assurance} (e.g.\ testing), {\bf project management} and the {\bf client}. In order to address their responsibilities, these stakeholders need certain model qualities. All of these stakeholders need \emph{semantic quality} (since they need to extract the intended meaning from the model). The syntactic and pragmatic quality needs differ across stakeholders, though \emph{simplicity} and \emph{consistency} are qualities which are beneficial to all.

{\bf Requirements engineers} (e.g.\ business analysts) use the model to capture the functional and non-functional requirements of services in the form of service contracts and function (business process) designs fulfilling these service contracts. They also may use the model for validation and documentation generation, needing \emph{syntactic quality} to make this feasible.  The \emph{pragmatic qualities} particularly relevant to requirements engineers include \emph{modifiability} (as they would have to modify the model in the context of changing requirements and function improvement), \emph{simplicity} assisting understandability and modifiability, \emph{completeness} for complete requirements and function designs, model \emph{consistency}, \emph{decoupling} of functional requirements from function design, \emph{reuse} to reduce model complexity and increase model consistency, and \emph{traceability} to facilitate model validation and improve \emph{modifiability}. 

We define ``architecture'' as the infrastructure within which services are deployed and executed. Thus, whilst implementation is concerned with the actual service implementations, architecture is concerned with the infrastructure hosting the services. The architecture may span organizational, hardware and software infrastructure. Organizational and systems architects need to be able to design an architecture hosting the services in such a way that the quality criteria (\emph{quality of service}) can be met. They also need to be able to assess whether an existing architecture can host a service as per contract and specify architectural modifications if that is not the case. {\bf Architecture} typically uses the high level service contracts and function specifications. These need to be \emph{complete} and \emph{consistent}. \emph{Simplicity}, \emph{cohesion} and \emph{decoupling} make it easier for architects to obtain the required information.

Within in an MDA-based approach, {\bf implementation} (e.g.\ developers or managers who implement and control manual processes) perform the implementation mapping of the services onto an architecture specified by the architecture team. To this end, they require \emph{syntactic quality} as well as \emph{completeness} and \emph{consistency}. Furthermore, model \emph{simplicity}, \emph{decoupling}, \emph{reuse} and \emph{traceability} typically lead to implementations with those same qualities \cite{podgorelec_estimating_2007}.

The responsibility of {\bf Quality assurance} is to detect and report defects in the development process and the outputs of the process. They need the service contracts specifying both functional and quality criteria for services. \emph{Testability}, \emph{syntactic quality}, \emph{completeness} and \emph{consistency} are needed for test generation. Potential service implementations would be tested against the service contract. \emph{Traceability} is needed for impact reporting.

{\bf Project Management} uses the model for reporting and estimation purposes. \emph{Completeness} and \emph{traceability} are needed for status reporting on the implementation mapping.

The {\bf client} (business) uses the model to obtain function (business process) documentation and service contract documentation. Function (business process) documentation is useful for the general understanding and running of the system/business as well as for business process optimization. Documentation generation requires \emph{completeness}, \emph{consistency} and \emph{syntactic quality}. \emph{Decoupling} via service contracts is important to the client in the context of sourcing and assessing different concrete service providers (e.g.\ business partners, off-the-shelf systems). More generally, the client commonly needs modifiability in order to have requirements changes cost-effectively implemented, benefits from \emph{service reuse} in the context of cost containment and consistency, and needs \emph{traceability} for dependency analysis.

\begin{table}[h]
\caption{Stakeholders and their primary model quality criteria.}
\label{tab:modelQualityRequirements}
\begin{tabular}{|l|cc|cccccccc|} \hline
\multirow{4}{*}{\bf Stakeholders} & \multicolumn{10}{c|}{\bf Model quality criteria} \\ \cline{2-11}
& & & \multicolumn{8}{c|}{Pragmatic model quality criteria}\\ \cline{4-11}
    & \begin{sideways}Semantic\end{sideways} & \begin{sideways}Syntactic\end{sideways}  & \begin{sideways}Simplicity\end{sideways}
    & \begin{sideways}Completeness\end{sideways} & \begin{sideways}Modifiability\end{sideways} & \begin{sideways}Consistency\end{sideways}
    & \begin{sideways}Decoupling\end{sideways} & \begin{sideways}Cohesion\end{sideways} & \begin{sideways}Reusability\end{sideways}
    & \begin{sideways}Traceability\end{sideways} \\ \hline
%                         Semantic     Syntax        Simplicity  Completeness   Modifiable  Consistent  Decoupled    Cohesion     Reuse        Traceable
Requirements engineering & \checkmark & \checkmark & \checkmark & \checkmark & \checkmark & \checkmark & \checkmark & \checkmark & \checkmark & \checkmark \\
Architecture             & \checkmark &            & \checkmark & \checkmark &            & \checkmark & \checkmark & \checkmark &            &       \\ 
Implementation           & \checkmark & \checkmark & \checkmark & \checkmark &            & \checkmark & \checkmark & \checkmark & \checkmark & \checkmark \\ 
Quality assurance        & \checkmark & \checkmark & \checkmark & \checkmark &            & \checkmark &            &       &            & \checkmark \\ 
Project management       & \checkmark &            & \checkmark & \checkmark &            & \checkmark &            &       &            & \checkmark \\ 
Client (business)        & \checkmark & \checkmark & \checkmark & \checkmark & \checkmark & \checkmark & \checkmark &            & \checkmark & \checkmark \\ \hline
\end{tabular}
\end{table}

%---------------------------------------------------------------------------------------------------------%

\subsection{Process stakeholders and their quality criteria}

The {\emph Capability Maturity Model} (CMM) \cite{paulk_capability_1993} defines a range of quality attributes for mature processes including \emph{process definition}, \emph{measurability}, \emph{trainability}, \emph{repeatability} and at the higher level the ability to continuously optimize processes for \emph{cost} and other qualities. Berard\cite{berard_what_1995} discusses a related set of general requirements for a methodology including \emph{repeatability}, \emph{trainability}, \emph{wide process applicability}, and quality improvement in the outputs of the methodology. 

{\bf Project management} needs to perform planning, estimation, resource management, monitoring, control and reporting during the analysis and design process. To this end, they need \emph{estimatability}, \emph{repeatability} for accurate estimation and \emph{measurably} and tool support facilitating estimation, monitoring, and reporting.

{\bf Requirements engineers} execute the process itself generating the analysis and design model. They requires a process which is \emph{usable} and \emph{trainable}, and \emph{repeatable}.

The {\bf client} (e.g.\ business) requires \emph{low process cost}, a \emph{trainable} process, enabling the client to easily train up additional requirements engineers, and \emph{isolation}. Process isolation refers here to the ability to grow islands of expertise around the process, extracting value from those teams which apply the process without requiring that the process needs to be rolled out either across an entire project team or across an entire organization. This enables the client to take a risk-averse, incremental approach which allows for customization and controlled, incremental roll-out.

A further property required of a process is internal consistency. This property will be discussed in section \ref{sec:urdadConsistency}.

\begin{table}[h]
\caption{Stakeholders and their primary process quality criteria.}
\label{tab:processQualityRequirements}
\begin{tabular}{|l|ccccccc|} \hline
\multirow{2}{*}{\bf Stakeholders} & \multicolumn{7}{c|}{\bf Process qualities} \\ \cline{2-8}
    & \begin{sideways}Low cost\end{sideways}  & \begin{sideways}Repeatability\end{sideways} & \begin{sideways}Estimatability\end{sideways}
    & \begin{sideways}Trainable\end{sideways}
    & \begin{sideways}Measurability\end{sideways} & \begin{sideways}Consistency\end{sideways} & \begin{sideways}Isolation\end{sideways} \\ \hline
%                          Cost         Repatable    Estimatable  Trainable    Measurable  Consistent   Isolated
Project management       &            & \checkmark & \checkmark & \checkmark &            & \checkmark & \checkmark \\
Requirements engineering & \checkmark & \checkmark & \checkmark & \checkmark & \checkmark & \checkmark & \checkmark \\
Client (business)        & \checkmark & \checkmark & \checkmark & \checkmark & \checkmark & \checkmark \\ \hline
\end{tabular}
\end{table}
