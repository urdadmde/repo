\section{Quality drivers}
\label{sec:qualityDrivers}
The relience on quality-drivers forms the basis for the \emph{`quality-driven'} process.

Model quality-drivers are activities in the process and/or aspects of the metamodel which which increase one or more model qualities. One approach which can be used to assess whether a methodology provides a quality-driven process is to assess which of the quality-drivers are employed by the methodology. The list of quality-drivers presented here is not intended to be exhaustive. It is purely a collection of quality-drivers which are widely known and used. In addition to identifying quality-drivers, we also note that quality metrics can be applied to the resultant model in order to assess model quality. This section discusses some commonly used quality metrics for the identified quality requirements suitable for a services-oriented approach. For a more general list of model quality metrics the reader should consult \cite{mohagheghi_existing_2009}.

\paragraph{Semantic Quality} is a measure of the accuracy and completeness of the meaning conveyed with the analysis and design model. Semantic quality-drivers include the specification of an ontology and/or a metamodel for the analysis and design model. One needs to assess the semantic overlap between the modeling domain and the ontology provided by the modeling language. Though a number of studies discuss the relationship between semantics and metamodels \cite{staab_model_2010,veldhuis_tool_2009,henderson-sellers_bridging_2011} and the effect of semantic quality on the quality of analysis and design models \cite{buder_effects_2010,staab_model_2010}, little work has thus far been done on the quantitative assessment of semantic model quality. Domain specific languages aim to explicitly and directly cover the ontological space required by the modeling methodology or/and the modeling domain. The semantic quality assessment will thus be largely confined to the assessment of the semantic support provided by the employed modeling language.

\paragraph{Syntactic Quality} is a measure of correct language usage. Syntactic quality-drivers include the availability of a formal metamodel for the language as well as the specification of concrete text and graphical grammars for the specification of model instances. In this paper we confine the assessment of syntactic quality to the existence of a formal modeling language and concrete grammar for model specification. We will thus assume that model instances will adhere to the grammar of the chosen modeling language as this can easily be verified. Syntactic quality can easily be measured as a ratio of the number of relationships without syntax errors relative to the total number of relationship elements in the model. Thus, if $\cal{E}$ is the set of errors and $\cal{R}$ is the set of all relationships in the model, then the syntactic quality $Q_{syntax}$ is defined by
\begin{equation}
  Q_{syntax} = \frac{|\cal{R}| - |\cal{E}|}{|\cal{R}|}
\end{equation}

\paragraph{Simplicity} is a measure of lack of complexity. The complexity of the modeling language is usually assessed by measuring the complexity of the metamodel for that language\cite{mohagheghi_evaluating_2007}. A lot of work has been done on model complexity itself. Common approaches include using information entropy measures\cite{abrahamsson_extreme_2004}, language-theoretic approaches\cite{podgorelec_estimating_2007} and process complexity assessment based on the McCabe complexity measure \cite{mccabe_complexity_1976}. A services-oriented approach already enforces certain drivers for simplicity. In particular the assembling of processes from independent, stateless services, the enforced decoupling through services contracts, the improved reuse through the implied service discovery, as well as an implied adapter layer facilitating reuse across technologies and interfaces mismatches. A metamodel confining the modeling constructs and relationships between these, as well as a convenient grammar, reduce complexity. Another core driver for simplicity is the decomposition across levels of granularity enabling the understanding and processing of one level of granularity before having to concern oneself with the details of the next lower level of granularity. The availability of a services contract enables one to look at a service from a user/service consumer perspective before understanding the service process. Finally, traceability links and in particular enforced satisfaction links (i.e.\ that a process may only use services which address functionality) are drivers for simplicity. The field of complexity metrics is large and varied with most complexity metrics applicable to an URDAD model. URDAD does not suggest a particular metric.

\paragraph{Completeness} is a measure of the lack of missing information in the analysis and design model. Completeness of requirements is difficult to assess, though the discovery of certain stakeholders for which no functional or non-functional requirements exist is an indication of missing requirements. Design completeness is easier to assess as there is a global measure - the degree to which the requirements are realized. Core quality-drivers for completeness include enforced satisfaction links, the enforcement of testable pre- and post-conditions, as well as having a metamodel which enforces certain content, either structurally, or through metamodel constraints. In a service-oriented approach, a completeness metric can be specified as a function of 1) the percentage of functional requirements not addressed in the designed processes (identified through missing satisfaction links)\cite{shim_design_2008}, 2) the fraction of pre- and post-conditions for which the test process has not been specified and 3) the percentage of request and result classes for which the data structures have not been specified. At the implementation level completeness can be assessed as the fraction of services for which service implementations are not yet available.

\paragraph{Consistency} is an important model quality measure. A lot of emphasis has been placed on model consistency for UML models. This is so because UML models are often inconsistent due to the multi-diagram approach and the complexity of the modeling language itself. A commonly used quality-driver for model consistency is to use a much smaller (less general) language with more formally defined semantics which enforces the consistency through both, its structure and a set of metamodel constraints.

\paragraph{Cohesion} in a service-oriented approach has been extensively studied by Mikhael Perepletchikov et al.\ \cite{perepletchikov_cohesion_2007,perepletchikov_impact_2010}. Their approach is narrowly related to that of unity criteria \cite{gonzalez_unity_2009} identifying and quantifying interfacing cohesion, usage cohesion and implementation cohesion. These cohesion measures can be applied directly to an URDAD model. Quality-drivers for cohesion include enforcing the single responsibility principle as well as process localization within a controller service. 

\paragraph{Decoupling} is enforced in a services-oriented approach by requiring that services are only consumed via services contracts. This decouples services across levels of granularity. Quality-drivers for decoupling include a metamodel which enforces contracts based decoupling through metamodel structure or constraints and localization of process logic within a controller so that the service providers of the lower level services remain decoupled. In a services-oriented architecture, the number of services which are directly coupled to concrete lower level services can be used as a measure of coupling\cite{shim_design_2008}.

\paragraph{Modifiability} refers to the efficiency with which model changes can be applied, i.e.\ it is an inverse measure of the cost required to make model changes. Modifiability is difficult to quantitatively measure, but it is supported by other model qualities like \emph{simplicity},  \emph{decoupling} (modifiability through pluggability), and \emph{cohesion} (localized maintenance) and with a further quality-driver in the form of localizing process logic within a controller service, thereby projecting out additional levels of granularity. \cite{shim_design_2008} defines a quality metric for modifiability (flexibility) of services-oriented designs as a weighted sum of coupling, service granularity, and parameter granularity. We feel that simplicity (or complexity) and decoupling should also be included in the quality measure for modifiability.

\paragraph{Reusability} provides a measure of the ability and likelihood that a service can be reused.  \cite{khoshkbarforoushha_metric_2010,choi_quality_2008,feuerlicht_determinants_2007}
\cite{khoshkbarforoushha_metric_2010} point out that service reusability is impeded by contract and requirements mismatch. The former can be addressed via adapters. Core quality-drivers for reusability include \emph{decoupling} via services contracts with the latter also driving discoverability and consumability, \emph{levels of granularity} via process localization within a reusable controller service, and \emph{cohesion} through enforcing the single responsibility principle. Indeed, \cite{shim_design_2008} defines a simple quantitative reusability measure for service-oriented systems as a weighted sum of coupling, cohesion, granularity, and consumability.

\paragraph{Traceability} is required for design validation and estimation. Validation includes assessing sufficiency and necessity. \cite{ramesh_toward_2001} identifies four types of traceability links in models including \emph{satisfaction} links used to assess whether requirements are satisfied, \emph{evolutionary} links to trace along the evolution of an artifact over time, \emph{rationale} links which typically link requirements to higher level business goals, and \emph{dependency} links enabling one to identify dependencies of model elements. Quality-drivers include thus the availability of these traceability links in the modeling language and their enforced usage through the modeling process.