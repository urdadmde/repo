\section{Conclusions and outlook}

In this paper we identified the stakeholders in the analysis and design methodology and the resultant requirements and technology neutral process design model and their quality requirements. We related these quality requirements to quality drivers and measures which can be applied within a services-oriented approach. We then identified that subset of quality drivers which has been embedded within the URDAD process and pointed out how some of these are enforced by the URDAD metamodel.

One of the practical benefits of the URDAD methodology is that it assists requirements engineers to make the paradigm shift\cite{haines_impact_2007} to defining stand-alone services contracts and to assemble processess from abstract, reusable, stateless services with the concrete service providers either selected during the implementation mapping phase or alternatively provided by the execution environment through mechanisms like real-time service provider selection and dependency injection.

Future work includes the specification of a graphical grammar making the domain specific language for URDAD accessible to requirements engineers like business analysts and the specification and development of quality assessment tools which can be used to either report quality measures or provide real time quality guidelines to modelers.