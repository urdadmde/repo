\section{Conclusions and outlook}

Looking at stakeholder quality requirements for a service-oriented analysis and design generating MDA's CIM, we were able to identify a set of quality-drivers which improve one or more of the required quality attributes. We were able to show that most of the identified quality drivers were built into the URDAD methodology. We also illustrated the internal consistency of the URDAD process, demonstrating that when we use URDAD to design a service-oriented analysis and design methodology, we regenerate URDAD and its metamodel.

One of the practical benefits of the URDAD methodology is that it assists requirements engineers to make the paradigm shift\cite{haines_impact_2007} to defining stand-alone services contracts and to assemble processes from abstract, reusable, stateless services with the concrete service providers either selected during the implementation mapping phase or alternatively provided by the execution environment through mechanisms like real-time service provider selection and dependency injection.

Future work includes the specification of a graphical grammar making the domain specific language for URDAD accessible to requirements engineers like business analysts and the specification and development of quality assessment tools which can be used to either report quality measures or provide real time quality guidelines to modelers.