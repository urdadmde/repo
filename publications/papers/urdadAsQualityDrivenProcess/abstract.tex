\begin{abstract}
  URDAD, the \emph{Use-Case, Responsibility Driven Analysis and Design} is a service-oriented methodology used by requirements engineers to produce the Computation Independent Models (CIMs) of the \emph{Model Driven Architecture} (MDA) with sufficient detail and precision that they can be used directly as \emph{Platform Independent Models} (PIMs). The analysis and design process is supported by a metamodel specifying the modeling semantics and a concrete grammar used to capture URDAD models. In this paper we identify quality criteria for the resultant requirements specification and for the process itself. For each quality criterion we identify a set of quality drivers and show how many of these quality drivers are embedded within the URDAD methodology. Finally we demonstrate the internal consistency of URDAD process by showing that using URDAD to design a service-oriented analysis and design methodology yields the URDAD process and its metamodel.
\end{abstract}
