\section{The URDAD analysis and design process}

In this section we will introduce URDAD by using it to design a service oriented analysis and design methodology. If the process is internally consistent it needs to generate itself.


URDAD is a service-oriented analysis and design process which generates services contracts and associated process specifications across levels of granularity. 

\subsection{Quality drivers of URDAD}

Discuss first the methodology, then the metamodel with example encodings in concrete text syntax.
\todo[color=blue!40,inline]{Fritz:Try and explain URDAD while using URDAD to design URDAD}


Include step for quantifying each quality requirement and list quantification measures

Refer to \cite{gonzalez_unity_2009} to use unity criteria to define responsibility domain boundaries.
\todo{Fritz:Look at unity models as in Espana et al and see progress report for responsibility localization and general short-comings in use case based approach regarding this.}

\missingfigure{URDAD process}

\subsection{URDAD views}

supplied via text encoding and later via graphical syntax.


adapters to existing legacy


Refer to \cite{wirfs-brock_object-oriented_1989}

One of the core challanges is to have requirements specialists make the paradigm shift to specify requirements in terms of services contracts for reusable services\cite{haines_impact_2007}. In practice we have found that URDAD assists significantly on that front.
