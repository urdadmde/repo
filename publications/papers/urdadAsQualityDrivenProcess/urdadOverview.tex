\section{The URDAD analysis and design process}

In this section we will introduce URDAD from the perspective of a quality driven process generating an analysis and design model which enforces certain model quality requirements. 

\todo[inline]{Fritz: Discuss first the methodology, then the metamodel with example encodings in concrete text syntax.}


\subsection{Quality drivers of URDAD}

Most of the quality drivers discussed in \ref{sec:modelQualityDriversAndMetrics} are builts into the URDAD process. Table \ref{tab:qualityDrivers} lists the employed quality drivers and the model qualities they are meant to support.

\begin{table}[h]
 \caption{URDAD model quality drivers for quality requirements}
 \label{tab:qualityDrivers}
\begin{tabular}{|l|cc|cccccccc|} \hline
\multirow{4}{*}{\bf Quality driver} & \multicolumn{10}{c|}{\bf Model qualities} \\ \cline{2-11}
& & & \multicolumn{8}{c|}{Pragmatic model qualities}\\ \cline{4-11}
    & \begin{sideways}Semantic\end{sideways} & \begin{sideways}Syntactic\end{sideways}  & \begin{sideways}Simplicity\end{sideways}
    & \begin{sideways}Completeness\end{sideways} & \begin{sideways}Modifiability\end{sideways} & \begin{sideways}Consistency\end{sideways}
    & \begin{sideways}Decoupling\end{sideways} & \begin{sideways}Cohesion\end{sideways} & \begin{sideways}Reusability\end{sideways}
    & \begin{sideways}Traceability\end{sideways} \\ \hline
%                                       Semantic     Syntax        Simplicity  Completeness   Modifiable  Consistent  Decoupled    Cohesion     Reuse        Traceable
Metamodel and/or ontology              & \checkmark & \checkmark & \checkmark & \checkmark & \checkmark & \checkmark & \checkmark &            &            & \checkmark \\
Graphical and/or text grammar          &            & \checkmark & \checkmark &            & \checkmark &            &            &            &            &       \\ 
Manage levels of granularity           &            &            & \checkmark &            & \checkmark &            &            &            & \checkmark & \checkmark \\ 
Enforce single reponsibility principle &            &            & \checkmark &            & \checkmark &            &            & \checkmark & \checkmark & \checkmark \\ 
Service dependency via contracts       &            &            & \checkmark &            & \checkmark &            & \checkmark &            & \checkmark & \checkmark \\ 
Testable pre- \& post-conditions       &            &            &            & \checkmark & \checkmark & \checkmark &            &            &            &  \\ 
Process locacalization in controller   &            &            & \checkmark &            & \checkmark &            & \checkmark & \checkmark & \checkmark & \checkmark \\ 
Traceability Links                     &            &            & \checkmark & \checkmark & \checkmark & \checkmark &            &            &            & \checkmark \\ \hline 
\end{tabular}
  
\end{table}

%------------------------------------------------------------------------

\subsection{The URDAD process}



\missingfigure{URDAD process}

%------------------------------------------------------------------------

\subsection{The URDAD metamodel and text grammar}

Many of the model qualities are either enforced or supported through the metamodel This includes
\begin{itemize}
 \item the enforced decoupling of services across levels of granularity through services contracts,
 \item a range of traceability links including enforced usage of both satisfiability links and \emph{`requiredBy'} links between requirements and stakeholders,
 \item the assignment of services into responsibility domains, and 
 \item the support of parametrized, reusable constraints for the specification of pre- and post-conditions as well as guard conditions for conditional functional requirements and alternative flows within processes.
\end{itemize}



URDAD is a methodology generating a services-oriented requirements and business process design model. 



URDAD does not prescribe the language which should be used to encode the requirements and process model representing OMG's CIM, but it provides the option of using a domain specific language for URDAD, the \emph{URDDAD-DSL}. URDAD require, however, that the modelling language used supports the semantics for specifying 1) \emph{services contracts} with pre- and post-conditions, quality requirements and data structure specification for the inputs and outputs, 2) parametrized state constraints applying to service environment, 3) object-oriented data structure specification, 4) service specification including 5a) the specification of which service is used to address which functional requirement and 5b) the process specification in the form of a choreography across lower level services and 6) the assigning of model artifacts including services contract and service specifications to responsibility domains. URDAD also requires the linkage between a requirement and the stakeholder who requires it. The stakeholder can be a responsibility domain or a service.

The contract specification should facilitate the generation of tests and the process specification the generation of an implementation, either in the form of an automated system process or in the form of a manual process. The


Historically UML together with OCL and an URDAD profile were used with services contracts represented by interfaces, pre- and post-conditions and quality requirements by constraints, data structure specifications by class diagrams and process orchestration specified in activity diagrams assigned to service implementation within classes. The activity diagrams were allowed to only only call operations and control logic. The approach was, however, tedious and error prone and the resultant models were seldom in a state which allowed for model transformation, code and test generation and solid documentation generation. Furthermore, additional semantic relationships, not all of which could be represented in diagrams, needed to be added. Full testable service contract specification and complete process specification facilitating full test and code generation proved particularly challenging.

The \emph{Business Process Modeling Notation}, which is commonly used for process specification for services oriented architectures is on the one side too technical \cite{} insufficient for full process specification and needs to be supplemented by UML or other languages to allow for full data structure and service contract specification.


For these reasons we have developed a domain specific language (DSL) for URDAD. The DSL has significantly lower complexity which supports the required semantics for URDAD directly. 
URDAD encoding
UML+OCL can provide limited encoding but the
  - insufficient expressiveness for specifying services contracts in a services oriented framework
  - tedious and error prone
BPMN
  - too technology focused (citations from Fouad et al "Embedding requirements ..."
  - no notion of services contracts and data structures  

URDAD is a service-oriented analysis and design process which generates services contracts and associated process specifications across levels of granularity. 

Refer to \cite{gonzalez_unity_2009} to use unity criteria to define responsibility domain boundaries.
\todo{Fritz:Look at unity models as in Espana et al and see progress report for responsibility localization and general short-comings in use case based approach regarding this.}

\subsection{Example: Designing URDAD with URDAD}
by using it to design a service oriented analysis and design methodology. If the process is internally consistent it needs to generate itself. 

\subsection{URDAD views}

supplied via text encoding and later via graphical syntax.


adapters to existing legacy


Refer to \cite{wirfs-brock_object-oriented_1989}

One of the core challanges is to have requirements specialists make the paradigm shift to specify requirements in terms of services contracts for reusable services\cite{haines_impact_2007}. In practice we have found that URDAD assists significantly on that front.
