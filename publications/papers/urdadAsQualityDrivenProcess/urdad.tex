\section{URDAD}
\label{sec:urdad}

In this section we will introduce URDAD from the perspective of a quality driven process generating an analysis and design model which enforces certain model and process qualities.  URDAD\cite{solms_generating_2009} provides a service oriented analysis and design methodology, a metamodel defining the semantics and structure of an URDAD model, and a domain specific language (the \emph{URDAD-DSL}) with a concrete text grammar which can be used to construct an URDAD model. A graphical grammar and diagram-based tooling around that grammar are under development. 

The central subject of an URDAD design is a service for which the requirements analysis and the function design is done. Services use lower level services and the URDAD process is repeated for these lower level services. The process terminates when all services are decomposed into leaf services which are sourced from the environment. Leaf services may include persistence services, services sourced from execution frameworks, implementation languages and external systems as well as manually executed services provided by business partners, business units.

\begin{enumerate}
 \item {\bf Requirements analysis} for level of granularity yielding service contract
  \begin{enumerate}
    \item Identify stakeholders for the services.
    \item Specify service contract 
      \begin{itemize}
       \item For each stakeholder identify service requirements including pre- and post-conditions and quality criteria.
       \item Assess consistency of stakeholder requirements.
       \item Define levels of granularity by consolidating lower level functional requirements into higher level ones. 
       \item Specify data structures for request and result classes.
       \item Formalize pre and post-conditions by specifying a test for each. Specify for each pre-condition the exception which will be raised if the pre-condition is not met.
     \end{itemize}
    \item Assign service contract to appropriate responsibility domain.
  \end{enumerate}

 \item {\bf Function design} for level of granularity defining a service implementation.
      \begin{enumerate}
	\item Assess whether the requested service falls within scope for the context (e.g.\ system, system component, organization, business unit, \dots).
	\item Check within the responsibility domain of the service whether realizing the services contract. If so return that service.
        \item Identify for each functional requirement an abstract service (in the form of a service contract) to be used to address that functional requirement.
	\item If a new service is defined, assign it to the appropriate responsibility domain, ensuring cohesion of the responsibility domain. 
	\item Define the function through a process which uses the abstract services used to address the functional requirements. Each failed pre-condition validation leads to a terminal activity raising the exception associated with the precondition. All paths for which all preconditions are met lead to the terminal activity of returning the result object.
      \end{enumerate}

  \item Repeat the process for each lower level service which is not available.
\end{enumerate}

%\missingfigure{URDAD process}

The methodology does not envisage that a single requirements engineer does the requirements and function design across levels of granularity. Instead requirements engineers specializing in different responsibility domains (e.g.\ business analysts across business units of the organization) collaborate to define the complete requirements model.

URDAD includes the specification of a domain specific language, the URDAD-DSL \cite{solmsfritz_domain-specific_????} which defines the semantics and structure for an URDAD model in the form of a metamodel and provides a concrete text syntax used to construct an URDAD model.