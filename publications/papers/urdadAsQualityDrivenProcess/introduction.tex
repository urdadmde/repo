\section{Introduction}\label{sec:Introduction}

Inferior requirements quality remains a core contributor to software quality\cite{heck_experiences_2008,_strategies_2011}. Furthermore, early defect detection is critical for cost containment\cite{betterRefernceThanBoehm1981WhichReliesOnWaterfall}. Model-Driven Engineering (MDE) and the Model-Driven Architecture (MDA) \cite{frankel_model_2003} in particular aim to address this by making models the primary artifacts. The core models are the \emph{Compution Independent Model} (CIM) containing the business requirements and the \emph{Platform Independent Models} containing the technology and architecture neutral system requirements \cite{needAGoodCitation}. These are developed either by domain experts or requirements engineers who have solid domain knowledge and not by technical experts. The platform independent model is enriched with architecture and technology information an the system artifacts (e.g.\ code) are generated either manually by developers or in an automated way using model-transformation and code generation technologies. In an MDA-based approach continuous quality improvement is focused on improving the requirements model and architecture specification as well as the transformation components used to generate the system artifacts.

\emph{Cuen: I think "Inferior requirements quality remains a core contributor to software quality" should rather be something to the effect of either:}

\begin{itemize}
  \item \emph{"Inferior requirements remain a core contributor to a lack of quality within software"}
  \item \emph{"Inferior requirements quality remains a core contributor to a lack of software quality"}
  \item \emph{"Inferior requirements quality remains a core contributor to poor software quality"}
\end{itemize}

\emph{Reasoning: the word 'quality', on its own, is often understood 'in the positive' or, in other words, 'as the presence of quality'.}

URDAD (Use-Case, Responsibility Dirven Analysis and Design) \cite{solms_technology_2007} is a services oriented methodology which generates MDA's PIM \cite{solms_generating_2009}. Note, however that URDAD collapses the CIM and PIM into a single model which is conceptually closer to the CIM but has the required information to act as PIM. The services contracts and processes included specified in the URDAD model may be deployed into a system or may be manually executed. The decision on whether the process is meant to be manual or automated is left to the architecture and implementation mapping phases. 

Formal methods aim to provide requirements and solutions which can be proven to be correct[14]. The complexity, high skills requirements and cost of these methods \cite{} has confined their use largely to a relatively small class of problems like aviation \cite{Hall} for which quality is paramount. This has resulted in a range of semi-formal methods which introduce a level of formality to improve quality whilst constraining complexity and cost \cite{}.

URDAD is a semi-formal method to technology-neutral analysis and design \cite{solms_urdad_2010} which we are in the process of making incrementally more formal. It aims to provide full testability and partial validatability, i.e.\ it supports partial model validation and full service test generation. Historically URDAD models were encoded in UML, but in order to reduce the complexity, tighten up the semantics and enforce certain model qualities through a rigid model structure, we have developed a domain specific language (DSL) for URDAD with a defined metamodel with a concrete text syntax \cite{solmsfritz_domain-specific_????}. Through this we aim to reduce the burden on requirements specialists and improve the model quality. Model qualities like completeness, consistency, traceability, reusability, testability, \dots typically result in code with similar quality attributes\cite{findItIfYouCan}.

\emph{This might be a little controversial:} URDAD provides an environment (an architecture) to perform the analysis and design. Like any architecture, it needs to provide a suitable infrastructure to support the quality attributes for the processes which are deployed in the architecture \cite{}, i.e.\ the architecture needs to be designed in order to address the quality requirements. The infrastructure includes the process definition for services oriented analysis and design as well as a metamodel enforcing the structure of the resultant analysis and design model.

Quality is often a subjective measure and is dependent on the quality measures. For the purpose of this paper we take a manufacturing view \cite{garvin_what_1984} to quality where we assess quality from the perspective of conformance to requirements, i.e. to what extend both, the functional and non-functional (quality) requirements are realised. We look look at URDAD from the perspective of both, process and model quality. We identify the stake holders and their quality requirements. For each quality requirement we identify quality drivers and analyze whether and how these quality drivers are embedded within the URDAD process. Finally we aim to identify a set of model quality measures and assess an URDAD model against these.


