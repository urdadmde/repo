\section{Introduction}
\label{sec:Introduction}

Inferior requirements quality remains a core contributor to poor software quality\cite{heck_experiences_2008}. Early defect detection is critical for cost containment\cite{biffl_software_2000}. Formal methods aim to address this through precise requirements specifications allowing for formal proofing of solutions \cite{hinchey_software_2008}. However, the complexity, high skill demands, and cost of these methods have confined their use largely to a relatively small class of problems for which correctness is paramount. This has resulted in a range of semi-formal methods which introduce a level of formality to improve quality whilst constraining complexity and cost \cite{razali_usability_2008}.

Model-Driven Engineering (MDE) \cite{frankel_model_2003} approaches fall within the class of semi-formal methods. The Model-Driven Architecture (MDA) is the MDE initiative of the Object Management Group (OMG). The initial (business) requirements model is the \emph{Computation Independent Model} (CIM) which is refined into a systems requirements model in the form of the \emph{Platform Independent Model} (PIM) \cite{_mda_2003}. In an MDE-based approach, continuous quality improvement is focused on improving the quality of requirements models, architecture specifications and the supporting model-driven methodologies.

URDAD (Use-Case, Responsibility-Driven Analysis and Design) \cite{solms_technology_2007} is a service-oriented methodology which generates an URDAD model. The URDAD model is technology-neutral, service oriented analysis and design model which has a specific structure and semantics as defined by a metamodel. The URDAD model represents MDA's CIM \cite{solms_generating_2009} with sufficient detail and formality for it to be usable as PIM. The formalization of URDAD is an ongoing project \cite{solms_urdad_2010}. We have also defined domain-specific language for URDAD (the \emph{URDAD-DSL}) with a formal text grammar through which syntactically correct URDAD models are specified. The resultant models support service test generation and partial validation.

This paper looks at URDAD from the perspective of both process and model quality. We identify the stakeholders and their quality requirements for both, the process and the resultant requirements model. For each quality criterion we identify quality-drivers and analyze whether and how these quality-drivers are embedded within the URDAD process. Finally we demonstrate the internal consistency of the methodology by using it to design a service-oriented analysis and design methodology and showing how the methodology generates its own metamodel and process.
