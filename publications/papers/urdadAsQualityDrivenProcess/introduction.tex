\section{Introduction}\label{sec:Introduction}

The Oxford dictionary defines quality as `the standard of something as measured against other things of a similar kind' or the `degree of excellence of something'. The concept of quality is intrinsically subjective - hence we have the saying that `quality is in the eye of the beholder'. For the purpose of this paper we define quality is a measure to which a solution fulfills the stakeholder functional and non-functional requirements\footnote{Non-functional requirements are often called quality requirements\cite{}. However, the degree to which a system fulfills its functional requirements is also an aspect of quality. In order to avoid confusion we do not use the term `quality requirements' in this paper} \cite{lange_managing_2005,lange_improving_2006}.


In the context of an analysis and design methodology one needs to consider both, the quality of the methodology or process as well as the quality of the resulting analysis and design model. It is questionable whether a process which yields low model quality can itself be viewed of high quality as the purpose of the process is to generate the model. On the other hand, one may have different processes which yield similar model qualities but with very different process qualities. 

In order to assess quality, we need to measure it. This paper will introduce quality measures for the design