\section{Introduction}\label{sec:Introduction}

Inferior requirements quality remains a core contributor to poor software quality\cite{heck_experiences_2008,_strategies_2011}. Furthermore, early defect detection is critical for cost containment\cite{betterReferenceThanBoehm1981WhichReliesOnWaterfall}. Model-Driven Engineering (MDE) \cite{frankel_model_2003} aims to address this by making models the primary artifacts for specifying requirements. In particular, the Object Management Group's MDE initiative, Model-Driven Architecture (MDA) defines two core models for this purpose: the \emph{Computation Independent Model} (CIM), and the \emph{Platform Independent Model} (PIM). The CIM is a domain model for specifying business requirements independent of systems and processes. The PIM goes a step further and specifies system requirements independent, however, of implementation platform \cite{needAGoodCitation}. Technical experts then map the specification provided by the PIM onto a target platform, creating a {Platform Specific Model} (PSM).  Actual system artifacts are generated either manually by developers or in an automated fashion using model-transformation or code-generation technologies. In an MDE-based approach continuous quality improvement is focused on improving the requirements model and architecture specification as well as the transformation components used to generate the system artifacts.

URDAD (Use-Case, Responsibility-Driven Analysis and Design) \cite{solms_technology_2007} is a services-oriented methodology which generates MDA's PIM \cite{solms_generating_2009}. Note, however, that URDAD collapses the CIM and PIM into a single model which is conceptually closer to the CIM but has the required information to act as the PIM. The services contracts and processes specified in the URDAD model may be deployed into a system or may be manually executed. The decision on whether the process is manual or automated is left to the architecture and implementation mapping phases. 

Formal methods aim to provide requirements and solutions which can be proven to be correct[14]. The complexity, high skills requirements, and cost of these methods \cite{}, has confined their use largely to a relatively small class of problems like aviation \cite{Hall} for which quality is paramount. This has resulted in a range of semi-formal methods which introduce a level of formality to improve quality whilst constraining complexity and cost \cite{}.

URDAD is a semi-formal method for technology-neutral analysis and design \cite{solms_urdad_2010} which is being incrementally refined to become more formal. It aims to provide full testability and partial validatability, i.e.\ it supports partial model validation and full service test generation. Historically URDAD models were encoded in UML, but in order to reduce the complexity, tighten up the semantics, and enforce various model qualities through a rigid model structure, a domain specific language (DSL) has been developed for URDAD, along with a defined metamodel, and a concrete text syntax \cite{solmsfritz_domain-specific_????}. This aims to reduce the burden on requirements specialists and improve model quality. Model qualities like completeness, consistency, traceability, reusability, testability, \dots typically result in systems with similar quality attributes\cite{findItIfYouCan}.

\emph{This might be a little controversial: URDAD provides an architecture (or environment) in which to perform analysis and design. Like any architecture, it needs to provide a suitable infrastructure to support the quality attributes for the processes which are deployed within \cite{}, i.e.\ the architecture needs to be designed in order to address the quality requirements. The infrastructure includes the process definition for services-oriented analysis and design, as well as a metamodel for enforcing the structure of the resultant analysis and design model.}

%Cuen: I think this last paragraph needs refinement - I will look at it further
Quality is often a subjective measure and is dependent on the quality measures. For the purpose of this paper, we take a manufacturing view \cite{garvin_what_1984} to quality where we assess quality from the perspective of conformance to requirements, i.e. to what extend both, the functional and non-functional (quality) requirements are realised. We look at URDAD from the perspective of both process and model quality. We identify the stakeholders and their quality requirements. For each quality requirement we identify quality drivers and analyze whether and how these quality drivers are embedded within the URDAD process. Finally we aim to identify a set of model quality measures and assess an URDAD model against these.