\section{Introduction}
\label{sec:Introduction}

Inferior requirements quality remains a core contributor to poor software quality\cite{heck_experiences_2008}. Early defect detection is critical for cost containment\cite{biffl_software_2000}. Formal methods aim to address this through precise requirements specifications allowing for formal proofing of solutions \cite{hinchey_software_2008}. However, the complexity, high skill requirements, and cost of these methods have confined their use largely to a relatively small class of problems for which quality is paramount. This has resulted in a range of semi-formal methods which introduce a level of formality to improve quality whilst constraining complexity and cost \cite{razali_usability_2008}.

Model-Driven Engineering (MDE) \cite{frankel_model_2003} approaches fall within the class of semi-formal methods with incremental formalization largely driven by formalizing the models. The Model-Driven Architecture (MDA) is the MDE initiative of the Object Management Group (OMG). The initial (business) requirements model is the \emph{Computation Independent Model} (CIM) which is refined into systems a requirements model int the form of the \emph{Platform Independent Model} (PIM) \cite{_mda_2003}. In an MDE-based approach, continuous quality improvement is focused on improving the quality of the requirements models, the architecture specification and the supporting model-driven methodologies.

URDAD (Use-Case, Responsibility-Driven Analysis and Design) \cite{solms_technology_2007} is a services-oriented methodology which generates MDA's CIM \cite{solms_generating_2009} which has sufficient detail and formality to be used as PIM. The methodology is in the process of being incrementally formalized \cite{solms_urdad_2010}. We have defined a metamodel introducing the URDAD semantics and a a domain-specific language for URDAD (the \emph{URDAD-DSL}) with a text grammar ensuring syntactic quality of model instances. The resultant models provide full testability and partial validatability, i.e.\ it supports partial model validation and full service test generation. 

In this paper we look at URDAD from the perspective of both process and model quality. We identify the stakeholders and their quality requirements for both, the process and the resultant requirements model. For each quality requirement we identify quality-drivers and analyze whether and how these quality-drivers are embedded within the URDAD process. Finally we demonstrate the internal consistency of the methodology by using it to design a services oriented analysis and design methodology and showing how the methodology generates its own metamodel and process.
