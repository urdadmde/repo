\documentclass{IOS-Book-Article}

\usepackage{times}
\normalfont
\usepackage[T1]{fontenc}
%\usepackage[mtplusscr,mtbold]{mathtime}

\usepackage{graphicx}
\usepackage{epsfig}
\usepackage{listings}
\usepackage{color}


% Additionally a command to generate the list of todos.
\makeatletter \newcommand \listoftodos{\section*{Todo list} \@starttoc{tdo}}
\newcommand\l@todo[2]
  {\par\noindent \textit{#2}, \parbox{10cm}{#1}\par} \makeatother

\lstdefinelanguage{urdad}
{
keywords=
  {Model,ResponsibilityDomain,Query,Constraint,QualityConstraint,FunctionalRequirements,receiving,yielding,
  StateConstraint,stateAssessmentProcess,InverseConstraint,inverseOf,AndConstraint,AND,OrConstraint,OR,
  XorConstraint,XOR,from,to,many,BasicDataType,DataStructure,is,abstract,has,Variable,ofType,Constant,
  ValueOf,Exception,attribute,identification,identifying,association,linking,aggregate,component,
  QualityRequirement,requiredBy,constraint,with,constructedUsing,ResultConstraint,PreCondition,
  raises,checks,PostCondition,ensures,use,toAddress,if,ServiceContract,undoneUsing,Request,Result,
  Service,realizes,doSequential,choice,else,doConcurrent,blocking,Concurrency,wait,until,create,set,
  equalTo,add,remove,requestService,on,raiseException,returnResult,while,do,forAll,Note},%
sensitive=true,%
alsoletter={\$},%
comment=[l]{\#},%
string=[b]",%
string=[b]'%
}

%\definecolor{OliveGreen}{cmyk}{0.64,0,0.95,0.40}
%\definecolor{CadetBlue}{cmyk}{0.62,0.57,0.23,0}
\definecolor{lightgray}{gray}{0.9}
\lstset{
language=urdad,  
basicstyle=\ttfamily\small,
keywordstyle=\itshape\color{blue},
%keywordstyle=\color{blue},        % Keywords font ('*' = uppercase)
commentstyle=\color{gray},           
numbers=left,                           % Line nums position
numberstyle=\tiny,                      % Line-numbers fonts
stepnumber=1,                           % Step between two line-numbers
numbersep=5pt,                          % How far are line-numbers from code
backgroundcolor=\color{lightgray}, % Choose background color
frame=none,                             % A frame around the code
tabsize=2,                              % Default tab size
captionpos=b,                           % Caption-position = bottom
breaklines=true,                        % Automatic line breaking?
breakatwhitespace=false,                % Automatic breaks only at whitespace?
showspaces=false,                       % Dont make spaces visible
showtabs=false,                         % Dont make tabls visible
columns=flexible,                       % Column format
%morekeywords={__global__, __device__},  % CUDA specific keywords
}

%
\begin{document}
\begin{frontmatter} 

\title{URDAD as a Quality-Driven Analysis and Design Process}
\thanks{We thank the national research foundation (NRF) of South Africa for financial support.}
\runningtitle{URDAD as Quality Driven Methodology}
%\subtitle{Subtitle}

\author[A]{\fnms{Fritz} \snm{Solms}}
,
\author[A]{\fnms{Stefan} \snm{Gruner}}
and
\author[A]{\fnms{Cuen} \snm{Edwards}}

\runningauthor{F. Solms et al.}
\address[A]{Department of Computer Science, University of Pretoria, South Africa}

\begin{abstract}
This is the amazing abstract. 

\end{abstract}

\begin{keyword}
Services-oriented design methodology\sep Model-driven development\sep Design quality\sep Model quality
\end{keyword}
\end{frontmatter}

\thispagestyle{empty}
\pagestyle{empty}

\maketitle

\listoftodos

\section{Outstanding issues}
\begin{itemize}
 \item Stefan: Discuss Quality Categories. 
 \item Fritz: Look at unity models as in Espana et al and see progress report for responsibility localization and general short-comings in use case based approach regarding this. 
 \item Fritz: INSERT REF - Quality in Model-Driven Engineering argues model quality determined by 1,2) modeling language and tools used to define model, 3) modeling process itself, 4) techniques used to assure quality 5)relative experience of individuals building model. 
\end{itemize}

\section{Introduction}\label{sec:Introduction}

Inferior requirements quality remains a core contributor to software quality\cite{heck_experiences_2008,_strategies_2011}. Furthermore, early defect detection is critical for cost containment\cite{betterRefernceThanBoehm1981WhichReliesOnWaterfall}. Model-Driven Engineering (MDE) and the Model-Driven Architecture (MDA) in particular aim to address this by making the models the primary artifacts which are to be developerd by requirements engineers who have solid domain knowledge and not by technical experts. The platform independent model is enriched with architecture and technology information an the system artifacts (e.g.\ code) are generated either manually by developers or in an automated way using model-transformation and code generation technologies.

URDAD (Use-Case, Responsibility Dirven Analysis and Design) \cite{solms_technology_2007} is a methodology which generates MDA's PIM \cite{solms_generating_2009}. It is a semi-formal \cite{solms_urdad_2010} approach to technology-neutral analysis and design which we are in the process of making incrementally more formal. Historically URDAD models were encoded in UML, but in order to reduce the complexity, tighten up the semantics and enforce certain model qualities through a rigid model structure, we have developed a domain specific language (DSL) for URDAD with a defined metamodel with a concrete text syntax \cite{solmsfritz_domain-specific_????}..

\emph{This might be a little controversial:} URDAD provides an environment (an architecture) to perform the analysis and design. Like any architecture, it needs to provide a suitable infrastructure to support the quality attributes for the processes which are deployed in the architecture \cite{}, i.e.\ the architecture needs to be designed in order to address the quality requirements.

In this paper we identify the stake holders who have an interest in the process and its outputs, their quality requirements and

This can be viewed as an architecture within which one can realize the quality requirements for the analysis and design process can be realized. The infrastructure includes the process definition for services oriented analysis and design as well as a metamodel enforcing the structure of the resultant analysis and design model.


In this paper we aim to identify quality requirements for the requirments model itself, identify quality drivers for the requirements model and analyze how these quality drivers are embedded in the URDAD methodology.

Nodel driven development aims to address this by making the requirements model

URDAD collapses CIM and PIM

URDAD is a process for architecture neutral analysis and design resulting in a model which does not address any architectural concerns, yet it is itself an architecture for doing the requirements process

The concept of quality implies some form of either absolute or relative measurement, i.e.\ something is of high quality yielding good results against some quality measures or it proofs to be of better quality when compared with an alternative. 
 The concept of quality is intrinsically subjective - hence we have the saying that `quality is in the eye of the beholder'. For the purpose of this paper we define quality is a measure to which a solution fulfills the stakeholder functional and non-functional (quality) requirements\cite{}. However, the degree to which a system fulfills its functional requirements is also an aspect of quality.  \cite{lange_managing_2005,lange_improving_2006}.


In the context of an analysis and design methodology one needs to consider both, the quality of the methodology or process as well as the quality of the resulting analysis and design model. It is questionable whether a process which yields low model quality can itself be viewed of high quality as the purpose of the process is to generate the model. On the other hand, one may have different processes which yield similar model qualities but with very different process qualities. 

In order to assess quality, we need to measure it. This paper will introduce quality measures for the design

Refer to \cite{wirfs-brock_object-oriented_1989}

One of the core challanges is to have requirements specialists make the paradigm shift to specify requirements in terms of services contracts for reusable services\cite{haines_impact_2007}. In practice we have found that URDAD assists significantly on that front.



Quality requirements for process  by infrastructure within which process executed, model qualities driven out by model infrastructure (metamodel) and system qualities ultimately by system architecture

process assists in improving quality of requirements by forcing certain questions

because requirements and design are arch and techn neutral, stakeholders (business) can understand them and validate them, improving quality

adapters to existing legacy

urdad as a semi formal method aims for full testability and partial validatability.

Formal methods complex, expensive, high skills, inflexible. Could be applied to certain critical services.

 
Yes, URDAD provides an architecture (infrastructure) for doing the analysis and design work - hence it addresses quality requirements. Any architecture needs to be quality driven

\section{Quality requirements}
\label{sec:qualityRequirements}


For the purpose of this paper, we take a manufacturing view \cite{garvin_what_1984} to quality, i.e.\ we assess quality from the perspective of conformance to requirements. This section discusses the stakeholders who have an interest in the model and the process generating the model and their respective quality requirements.

%-------------------------------------------------

\subsection{Model stakeholders and their quality requirements}
\label{sec:modelStakeholdersAndQualityRequirements}


A lot of work has historically been done on defining and measuring code quality (see, for example, \cite{boehm_barry_w._characteristics_1978}). Recently the focus has shifted onto understanding and measuring model quality \cite{lange_managing_2005,lange_improving_2006,shim_design_2008,qi_yu-dong_analysis_2010}. Though most of the work is applied to UML models, the concepts are generally applicable across analysis and design models, i.e.\ they are generally not UML specific. 

Lindland\cite{lindland_understanding_1994} provides a widely used quality categorization into \emph{semantic quality}, which assesses whether a model confers the meaning it was meant to confer, \emph{syntactic quality}, which is a measure of the conformance to the language used to specify the model, and \emph{pragmatic quality}, which is a measure of the extent to which the model can satisfy its intended use. Christiaan Lange  \cite{lange_christiaan_assessing_2007} related model quality back to model purpose and identified a range of pragmatic model quality characteristics including complexity, traceability, modularity, completeness, consistency and communicativeness. Models that exhibit such qualities have been shown to commonly lead to systems with similar quality attributes \cite{podgorelec_estimating_2007}.

In this paper we do not consider communicativeness and social quality as we feel these will be strongly influenced by both, the semantics and the model user interface(s). For example, requirements specialists would typically use a diagrammatic user interface into the model provided by a graphical grammar and appropriate tool support. The client model interface is typically a read-only interface in the form of documentation generated from the model.

The core stakeholders in the model include {\bf requirements engineering} (e.g.\ business analysis), {\bf architecture}, {\bf implementation} (e.g.\ developers), {\bf quality assurance} (e.g.\ testing), {\bf project management} and the {\bf client} (e.g.\ business). In order to address their respective responsibilities, these stakeholders require certain model qualities. All of these stakeholders require \emph{semantic quality} (since they need to extract the correct meaning from the model). The syntactic and pragmatic quality requirements differ across stakeholders, though \emph{simplicity} and \emph{consistency} are qualities which are beneficial to all.

{\bf Requirements engineers} (e.g.\ business analysts) use the model to capture the functional and non-functional requirements of services in the form of services contracts and (business) process designs realizing these services contracts. They also may use the model for model validation and documentation generation, requiring \emph{syntactic quality} to make this feasible.  The \emph{pragmatic qualities} particularly relevant to requirements engineers include \emph{modifiability} (as they would have to modify the model in the context of changing requirements and process improvement), \emph{simplicity} assisting understandability and modifiability, \emph{completeness} for complete requirements and process designs, model \emph{consistency}, \emph{decoupling} of functional requirements from process design, \emph{reuse} to reduce model complexity and increase model consistency, and \emph{traceability} to facilitate model validation and improve \emph{modifiability}. 

For the purpose of this paper, we define architecture as the infrastructure within which processes are deployed and executed. Thus, whilst implementation is concerned with the actual services implementations, the architecture team is concerned with the infrastructure hosting the services. The architecture may span organizational, hardware and software infrastructure. Organizational and systems architects need to be able to design an architecture hosting the services in such a way that the quality requirements (\emph{quality of service}) can be met. They also need to be able to assess whether an existing architecture can host a service as per contract and specify architectural modifications if that is not the case. {\bf Architecture} typically uses the high level services contracts and process specifications. These need to be \emph{complete} and \emph{consistent}. \emph{Simplicity} \emph{cohesion} and \emph{decoupling} make it easier for architects to obtain the required information.

Within in an MDA-based approach, {\bf implementation} (e.g.\ developers or managers who implement and control manual processes) perform the implementation mapping of the services onto a realization within an architecture specified by the architecture team. To this end, they require \emph{syntactic quality} as well as \emph{completeness} and \emph{consistency}. Furthermore, model \emph{simplicity}, \emph{decoupling}, \emph{reuse} and \emph{traceability} typically lead to implementations with those same qualities \cite{podgorelec_estimating_2007}.

{\bf Quality assurance} needs to span across the full development process including quality assurance on the process itself, on the outputs of the different analysis and development activities, and detect and report quality defects in service providers. They require the services contracts specifying both functional and quality requirements for services. \emph{Testability}, \emph{syntactic quality}, \emph{completeness} and \emph{consistency} are required for test generation. Potential service implementations would be tested against the respective service contract. \emph{Traceability} is required for impact reporting.

{\bf Project Management} uses the model measures for reporting and estimation purposes. \emph{Completeness} and \emph{traceability} are required for status reporting on the implementation mapping.

The {\bf client} (business) uses the model to obtain (business) process documentation and services contract documentation. Business process documentation is useful for the general understanding and running of the business as well as for business process optimization. Documentation generation requires \emph{completeness}, \emph{consistency} and \emph{syntactic quality}. \emph{Decoupling} via services contracts is important to the client in the context of sourcing and assessing different concrete service providers (e.g.\ business partners, off-the-shelf systems). More generally, the client commonly requires modifiability in order to have requirements changes cost-effectively realized, benefits from \emph{service reuse} in the context of cost containment and consistency, and requires \emph{traceability} for dependency analysis.

\begin{table}[h]
\caption{Stakeholders and their primary model quality requirements.}
\label{tab:modelQualityRequirements}
\begin{tabular}{|l|cc|cccccccc|} \hline
\multirow{4}{*}{\bf Stakeholders} & \multicolumn{10}{c|}{\bf Model qualities} \\ \cline{2-11}
& & & \multicolumn{8}{c|}{Pragmatic model qualities}\\ \cline{4-11}
    & \begin{sideways}Semantic\end{sideways} & \begin{sideways}Syntactic\end{sideways}  & \begin{sideways}Simplicity\end{sideways}
    & \begin{sideways}Completeness\end{sideways} & \begin{sideways}Modifiability\end{sideways} & \begin{sideways}Consistency\end{sideways}
    & \begin{sideways}Decoupling\end{sideways} & \begin{sideways}Cohesion\end{sideways} & \begin{sideways}Reusability\end{sideways}
    & \begin{sideways}Traceability\end{sideways} \\ \hline
%                         Semantic     Syntax        Simplicity  Completeness   Modifiable  Consistent  Decoupled    Cohesion     Reuse        Traceable
Requirements engineering & \checkmark & \checkmark & \checkmark & \checkmark & \checkmark & \checkmark & \checkmark & \checkmark & \checkmark & \checkmark \\
Architecture             & \checkmark &            & \checkmark & \checkmark &            & \checkmark & \checkmark & \checkmark &            &       \\ 
Implementation           & \checkmark & \checkmark & \checkmark & \checkmark &            & \checkmark & \checkmark & \checkmark & \checkmark & \checkmark \\ 
Quality assurance        & \checkmark & \checkmark & \checkmark & \checkmark &            & \checkmark &            &       &            & \checkmark \\ 
Project management       & \checkmark &            & \checkmark & \checkmark &            & \checkmark &            &       &            & \checkmark \\ 
Client (business)        & \checkmark & \checkmark & \checkmark & \checkmark & \checkmark & \checkmark & \checkmark &            & \checkmark & \checkmark \\ \hline
\end{tabular}
\end{table}

%---------------------------------------------------------------------------------------------------------%

\subsection{Process stakeholders and their quality requirements}

The {\emph Capability Maturity Model} (CMM) \cite{paulk_capability_1993} defines a range of quality attributes for mature processes including \emph{process definition}, \emph{measurability}, \emph{trainability}, \emph{repeatability} and at the higher level the ability to continuously optimize processes for \emph{cost} and other qualities. Berard\cite{berard_what_1995} discusses a related set of general requirements for a methodology including \emph{repeatability}, \emph{trainability}, \emph{wide process applicability}, and quality improvement in the outputs of the methodology. The quality requirements for the outputs of the process have been discussed in \ref{sec:modelStakeholdersAndQualityRequirements}. In this text, the applicability of process has been limited to the services-oriented application domains. Very event-centric projects like game development are not currently targeted service-oriented methodologies. Here we identify the stakeholders in the process itself and their quality requirements for the process. 

{\bf Project management} needs to perform planning, estimation, resource management, monitoring, control and reporting during the analysis and design process. To this end, they require \emph{estimatability}, \emph{repeatability} for accurate estimation and \emph{measurably} and tool support facilitating estimation, monitoring, and reporting.

{\bf Requirements engineers} execute the process itself generating the analysis and design model. They requires a process which is \emph{usable} and \emph{trainable}, and \emph{repeatable}.

The {\bf client} (e.g.\ business) requires \emph{low process cost}, a \emph{trainable} process, enabling the client to easily train up additional requirements engineers, and \emph{isolation}. Process isolation refers here to the ability to grow islands of expertise around the process, extracting value from those teams which apply the process without requiring that the process needs to be rolled out either across an entire project team or across an entire organization. This enables the client to take a risk-averse, incremental approach which allows for proofing, customization and controlled roll-out.

A further quality required of a process is internal consistency. This quality will be discussed in section \ref{sec:urdadConsistency}.

\begin{table}[h]
\caption{Stakeholders and their primary process quality requirements.}
\label{tab:processQualityRequirements}
\begin{tabular}{|l|ccccccc|} \hline
\multirow{2}{*}{\bf Stakeholders} & \multicolumn{7}{c|}{\bf Process qualities} \\ \cline{2-8}
    & \begin{sideways}Low cost\end{sideways}  & \begin{sideways}Repeatability\end{sideways} & \begin{sideways}Estimatability\end{sideways}
    & \begin{sideways}Trainable\end{sideways}
    & \begin{sideways}Measurability\end{sideways} & \begin{sideways}Consistency\end{sideways} & \begin{sideways}Isolation\end{sideways} \\ \hline
%                          Cost         Repatable    Estimatable  Trainable    Measurable  Consistent   Isolated
Project management       &            & \checkmark & \checkmark & \checkmark &            & \checkmark & \checkmark \\
Requirements engineering & \checkmark & \checkmark & \checkmark & \checkmark & \checkmark & \checkmark & \checkmark \\
Client (business)        & \checkmark & \checkmark & \checkmark & \checkmark & \checkmark & \checkmark \\ \hline
\end{tabular}
\end{table}


\section{The URDAD analysis and design process}

In this section we will introduce URDAD from the perspective of a quality driven process generating an analysis and design model which enforces certain model quality requirements.  URDAD\cite{solms_generating_2009} provides a service oriented analysis and design methodology, a metamodel introducing the semantics (modelling constructs) for an URDAD based domain specific language (the \emph{URDAD-DSL}), as well as a concrete text grammar which can be used to populate an URDAD model complying to the metamodel. A graphical grammar and diagram-based tooling around that grammar are under development. 

The resultant URDAD model represents a \emph{Computation Independent Model} (CIM) as the services contracts and processes can be realized as system  or business contracts (SLAs) and system or manual processes respectively. Nevertheless, the model has a level of detail and preciseness which is generally associated with \emph{Platform Independent Models} (PIMs) and not commonly with CIMs as it contains testable service contracts and and implementable process specifications for services across levels of granularity.

%-----------------------------------------------------------

\subsection{Quality drivers employed by URDAD}

Most of the quality drivers discussed in \ref{sec:modelQualityDriversAndMetrics} are builts into the URDAD process. Table \ref{tab:qualityDrivers} lists the employed quality drivers and the model qualities they are meant to support.

\begin{table}[h]
 \caption{URDAD model quality drivers for quality requirements}
 \label{tab:qualityDrivers}
\begin{tabular}{|l|cc|cccccccc|} \hline
\multirow{4}{*}{\bf Quality driver} & \multicolumn{10}{c|}{\bf Model qualities} \\ \cline{2-11}
& & & \multicolumn{8}{c|}{Pragmatic model qualities}\\ \cline{4-11}
    & \begin{sideways}Semantic\end{sideways} & \begin{sideways}Syntactic\end{sideways}  & \begin{sideways}Simplicity\end{sideways}
    & \begin{sideways}Completeness\end{sideways} & \begin{sideways}Modifiability\end{sideways} & \begin{sideways}Consistency\end{sideways}
    & \begin{sideways}Decoupling\end{sideways} & \begin{sideways}Cohesion\end{sideways} & \begin{sideways}Reusability\end{sideways}
    & \begin{sideways}Traceability\end{sideways} \\ \hline
%                                       Semantic     Syntax        Simplicity  Completeness   Modifiable  Consistent  Decoupled    Cohesion     Reuse        Traceable
Metamodel and/or ontology              & \checkmark & \checkmark & \checkmark & \checkmark & \checkmark & \checkmark & \checkmark &            &            & \checkmark \\
Graphical and/or text grammar          &            & \checkmark & \checkmark &            & \checkmark &            &            &            &            
& \\
Fix levels of granularity              &            &            & \checkmark &            & \checkmark &            &            &            &
\checkmark & \checkmark \\ 
Enforce single reponsibility principle &            &            & \checkmark &            & \checkmark &            &            & \checkmark & \checkmark & \checkmark \\ 
Service dependency via contracts       &            &            & \checkmark &            & \checkmark &            & \checkmark &            & \checkmark & \checkmark \\ 
Testable pre- \& post-conditions       &            &            &            & \checkmark & \checkmark & \checkmark &            &            &            &  \\ 
Process localization in controller     &            &            & \checkmark &            & \checkmark &            & \checkmark & \checkmark & \checkmark & \checkmark \\ 
Traceability Links                     &            &            & \checkmark & \checkmark & \checkmark & \checkmark &            &            &            & \checkmark \\ \hline 
\end{tabular}
  
\end{table}

%------------------------------------------------------------------------

\subsection{The URDAD process}
\label{sec:urdadProcess}

\begin{enumerate}
 \item {\bf Requirements analysis} for level of granularity yielding service contract
  \begin{enumerate}
    \item Identify stakeholders for the services.
    \item For each stakeholder, identify functional requirements (pre- and post-conditions) and quality requirements for the service.
    \item Assess consistency of stakeholder requirements.
    \item Fix level of granularity by consolidating lower level pre- and post-conditions into higher level pre- and post-conditions. 
	  \emph{\textbf{\textit{Note:}} This includes the absorption of certain pre and/or post-conditions into encompassing pre- or post-conditions and the identification of a higher level pre- or post-condition which groups the some of the specified pre- and/or post-conditions into a higher level functional requirement addressing a at some level of granularity a single responsibility (enforcing the single responsibility principle). The purpose of this process is to project out levels of granularity in a repeatable way, simplifying the process at a particular level of granularity and improving reuse and cohesion.}
    \item Specify data structure for request and result classes.
    \item Formalize the pre and post-conditions by specifying the user test process for each and for each pre-condition the exception which will be raised if the pre-condition is not met.
    \item Identify the responsibility domain for the service and assign the services contract to that responsibility domain making sure that an appropriate service contract has not already been assigned to that responsibility domain. 
  \end{enumerate}

 \item {\bf Process design} for level of granularity defining a service implementation.
      \begin{enumerate}
	\item Assess whether the requested service falls within scope for the context (e.g.\ system, system component, organization, business unit, \dots).
	\item Check within the responsibility domain of the service whether realizing the services contract. If so return that service.
        \item Identify for each functional requirement an abstract service (in the form of a service contract) to be used to address that functional requirement.
	\item If a new service is defined, assign it to the appropriate responsibility domain, ensuring cohesion of the responsibility domain. \emph{\textbf{\textit{Note:}} Responsibility domains contain lower level responsibility domains and the service needs to be assigned to a responsibility domain at the appropriate level of granularity.}
	\item Choreograph the process across the abstract services used to address the functional requirements with each pre-condition assessment leading potentially leading to a terminal activity of raising an exception and all other paths leading to the terminal activity of returning the result object. \emph{\textbf{\textit{Note:}} The services across levels of granularity are decoupled through services contracts.}
      \end{enumerate}

  \item Recursively repeat the process for each lower level service which is not available.
\end{enumerate}

%\missingfigure{URDAD process}

The methodology does not envisage that a single requirements engineer does the requirements and process design across levels of granularity. Instead requirements engineers specializing in different responsibility domains (e.g.\ business analysts across business units of the organization) collaborate to define the complete requirements model.

%------------------------------------------------------------------------

\subsection{The URDAD metamodel and text grammar}
\label{sec:urdadDsl}

URDAD does not prescribe the language which should be used to encode the requirements and process model representing OMG's CIM, but it provides the option of using a domain specific language for URDAD, the \emph{URDDAD-DSL}. URDAD require, however, that the modelling language used supports the semantics for specifying 1) \emph{services contracts} with pre- and post-conditions, quality requirements and data structure specification for the inputs and outputs, 2) parametrized state constraints applying to service environment, 3) object-oriented data structure specification, 4) service specification including 5a) the specification of which service is used to address which functional requirement and 5b) the process specification in the form of a choreography across lower level services and 6) the assigning of model artifacts including services contract and service specifications to responsibility domains. URDAD also requires the linkage between a requirement and the stakeholder who requires it. The stakeholder can be a responsibility domain or a service.

Historically UML together with OCL and an URDAD profile were used with services contracts represented by interfaces, pre- and post-conditions and quality requirements by constraints, data structure specifications by class diagrams and process choreography specified in activity diagrams assigned to service implementation within classes. The activity diagrams were allowed to only use call operations and control logic. The approach was, however, tedious and error prone and the resultant models were seldom in a state which allowed for model transformation, code and test generation and solid documentation generation. Furthermore, additional semantic relationships, not all of which could be represented in diagrams, needed to be added. Full testable service contract specification and complete process specification facilitating full test and code generation proved particularly challenging.

Another modeling language commonly used, particularly in the context of service oriented process specification, is the \emph{Business Process Modeling Notation}. However, it suffers from being too technical\cite{fouad_embedding_2010} and needs to be supplemented by UML or other languages to allow for full data structure and service contract specification. Many UML tool vendors make BPMN available as a UML profile, enabling the integration of BPMN process specification with other UML modeling elements and diagrams. This approach, however, effectively increases the already substantial size of the UML even further, resulting in further learning costs and inconsistency risks.

In order to simplify the URDAD model encoding and to improve the quality of URDAD models an URDAD metamodel and text grammar have been sepcified (a separate paper entitles \emph{`A Domain-Specific Language for URDAD Based Requirements Elicitation'} is currently being reviewed. Here we discuss aspects of the metamodel and text grammar from a quality perspective. In particular, many of the model qualities are either enforced or supported through the metamodel including
\begin{itemize}
 \item decoupling of services across levels of granularity through services contracts,
 \item a range of traceability links including enforced usage of both satisfiability links and \emph{`requiredBy'} links between requirements and stakeholders,
 \item the assignment of services into responsibility domains, and 
 \item the support of parametrized, reusable constraints for the specification of pre- and post-conditions as well as guard conditions for conditional functional requirements and alternative flows within processes.
\end{itemize}

The URDAD metamodel is modularized with a \emph{core module} introducing base concepts like that of a model, an element, a responsibility domain and an expression. A {\emph data module} introduces standard modeling constructs for object-oriented data specification. More interesting is the constraints module which introduces modelling constructs through which one can specify reusable parametrized constraints which are applicable to a services oriented approach.

Constraints are required for the specification of functional requirements (pre- and post-conditions), data structure constraints and decision points in processes. Note that the same constraint can be a pre-condition for some services, a post-condition for other services and a guard condition within a conditional flow of a process. The Object-Constraint Language (OCL)\cite{_object_2010}  has become the de-facto standard for specifying constraints across object graphs. However, in a services-oriented approach and in the context of reusable, parametrised constraints the OCL alone is not expressive enough. This is due to two reasons.

Firstly, in a service-oriented context the actual environmental state is only accessible through services which report on the environment and not by traversing an object graph. The specification of constraints must thus relate to the specification of services through which information is sourced from the environment together with a set of data structure constraints on the obtained information. OCL can be used for the latter, but is insufficient for the complete specification of a constraint.

Secondly, the definition of reusable constraints requires support for binding parameters. \emph{For example}, assume a constraint that some `Person' must be `registered'. Such a constraint can contribute to pre- or post-conditions of services. Hence, to be able to do so, the person identifier would have to be passed as a parameter to the constraint entity. 

The following listing shows an simple example of a parametrized constraint, \emph{studentEnrolledForPresentation}, which demonstrated that a state constraint is assembled from a process that extracts information from the environment and a set of data constraints applied to the obtained information.
\lstset{language=urdad,caption=Specifying a state constraint in the URDAD text grammar.,label=constraintTextSyntax}
\begin{lstlisting}[numbers=left,escapechar=|]
StateConstraint studentEnrolledForPresentation receiving Variable enrollForPresentationRequest ofType EnrollForPresentationRequest
{
  stateAssessmentProcess doSequential
  {
    create Variable getEnrollmentsRequest ofType GetEnrollmentsRequest
    set Query OCL:"getEnrollmentsRequest.presentationIdentifier" equalTo Query OCL:"enrollForPresentationRequest.presentationIdentifier"
    requestService getEnrollments with getEnrollmentsRequest yielding Variable getEnrollmentsResult ofType GetEnrollmentsResult
  }
  Constraint OCL:"getEnrollmentsResult.enrollments.includes (enrollForPresentationRequest.personIdentifier)"
}
\end{lstlisting}

Figure \ref{fig:constraintModule}, shows the elements provided with the constraints module of the URDAD metamodel as well as their relationship with other modeling constructs.
\begin{figure}[Htbp]
  \centering
  \includegraphics{constraint}
  \caption{The modeling constructs available in URDAD facilitating the specification of constraints}
  \label{fig:constraintModule}
\end{figure}

The \emph{contract module} of the URDAD metamodel has the modeling constructs used to specify services contracts including \emph{requiredBy} links between requirements and stakeholders. The specification of a services contract within the URDAD DSL text grammar is illustrated with the following listing.
\lstset{language=urdad,caption=Specifying a service contract in the textual URDAD DSL syntax.,label=contractTextSyntax}
\begin{lstlisting}[numbers=left,escapechar=|]
ServiceContract enrollForPresentation
{
   FunctionalRequirements receiving Variable enrollForPresentationRequest ofType EnrollForPresentationRequest
   {
      PreCondition enrollmentPrerequisitesMet requiredBy (TrainingRegulator Student) raises EnrollmentPrerequisitesNotSatisfiedException checks constraint enrollmentPrerequisitesForPresentationMet with ValueOf enrollForPresentationRequest
      PostCondition enrollmentProcessPerformed requiredBy (Student Client TrainingRegulator) ensures constraint studentEnrolledForPresentation          with ValueOf studentEnrolledRequest constructedUsing doSequential
         {
            create Variable studentEnrolledRequest ofType StudentEnrolledRequest
            set Query OCL:"studentEnrolledRequest.personIdentifier" equalTo Query OCL:"enrollForPresentationRequest.personIdentifier"                            
            set Query OCL:"studentEnrolledRequest.presentationIdentifier" equalTo Query OCL:"enrollForPresentationRequest.presentationIdentifier"                            
         }  
      PostCondition invoiceIssued ...
    }            
    Request DataStructure EnrollForPresentationRequest 
    {
       has identification presentationIdentifier identifying Presentation
       has identification studentIdentifier identifying Person
       has identification clientIdentifier identifying LegalEntity         
    }
    Result DataStructure EnrollForPresentationResult 
    {
       has component proofOfEnrollment ofType ProofOfEnrollment
       has component invoice ofType Invoice
       has component studyGuide ofType StudyGuide
    }
}
\end{lstlisting}

The contract specification includes pre- and post-conditions, quality requirements and the data structure specifications for the request and result objects of the service. Figure \ref{fig:contractModule} shows the elements of the contract modules of the URDAD metamodel.

\begin{figure}[Htbp]
  \centering
  \includegraphics{contract}
  \caption{The modeling constructs available within URDAD to specify services contracts.}
  \label{fig:contractModule}
\end{figure}

The \emph{process module} depicted \ref{fig:processModule} in figure contains the specification of the services used to address the functional requirements and the process specification for the service. The metamodel enforces realization links between a lower level abstract service (service contract) which is used in a process and the functional requirement for which the lower level service is used. These links are manifested as \verb+usedToAddress+ links representing satsifaction links of \cite{ramesh_toward_2001}). 

The service process is specified as a choreography across the lower level services used to address the functional requirements and supports control constructs, process state management, the handling exceptions raised by lower level services, activities for raising an exception associated with a pre-condition of the service, and the return of a result object for the service. 

\begin{figure}[Htbp]
  \centering
  \includegraphics{process}
  \caption{The modeling constructs available for specifying services and processes in URDAD}
  \label{fig:processModule}
\end{figure}

The following listing illustrates how the specification of a service within the URDAD DSL text grammar.
\lstset{language=urdad,caption=Specifying a service in the textual URDAD DSL syntax.,label=serviceTextSyntax}
\begin{lstlisting}[numbers=left,escapechar=|]
Service enrollForPresentationImpl realizes enrollForPresentation receiving Variable enrollForPresentationRequest ofType EnrollForPresentationRequest
{
  use checkStudentSatisfiesEnrollmentPrerequisites toAddress (enrollmentPrerequisitesMet)
  use issueInvoice toAddress (financialPrerequisitesSatisfied invoiceIssued) 
  use performEnrollment toAddress (invoiceIssued)
   
  Process doSequential
  {
    create Variable checkStudentSatisfiesEnrollmentPrerequisitesRequest ofType CheckStudentSatisfiesEnrollmentPrerequisitesRequest               
    set Query OCL:"enrollForPresentationRequest.studentIdentifier" equalTo Query OCL:"checkEnrollmentPrerequisitesRequest.studentIdentifier"
    set Query OCL:"enrollForPresentationRequest.presentationIdentifier" equalTo Query OCL:"checkEnrollmentPrerequisitesRequest.presentationIdentifier"
                     
    requestService checkStudentSatisfiesEnrollmentPrerequisites with checkStudentSatisfiesEnrollmentPrerequisitesRequest yielding Variable checkStudentSatisfiesEnrollmentPrerequisitesResult ofType CheckStudentSatisfiesEnrollmentPrerequisitesResult
    choice
    {
      if Constraint enrollmentMeetsPrerequisitesMet OCL:"checkStudentSatisfiesEnrollmentPrerequisitesResult.enrollmentPrerequisitesMet = true"
        doSequential
        {
          ...
          requestService issueInvoice with issueInvoiceRequest yielding Variable issueInvoiceResult ofType IssueInvoiceResult
          {
            on FinancialPrerequisitesNotSatisfiedException raiseException FinancialPrerequisitesNotSatisfiedException
          }
	      ...
          requestService performEnrollment with enrollRequest yielding Variable performEnrollmentResult ofType PerformEnrollmentResult
          
          create Variable enrollForPresentationResult ofType EnrollForPresentationResult
          set Query OCL:"issueInvoiceResult.invoice" equalTo Query OCL:"enrollForPresentationResult.invoice"
          ...                       
          returnResult  enrollForPresentationResult
        }
      else raiseException EnrollmentPrerequisitesNotSatisfiedException
    }
  }
}                 
\end{lstlisting}


%---------------------------------------------------------------

\subsection{URDAD views}
\label{sec:urdadViews}

supplied via text encoding and later via graphical syntax.

%---------------------------------------------------------------

\subsection{Example: Designing URDAD with URDAD}
\label{sec:urdadExample}
by using it to design a service oriented analysis and design methodology. If the process is internally consistent it needs to generate itself. 


%---------------------------------------------------------------

\todo{Fritz: mention adapters to existing legacy}

\todo{Fritz: Add that one of the core challanges is to have requirements specialists make the paradigm shift to specify requirements in terms of services contracts for reusable services\cite{haines_impact_2007}. In practice we have found that URDAD assists significantly on that front.}


\section{Model quality}

Emphasize that model a functional model, bot addressing non-functional (quality requirements) of system

\begin{itemize}
  \item \emph{testability} measured as fraction of non-testable requirements.
  \item \emph{traceability} measured as fraction of requirements which are not related to the stake holder who requires them, fraction of activities which are not traceable to the functional requirement they aim to address, 
  \item \emph{complexity} based on number and complexity of services with complexity of service necessarily related to Mcabe (cyclomatic) complexity - not sure that URDAD addresses this.
  \item \emph{cohesion} - will be interesting to try and find a measure for cohesion in the service oriented world.
  \item \emph{reuse} - number of
  \item \emph{coupling} - URDAD enforces decoupling across levels of granularity in both, the process and the metamodel.
\end{itemize}


\section{Process quality}



\cite{berard_what_1995}

Stake holders:
\begin{itemize}
  \item Project management (measurability, repeatability, estimatability)
  \item Requirements specialists (ease of use, simple process, defined process activities, defined inputs and outputs, tool support)
  \item Business (low cost, trainable, grow in islands)
\end{itemize}


\begin{itemize}
  \item Process measurability
  \item Repeatability
  \item Defined inputs and outputs
  \item Clearly specified tasks with defined activities
  \item Process consistency (URDAD generates itself)
\end{itemize}

Can apply process to a sub-world, decouples from higher and lower level granularities via contracts

Could introduce more abstract qualities and things the process must have to realize these, e.g. \emph{usability} affected by many of these

CMM requires process definition


\subsection{Internal process consistency}

Here show that if you use URDAD to design an analysis and design methodlogy, you will get URDAD. Feed additional concepts into URDAD.

process assists in improving quality of requirements by forcing certain questions


\section{Related work \label{sec:relatedWork}}

NOTE: See the Zotero reference collection for this paper. Please add any additional references to that collection.

\begin{itemize}
 \item \cite{iacob_model-driven_2008} discuss the mapping of business rules specified using OMG's {\em Semantics for Business Vocabulary and Rules} (SBVR) to service specification and orchestration, mapping onto BPEL process specifications via MDA tools.

  \item \cite{asnina_computation_2010} stress the need for performing the modeling in the problem domain as well as the need to accumulate the requirements within a single model. They effectively also group services into responsibility domains represented by their notion of feature sets, decompose functional requirements across levels of granularity, orchestrate higher level processes across lower level services and define the notion of functional with cause and effect which can be viewed as a way of specifying a services contract. In addition they provide a {\em topological functional model} (TFM) for mapping technology neutral service requirements onto available concrete services pool. The TFM is independent of the modeling technique and can be applied to an URDAD model






\end{itemize}


\section{Conclusions and outlook}

In this paper we identified the stakeholders in the analysis and design methodology and the resultant requirements and technology neutral process design model and their quality requirements. We related these quality requirements to quality-drivers and measures which can be applied within a services-oriented approach. We then identified that subset of quality-drivers which has been embedded within the URDAD process and pointed out how some of these are enforced by the URDAD metamodel.

One of the practical benefits of the URDAD methodology is that it assists requirements engineers to make the paradigm shift\cite{haines_impact_2007} to defining stand-alone services contracts and to assemble processes from abstract, reusable, stateless services with the concrete service providers either selected during the implementation mapping phase or alternatively provided by the execution environment through mechanisms like real-time service provider selection and dependency injection.

Future work includes the specification of a graphical grammar making the domain specific language for URDAD accessible to requirements engineers like business analysts and the specification and development of quality assessment tools which can be used to either report quality measures or provide real time quality guidelines to modellers.

\bibliographystyle{plain}  %%abbrv
\bibliography{../../bibliography}

\end{document}
