\documentclass{IOS-Book-Article}

\normalfont
\usepackage[T1]{fontenc}

\usepackage[colorinlistoftodos, textwidth=4cm, shadow]{todonotes}

\usepackage{times}
\usepackage{graphicx}
\usepackage{epsfig}
\usepackage{rotating}
\usepackage{multirow}
\usepackage{amssymb}
\usepackage{pifont} 

\newcommand{\tick}{\ding{52}}

\usepackage{color}
\usepackage{listings}
\lstdefinelanguage{urdad}
{
keywords=
  {Model,ResponsibilityDomain,Query,Constraint,QualityConstraint,FunctionalRequirements,receiving,yielding,
  StateConstraint,stateAssessmentProcess,InverseConstraint,inverseOf,AndConstraint,AND,OrConstraint,OR,
  XorConstraint,XOR,from,to,many,BasicDataType,DataStructure,is,abstract,has,Variable,ofType,Constant,
  ValueOf,Exception,attribute,identification,identifying,association,linking,aggregate,component,
  QualityRequirement,requiredBy,constraint,with,constructedUsing,ResultConstraint,PreCondition,
  raises,checks,PostCondition,ensures,use,toAddress,if,ServiceContract,undoneUsing,Request,Result,
  Service,realizes,doSequential,choice,else,doConcurrent,blocking,Concurrency,wait,until,create,set,
  equalTo,add,remove,requestService,on,raiseException,returnResult,while,do,forAll,Note},%
sensitive=true,%
alsoletter={\$},%
comment=[l]{\#},%
string=[b]",%
string=[b]'%
}

%\definecolor{OliveGreen}{cmyk}{0.64,0,0.95,0.40}
%\definecolor{CadetBlue}{cmyk}{0.62,0.57,0.23,0}
\definecolor{lightgray}{gray}{0.9}
\lstset{
language=urdad,  
basicstyle=\ttfamily\tiny,
keywordstyle=\itshape\color{blue},
%keywordstyle=\color{blue},        % Keywords font ('*' = uppercase)
commentstyle=\color{gray},           
numbers=left,                           % Line nums position
numberstyle=\tiny,                      % Line-numbers fonts
stepnumber=1,                           % Step between two line-numbers
numbersep=5pt,                          % How far are line-numbers from code
backgroundcolor=\color{lightgray}, % Choose background color
frame=none,                             % A frame around the code
tabsize=2,                              % Default tab size
captionpos=b,                           % Caption-position = bottom
breaklines=true,                        % Automatic line breaking?
breakatwhitespace=false,                % Automatic breaks only at whitespace?
showspaces=false,                       % Dont make spaces visible
showtabs=false,                         % Dont make tabls visible
columns=flexible,                       % Column format
%morekeywords={__global__, __device__},  % CUDA specific keywords
}

%
\begin{document}
\begin{frontmatter} 

\title{URDAD as a Quality-Driven Analysis and Design Process}
\thanks{We thank the national research foundation (NRF) of South Africa for financial support.}
\runningtitle{URDAD as Quality-Driven Methodology}
%\subtitle{Subtitle}

\author[A]{\fnms{Fritz} \snm{Solms}}
,
\author[A]{\fnms{Stefan} \snm{Gruner}}
and
\author[B]{\fnms{Cuen} \snm{Edwards}}

\runningauthor{F. Solms et al.}
\address[A]{Department of Computer Science, University of Pretoria, South Africa}
\address[A]{Solms Training and Consulting, 113 Barry Hertzog Ave, Johannesburg, South Africa}

\begin{abstract}
Use-Case Responsibility-Driven Analysis and Design (URDAD) is a service-oriented software analysis and design methodology. It is used by requirements engineers to develop technology-neutral, semi-formal platform-indepen\-dent models (PIM) within the OMG's MDA. In the past, URDAD models were denoted in UML. However, that was tedious and error-prone. The resulting models were often of rather poor quality. In this paper we introduce and discuss a new Domain-Specific Language (DSL) for URDAD. Its meta model is consistent and satisfiable. We show that URDAD DSL specifications are simpler and allow for more complete service contract specifications than their corresponding UML expressions. They also enable traceability and test case generation.
\end{abstract}

\begin{keyword}
Services-oriented design methodology\sep Model-driven development\sep Design quality\sep Model quality
\end{keyword}
\end{frontmatter}

\thispagestyle{empty}
\pagestyle{empty}

\maketitle

\section{Introduction}
Insufficiency in requirements engineering is still regarded as a root cause of poor software quality. This is due to various factors, both human and technological, including vague specification languages with only informally defined semantics. Insufficient language support for \emph{layered} specifications (i.e., decompositional system descriptions at different levels of granularity), leads software developers to making wrong presumptions about lower level requirements \cite{espana_evaluating_2009}. Tool support for the validation of requirements specifications, or for the automatic extraction of test cases from them, is also still weak \cite{bashardoust-tajali_extracting_2008}.

Model-Driven Engineering (MDE) \cite{schmidt_model_2006} aims at solving some of those problems by using modelling languages with well defined semantics, by requiring primary models to be domain models, not technical models \cite{asnina_computation_2010} and by providing tool support for MDE processes. Consequently, technology-neutral domain models are developed by requirements specialists, not by technical experts \cite{asnina_computation_2010}.

\emph{URDAD}, the Use-Case Responsibility-Driven Analysis and Design methodology \cite{fritz_solms_technology_2007} supports MDE in a service-oriented way \cite{solms_urdad_2010}. It is used by requirements specialists to develop and validate technology-neutral requirements models. URDAD models are thus Platform-Independent Models (PIM) in the Model-Driven Architecture (MDA) context \cite{solms_urdad_2010}. For each level of granularity the method leads to testable service contracts and for non-leaf services a technology neutral process realizing the service contract through the use of lower level services. Higher-level services are thus a functional composition of lower-level services, similar to the classical DFD technique \cite{demarco_tom_structured_1978}, with the levels of granularity decoupled through service contracts.

Requirements engineers have traditionally used the Unified Model Language (UML) to encode URDAD models. UML was a reasonable choice for this purpose because of its tool-supported use in the software industry. However, UML is an object oriented modeling language which is not conceptually aligned with a service oriented approach where stateless services are always assembled form lower level stateless services. On the one side it allows for a higher level services are assembled does not support many of the concepts required by the URDAD methodology explicitly and allows for a wide variety of model structures, most of which would not comply to the services-oriented structure of an URDAD model and on the other side it does not explicitly support many of the concepts required by URDAD. For example, the concept of a responsibility domain, a stakeholder, are not explicitly supported. Indeed, a specific UML \emph{profile} could be used to restrict the use of UML according to URDAD's intentions and at the same time introduce explicitly concepts required by URDAD. In practice, however, such a UML profile would contain an excessive number of metamodel constraints ensuring that a UML model complies structurally to a service-oriented URDAD model.

In this paper we present a new domain-specific language (DSL) for the domain of technology-neutral service-orien\-ted requirements modelling. Our new URDAD DSL is described in terms of a MOF/EMOF meta model. This makes it amenable to MDA tool suites for model transformations, as well as the generation of concrete textual and diagrammatic syntaxes with tool support \cite{gronback_model_2008}. To this end we analyse theoretically the modelling constructs required by URDAD. We elucidate and critically assess the URDAD meta model, and we propose a concrete textual syntax for an URDAD DSL. A Description Logics (DL)-based representation of the URDAD meta model is derived from the MOF/EMOF meta model in order to show its consistency and satisfiability.

Consequently we argue (also w.r.t. related work) that the URDAD DSL has two main advantages over the use of an URDAD UML profile. The language is considerably simpler than the UML and, with appropriate tool support, is expected to simplify the process through which requirements engineers can build high-level, technology-neutral models. Our new DSL enforces the structure required for a valid URDAD model, thereby requiring only a rather small and simple set of meta model constraints at the basis of tool-supported model validation. In addition the URDAD DSL provides better support for specifying service contracts within a service-oriented approach.



\subsection{Quality requirements}

In order to be able to measure quality, we need to identify the quality requirements. Requirements only make sense from the perspective of the stakeholder who requires the requirment. Hence to identify quality requirements we need to first identify the stakeholders who have an interest in the process as well as those who have an interest in the resultant analysis and design model. Once we have identified the stakeholders, we can elicit the quality requirements

We differentiate between quality requirements for the process itself from the quality requirements on the outputs (the analysis and design model).

\cite{berard_what_1995}

Stake holders:
\begin{itemize}
  \item Project management (measurability, repeatability, estimatability)
  \item Requirements specialists (ease of use, simple process, defined process activities, defined inputs and outputs, tool support)
  \item Business (low cost, trainable, grow in islands)
\end{itemize}


\begin{itemize}
  \item Process measurability
  \item Repeatability
  \item Defined inputs and outputs
  \item Clearly specified tasks with defined activities
  \item Process consistency (URDAD generates itself)
\end{itemize}

Can apply process to a sub-world, decouples from higher and lower level granularities via contracts

Could introduce more abstract qualities and things the process must have to realize these, e.g. \emph{usability} affected by many of these

CMM requires process definition


\subsection{Internal process consistency}

Here show that if you use URDAD to design an analysis and design methodlogy, you will get URDAD. Feed additional concepts into URDAD.

Here identify the stakeholders in both, the process and the outputs of the process and their quality requirements.

\section{Quality drivers}
\label{sec:qualityDrivers}

A quality drivers is defined as an activity which improves one or more process or model quality criteria\cite{petersen_software_1989}. The relience on quality-drivers forms the basis for the concept of a \emph{`quality-driven'} process. In this section we look at each model and process quality criterion and discuss activities (quality drivers) which can be used to improve the quality criterion. The list of quality-drivers presented here is not intended to be exhaustive but includesrepresents a collection of quality-drivers which are widely known and used. In addition to identifying quality-drivers, we also note that quality metrics can be applied to the resultant model in order to assess model quality. This section discusses some commonly used quality metrics for the identified quality criteria suitable for a service-oriented approach. For a more general list of model quality metrics the reader should consult \cite{mohagheghi_existing_2009}.

\subsection{Model quality drivers}

\emph{Semantic Model Quality} relates to semantic accuracy and completeness. It is influenced by the quality of the modeling language and hence by the semantic quality of the metamodel of the modeling language. The latter is determined by its semantic completeness, consistency and complexity\cite{buder_effects_2010}. Semantic completeness refers to the power of the formal language to express all propositions. More precisesly, a formal system is said to be semantically complete if and only if every theorem of the system is provable in the system.  A core semantic quality-driver is thus the specification of a metamodel for the analysis and design model which itself is shown to be semantically complete and consistent.

\emph{Syntactic Quality} elates to correct language usage\cite{lange_christiaan_assessing_2007}, i.e.\ that the grammatical statements made comply to the abstract syntax as specified via the metamodel and its constraints or an ontology and its associated rules. A commonly used syntactic quality driver is the definition of concrete text and/or graphical grammars and a transformation between the concrete and abstract syntax. Models which are defined through and verified against the concrete syntax can be shown to comply to the abstract syntax of the associated metamodel. Editors generated from the concrete syntax definition generally validate syntactic correctness. In this paper we confine the assessment of syntactic quality to the existence of a formal modeling language and concrete grammar for model specification. We will thus assume that model instances will adhere to the grammar of the chosen modeling language as this can easily be verified.

\emph{Simplicity} is the inverse of complexity. The complexity of the modeling language is usually assessed by measuring the complexity of the metamodel for that language\cite{mohagheghi_evaluating_2007}. A lot of work has been done on model complexity itself. Common approaches include using information entropy measures\cite{abrahamsson_extreme_2004}, language-theoretic approaches\cite{podgorelec_estimating_2007} and function complexity assessment based on the McCabe complexity measure \cite{mccabe_complexity_1976}. A service-oriented approach already enforces certain drivers for simplicity. In particular it enforces the assembling of functions from independent, stateless services, the enforced decoupling through services contracts, the improved reuse through the implied service discovery, as well as an implied adapter layer facilitating reuse across technologies and interfaces mismatches. A metamodel confining the modeling constructs and relationships between these, as well as a convenient grammar, reduce complexity. Another core driver for simplicity is the decomposition across levels of granularity enabling the understanding and processing of one level of granularity before having to concern oneself with the details of the next lower level of granularity. The availability of a services contract enables one to look at a service from a user/service consumer perspective before understanding the service function. Finally, traceability links and in particular enforced satisfaction links (i.e.\ that a functions may only use services which address functional requirements of the service) are drivers for simplicity.

\emph{Completeness} of requirements is difficult to assess, though the discovery of certain stakeholders for which no functional or non-functional requirements exist is an indication of missing requirements. Design completeness is easier to assess as there is a global point of reference - the degree to which the requirements are fulfilled. Core quality-drivers for completeness include enforced satisfaction links, the enforcement of testable pre- and post-conditions, as well as having a metamodel which enforces certain content, either structurally, or through metamodel constraints. In a service-oriented approach, a completeness metric can be specified as a function of 1) the percentage of functional requirements not addressed in the designed functions (identified through missing satisfaction links)\cite{shim_design_2008}, 2) the fraction of pre- and post-conditions for which the test process has not been specified and 3) the percentage of request and result classes for which the data structures have not been specified. At the implementation level completeness can be assessed as the fraction of services for which service implementations are not yet available.

\emph{Consistency} is an important model property. A lot of emphasis has been placed on model consistency for UML models. This is so because UML models are often inconsistent due to the multi-diagram approach and the complexity of the modeling language itself\cite{lange_empirical_2004}. A commonly quality-driver for model consistency is to use a much smaller (less general) language with more formally defined semantics which enforces the consistency through both, its structure and a set of metamodel constraints.

\emph{Cohesion} in a service-oriented approach has been extensively studied by Mikhael Perepletchikov et al.\ \cite{perepletchikov_cohesion_2007,perepletchikov_impact_2010}. Their approach is closely related to that of unity criteria \cite{gonzalez_unity_2009} identifying and quantifying interfacing cohesion, usage cohesion and implementation cohesion. These cohesion measures can be applied directly to an URDAD model. Quality-drivers for cohesion include enforcing the single responsibility principle as well as localizing all controll and decision logic for a service in a controller service. 

\emph{Decoupling} is enforced in a service-oriented approach by requiring that services are only consumed via services contracts. This decouples services across levels of granularity. Quality-drivers for decoupling include a metamodel which enforces contracts based decoupling through metamodel structure and localization of function logic within a controller. The latter ensures that lower level services remain decoupled, i.e.\ that they do not call each other. In a service-oriented architecture, the number of services which are directly coupled to concrete lower level services can be used as a measure of coupling\cite{shim_design_2008}.

\emph{Modifiability} refers to the efficiency with which model changes can be applied, i.e.\ it is related to the inverse of the cost required to make model changes. Modifiability is difficult to quantitatively measure. It is supported by other model qualities like \emph{simplicity},  \emph{decoupling} (modifiability through pluggability), and \emph{cohesion} (localized maintenance). A further quality-driver in the form of localizing function logic within a controller service, thereby projecting out additional levels of granularity. Shim et al.\ \cite{shim_design_2008} define a quality metric for modifiability (flexibility) of service-oriented designs as a weighted sum of coupling, service granularity, and parameter granularity. Since increased complexity and coupling reduce modifiability, the quality drivers for simplicity and decoupling are also quality drivers for modifiability.

\emph{Reusability} provides a measure of the ability and likelihood that a service can be reused. Khoshkbarforoushha et al.\  \cite{khoshkbarforoushha_metric_2010,choi_quality_2008,feuerlicht_determinants_2007}
\cite{khoshkbarforoushha_metric_2010} point out that service reusability is often caused by contract and requirements mismatch. The former can be addressed via adapters. Core quality-drivers for reusability include \emph{decoupling} via services contracts with the latter also driving discoverability and consumability, \emph{levels of granularity} via process localization within a reusable controller service, and \emph{cohesion} through enforcing the single responsibility principle. Indeed, \cite{shim_design_2008} defines a simple quantitative reusability measure for service-oriented systems as a weighted sum of coupling, cohesion, granularity, and consumability.

\emph{Traceability} is needed for design validation and estimation. Validation includes assessing sufficiency and necessity. Ramesh and Jarke \cite{ramesh_toward_2001} identify four types of traceability links in models including \emph{satisfaction} links used to assess whether requirements are satisfied, \emph{evolutionary} links to trace along the evolution of an artifact over time, \emph{rationale} links which typically link requirements to higher level business goals, and \emph{dependency} links enabling one to identify dependencies of model elements. Quality-drivers include thus the availability of these traceability links in the modeling language and their enforced usage through the modeling process.

\section{URDAD}

\begin{itemize}
  \item Modeling in problem domain
  \item sevices-oriented analysis and design methodology
  \item contract focused
  \item Responsibility driven
  \item semi-formal (agile/formal)
  \item levels of granularity
  \item
\end{itemize}


\section{Related work \label{sec:relatedWork}}

The URDAD methodology provides a services-oriented methodology for generating a semi-formal analysis and design model representing MDA's PIM and supporting test and implementation generation. \cite{iacob_model-driven_2008} discuss an alternative approach. Business Rules are specified using OMG's {\em Semantics for Business Vocabulary and Rules} (SBVR) to service specification and orchestration and BPEL process specifications are generated using MDA tools. Processes are assembled from services which are related to business rules. This is similar to our satisfiability links specifying the services used to realize the different functional requirements.

The URDAD DSL allows for the specification of textual and graphical grammars through which the URDAD model is populated. An alternative approach is to define a separate metamodel for the use case narrative and to transform the narrative model requirements model \cite{hoffmann_towards_2009,osis_transforming_2010}. This approach introduces the complexities of having to transform from the narrative to the UML model and requires extensive consistency checks between the narrative and the UML models.

\cite{asnina_computation_2010} stress the need of modelling in the problem domain as well the benefits of accumulating requirements within a single model. Services are grouped into feature sets which are related to responsibility domains. Functional requirements are decomposed across levels of granularity and higher level processes are orchestrated across lower level services. They define the notion of functionals with cause and effect which can be related to the concept of a services contract. In addition they provide a {\em topological functional model} (TFM) for mapping technology neutral service requirements onto available concrete services pool. The TFM is independent of the modelling technique and can be applied to an URDAD model. 

In the {\em Requirements Driven Design Automation} methodology (RDDA) \cite{cardei_model_2008} one encodes requirements specifications in SYSML diagrams. The SYSML model is enriched with semantic descriptions after which the model is transformed to the {\em One Pass to Production} (OPP) design language, the ODL. ODL is an OWL based ontology from which the requirements are validated for consistency and completeness. The approach is, however, structure focused with little emphasis on services contracts and recursive orchestration of higher level services from lower level services.


\section{Conclusions and outlook}

In this paper we identified the stakeholders in the analysis and design methodology and the resultant requirements and technology neutral process design model and their quality requirements. We related these quality requirements to quality drivers and measures which can be applied within a services-oriented approach. We then identified that subset of quality drivers which has been embedded within the URDAD process and pointed out how some of these are enforced by the URDAD metamodel.

One of the practical benefits of the URDAD methodology is that it assists requirements engineers to make the paradigm shift\cite{haines_impact_2007} to defining stand-alone services contracts and to assemble processess from abstract, reusable, stateless services with the concrete service providers either selected during the implementation mapping phase or alternatively provided by the execution environment through mechanisms like real-time service provider selection and dependency injection.

Future work includes the specification of a graphical grammar making the domain specific language for URDAD accessible to requirements engineers like business analysts and the specification and development of quality assessment tools which can be used to either report quality measures or provide real time quality guidelines to modelers.

\bibliographystyle{plain}  %%abbrv
\bibliography{../../bibliography}

\end{document}
