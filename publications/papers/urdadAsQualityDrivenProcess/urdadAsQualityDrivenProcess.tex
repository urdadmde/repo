\documentclass{IOS-Book-Article}

\usepackage{times}
\normalfont
\usepackage[T1]{fontenc}
%\usepackage[mtplusscr,mtbold]{mathtime}

\usepackage{graphicx}
\usepackage{epsfig}
\usepackage{listings}
\usepackage{color}

\lstdefinelanguage{urdad}
{
keywords=
  {Model,ResponsibilityDomain,Query,Constraint,QualityConstraint,FunctionalRequirements,receiving,yielding,
  StateConstraint,stateAssessmentProcess,InverseConstraint,inverseOf,AndConstraint,AND,OrConstraint,OR,
  XorConstraint,XOR,from,to,many,BasicDataType,DataStructure,is,abstract,has,Variable,ofType,Constant,
  ValueOf,Exception,attribute,identification,identifying,association,linking,aggregate,component,
  QualityRequirement,requiredBy,constraint,with,constructedUsing,ResultConstraint,PreCondition,
  raises,checks,PostCondition,ensures,use,toAddress,if,ServiceContract,undoneUsing,Request,Result,
  Service,realizes,doSequential,choice,else,doConcurrent,blocking,Concurrency,wait,until,create,set,
  equalTo,add,remove,requestService,on,raiseException,returnResult,while,do,forAll,Note},%
sensitive=true,%
alsoletter={\$},%
comment=[l]{\#},%
string=[b]",%
string=[b]'%
}

%\definecolor{OliveGreen}{cmyk}{0.64,0,0.95,0.40}
%\definecolor{CadetBlue}{cmyk}{0.62,0.57,0.23,0}
\definecolor{lightgray}{gray}{0.9}
\lstset{
language=urdad,  
basicstyle=\ttfamily\small,
keywordstyle=\itshape\color{blue},
%keywordstyle=\color{blue},        % Keywords font ('*' = uppercase)
commentstyle=\color{gray},           
numbers=left,                           % Line nums position
numberstyle=\tiny,                      % Line-numbers fonts
stepnumber=1,                           % Step between two line-numbers
numbersep=5pt,                          % How far are line-numbers from code
backgroundcolor=\color{lightgray}, % Choose background color
frame=none,                             % A frame around the code
tabsize=2,                              % Default tab size
captionpos=b,                           % Caption-position = bottom
breaklines=true,                        % Automatic line breaking?
breakatwhitespace=false,                % Automatic breaks only at whitespace?
showspaces=false,                       % Dont make spaces visible
showtabs=false,                         % Dont make tabls visible
columns=flexible,                       % Column format
%morekeywords={__global__, __device__},  % CUDA specific keywords
}

%
\begin{document}
\begin{frontmatter} 

\title{URDAD as a Quality-Driven Analysis and Design Methodology}
\thanks{We thank the national research foundation (NRF) of South Africa for financial support.}
\runningtitle{URDAD as Quality Driven Methodology}
%\subtitle{Subtitle}

\author[A]{\fnms{Fritz} \snm{Solms}}
,
\author[A]{\fnms{Stefan} \snm{Gruner}}
and
\author[A]{\fnms{Alexander} \snm{Paar}}

\runningauthor{F. Solms et al.}
\address[A]{Department of Computer Science, University of Pretoria, South Africa}

\begin{abstract}
Use-Case Responsibility-Driven Analysis and Design (URDAD) is a service-oriented software analysis and design methodology. It is used by requirements engineers to develop technology-neutral, semi-formal platform-indepen\-dent models (PIM) within the OMG's MDA. In the past, URDAD models were denoted in UML. However, that was tedious and error-prone. The resulting models were often of rather poor quality. In this paper we introduce and discuss a new Domain-Specific Language (DSL) for URDAD. Its meta model is consistent and satisfiable. We show that URDAD DSL specifications are simpler and allow for more complete service contract specifications than their corresponding UML expressions. They also enable traceability and test case generation.
\end{abstract}

\begin{keyword}
Services-oriented design methodology\sep Model-driven development\sep Design quality\sep Model quality
\end{keyword}
\end{frontmatter}

\thispagestyle{empty}
\pagestyle{empty}

\maketitle

\section{Outstanding issues}
\begin{itemize}
 \item Stefan: Meaning of quality how defined for paper 
 \item Stefan: Product vs process quality 
 \item Stefan: Causal connection btw product and process quality? 
 \item Stefan: Discuss Quality Categories. 
 \item Fritz: Look at unity models as in Espana et al and see progress report for responsibility localization and general short-comings in use case based approach regarding this. 
 \item Fritz: INSERT REF - Quality in Model-Driven Engineering argues model quality determined by 1,2) modeling language and tools used to define model, 3) modeling process itself, 4) techniques used to assure quality 5)relative experience of individuals building model. 
\end{itemize}

\section{Introduction}
Insufficiency in requirements engineering is still regarded as a root cause of poor software quality. This is due to various factors, both human and technological, including vague specification languages with only informally defined semantics. Insufficient language support for \emph{layered} specifications (i.e., decompositional system descriptions at different levels of granularity), leads software developers to making wrong presumptions about lower level requirements \cite{espana_evaluating_2009}. Tool support for the validation of requirements specifications, or for the automatic extraction of test cases from them, is also still weak \cite{bashardoust-tajali_extracting_2008}.

Model-Driven Engineering (MDE) \cite{schmidt_model_2006} aims at solving some of those problems by using modelling languages with well defined semantics, by requiring primary models to be domain models, not technical models \cite{asnina_computation_2010} and by providing tool support for MDE processes. Consequently, technology-neutral domain models are developed by requirements specialists, not by technical experts \cite{asnina_computation_2010}.

\emph{URDAD}, the Use-Case Responsibility-Driven Analysis and Design methodology \cite{fritz_solms_technology_2007} supports MDE in a service-oriented way \cite{solms_urdad_2010}. It is used by requirements specialists to develop and validate technology-neutral requirements models. URDAD models are thus Platform-Independent Models (PIM) in the Model-Driven Architecture (MDA) context \cite{solms_urdad_2010}. For each level of granularity the method leads to testable service contracts and for non-leaf services a technology neutral process realizing the service contract through the use of lower level services. Higher-level services are thus a functional composition of lower-level services, similar to the classical DFD technique \cite{demarco_tom_structured_1978}, with the levels of granularity decoupled through service contracts.

Requirements engineers have traditionally used the Unified Model Language (UML) to encode URDAD models. UML was a reasonable choice for this purpose because of its tool-supported use in the software industry. However, UML is an object oriented modeling language which is not conceptually aligned with a service oriented approach where stateless services are always assembled form lower level stateless services. On the one side it allows for a higher level services are assembled does not support many of the concepts required by the URDAD methodology explicitly and allows for a wide variety of model structures, most of which would not comply to the services-oriented structure of an URDAD model and on the other side it does not explicitly support many of the concepts required by URDAD. For example, the concept of a responsibility domain, a stakeholder, are not explicitly supported. Indeed, a specific UML \emph{profile} could be used to restrict the use of UML according to URDAD's intentions and at the same time introduce explicitly concepts required by URDAD. In practice, however, such a UML profile would contain an excessive number of metamodel constraints ensuring that a UML model complies structurally to a service-oriented URDAD model.

In this paper we present a new domain-specific language (DSL) for the domain of technology-neutral service-orien\-ted requirements modelling. Our new URDAD DSL is described in terms of a MOF/EMOF meta model. This makes it amenable to MDA tool suites for model transformations, as well as the generation of concrete textual and diagrammatic syntaxes with tool support \cite{gronback_model_2008}. To this end we analyse theoretically the modelling constructs required by URDAD. We elucidate and critically assess the URDAD meta model, and we propose a concrete textual syntax for an URDAD DSL. A Description Logics (DL)-based representation of the URDAD meta model is derived from the MOF/EMOF meta model in order to show its consistency and satisfiability.

Consequently we argue (also w.r.t. related work) that the URDAD DSL has two main advantages over the use of an URDAD UML profile. The language is considerably simpler than the UML and, with appropriate tool support, is expected to simplify the process through which requirements engineers can build high-level, technology-neutral models. Our new DSL enforces the structure required for a valid URDAD model, thereby requiring only a rather small and simple set of meta model constraints at the basis of tool-supported model validation. In addition the URDAD DSL provides better support for specifying service contracts within a service-oriented approach.



%\subsection{Quality requirements}

In order to be able to measure quality, we need to identify the quality requirements. Requirements only make sense from the perspective of the stakeholder who requires the requirment. Hence to identify quality requirements we need to first identify the stakeholders who have an interest in the process as well as those who have an interest in the resultant analysis and design model. Once we have identified the stakeholders, we can elicit the quality requirements

We differentiate between quality requirements for the process itself from the quality requirements on the outputs (the analysis and design model).

\cite{berard_what_1995}

Stake holders:
\begin{itemize}
  \item Project management (measurability, repeatability, estimatability)
  \item Requirements specialists (ease of use, simple process, defined process activities, defined inputs and outputs, tool support)
  \item Business (low cost, trainable, grow in islands)
\end{itemize}


\begin{itemize}
  \item Process measurability
  \item Repeatability
  \item Defined inputs and outputs
  \item Clearly specified tasks with defined activities
  \item Process consistency (URDAD generates itself)
\end{itemize}

Can apply process to a sub-world, decouples from higher and lower level granularities via contracts

Could introduce more abstract qualities and things the process must have to realize these, e.g. \emph{usability} affected by many of these

CMM requires process definition


\subsection{Internal process consistency}

Here show that if you use URDAD to design an analysis and design methodlogy, you will get URDAD. Feed additional concepts into URDAD.

Here identify the stakeholders in both, the process and the outputs of the process and their quality requirements.

%\section{Overview of the URDAD methodology \label{sec:urdadMethodology}}

URDAD as a methodology generating the requirements model

URDAD is an algorithmic, semi-formal methodology for eliciting and capturing service/use case requirements and technology neutral business process designs\cite{solms_urdad_2010}. 
TODO: REWRITE FOLLOWING PART OF PARAGRAPH
 The methodology recursively decomposes cohesive units of functional requirements into lower level functional requirements until the functional requirements are
fully specified in terms of well understood pieces of functionality.

Services are recursively decomposed until the lowest level services which are services sourced from the environment (e.g. operating system, frameworkd, off-the shelf solutions and services sourced from external service providers). These low level provided by the environment are treated as atomic services.

The process is an iterative process, iterating across levels of granularity. At each level of granularity the required functionality is decomposed into cohesive lower level functional units called services. The required logic for assembling the higher level functionality/service from lower level services is specified in a process.

The steps at any level of granularity include
\begin{enumerate}
 \item lower level {functional requirements elicitation}
 
Decompose the higher level functional requirement (service) into lower level functional requirements required by different stake holders.
 \item Specify the services contract for the higher level service including the pre- and post-conditions, data structures for the result and the quality requirements
 \item 
\end{enumerate}

Discuss recursive nature of URDAD, i.e. how URDAD can be used to design itself but refer to quality paper - feed that into quality paper.



%\section{Tools \label{sec:tools}}

%\section{Related work \label{sec:relatedWork}}

The URDAD methodology provides a services-oriented methodology for generating a semi-formal analysis and design model representing MDA's PIM and supporting test and implementation generation. \cite{iacob_model-driven_2008} discuss an alternative approach. Business Rules are specified using OMG's {\em Semantics for Business Vocabulary and Rules} (SBVR) to service specification and orchestration and BPEL process specifications are generated using MDA tools. Processes are assembled from services which are related to business rules. This is similar to our satisfiability links specifying the services used to realize the different functional requirements.

The URDAD DSL allows for the specification of textual and graphical grammars through which the URDAD model is populated. An alternative approach is to define a separate metamodel for the use case narrative and to transform the narrative model requirements model \cite{hoffmann_towards_2009,osis_transforming_2010}. This approach introduces the complexities of having to transform from the narrative to the UML model and requires extensive consistency checks between the narrative and the UML models.

\cite{asnina_computation_2010} stress the need of modelling in the problem domain as well the benefits of accumulating requirements within a single model. Services are grouped into feature sets which are related to responsibility domains. Functional requirements are decomposed across levels of granularity and higher level processes are orchestrated across lower level services. They define the notion of functionals with cause and effect which can be related to the concept of a services contract. In addition they provide a {\em topological functional model} (TFM) for mapping technology neutral service requirements onto available concrete services pool. The TFM is independent of the modelling technique and can be applied to an URDAD model. 

In the {\em Requirements Driven Design Automation} methodology (RDDA) \cite{cardei_model_2008} one encodes requirements specifications in SYSML diagrams. The SYSML model is enriched with semantic descriptions after which the model is transformed to the {\em One Pass to Production} (OPP) design language, the ODL. ODL is an OWL based ontology from which the requirements are validated for consistency and completeness. The approach is, however, structure focused with little emphasis on services contracts and recursive orchestration of higher level services from lower level services.


%\section{Conclusions and outlook}

In this paper we identified the stakeholders in the analysis and design methodology and the resultant requirements and technology neutral process design model and their quality requirements. We related these quality requirements to quality drivers and measures which can be applied within a services-oriented approach. We then identified that subset of quality drivers which has been embedded within the URDAD process and pointed out how some of these are enforced by the URDAD metamodel.

One of the practical benefits of the URDAD methodology is that it assists requirements engineers to make the paradigm shift\cite{haines_impact_2007} to defining stand-alone services contracts and to assemble processess from abstract, reusable, stateless services with the concrete service providers either selected during the implementation mapping phase or alternatively provided by the execution environment through mechanisms like real-time service provider selection and dependency injection.

Future work includes the specification of a graphical grammar making the domain specific language for URDAD accessible to requirements engineers like business analysts and the specification and development of quality assessment tools which can be used to either report quality measures or provide real time quality guidelines to modelers.

%\begin{figure}
%  \centering
%  \includegraphics{myFig}
%  \caption{my caption.}
%  \label{fig:myFig}
%\end{figure}

\bibliographystyle{plain}  %%abbrv
\bibliography{bibliography}

\end{document}
