\section{Related work}
\label{sec:relatedWork}

Philip Mayer, Andreas Schroeder and Nora Koch\cite{mayer_mdd4soa:_2008} define a UML4SOA profile which is mean to support model-driven service orchestration in UML as well as a MDD4SOA approach which has specifies a service orchestration methodology using the UML4SOA profile. They target directly SOA implementation and hence do not aim to be technology and architecture neutral. URDAD, on the other hand, provides a services-oriented analysis and design methodology in the context of MDA with potentially different implementation mappings for layered enterprise architectures, services-oriented or event-driven architectures, legacy systems or even manual processes. Their main concern is to provide a natural and sufficient environment to generate BPEL processes and WSDL contracts and do not directly consider model or process quality requirements and drivers.

Moosavi, Seyyedi and Moghadam\cite{moosavi_method_2009} introduce a method for service-oriented design which aims to address aspects of quality by incorporating accepted `service principles'. They explicitly support layering, defining a data layer, a service access layer containing a set of basic functions which are regarded as leaf services and a collaborative layer which contains the user-centric services orchestrated across lower level services. They explicitly promote a multi-layer approach based on responsibility and reuse based layering strategies in order to promote cohesion and reuse. In particular, they use workflow, composition, interface and logic variation to package units of functionality within reusable and adaptable services. The resultant model is a BPMN model specifying the service orchestration across a services layer which is meant to be optimized for cohesion, reusability and adaptability.

Ionut Cardei, Mihai Fronoage and Ravi Shankar\cite{cardei_model_2008} discuss a methodology within which functional product requirements are specified using an OWL based ontology targeting the mobile software development space. They provide a framework for model verification for completeness and consistency. The functional product requirements are augmented with system requirements specifications specified using SysML.