\section{Quality requirements}
\label{sec:qualityRequirements}


For the purpose of this paper, we take a manufacturing view \cite{garvin_what_1984} to quality, i.e.\ we assess quality from the perspective of conformance to requirements. This section identify the stakeholders who have an interest in the model and the process generating the model and argue their respective quality requirements.

%-------------------------------------------------

\subsection{Model stakeholders and their quality requirements}
\label{sec:modelStakeholdersAndQualityRequirements}


A lot of work has historically been done on defining and measuring code quality (see, for example, \cite{boehm_barry_w._characteristics_1978}). Recently the focus has shifted onto understanding and measuring model quality \cite{lange_managing_2005,lange_improving_2006,shim_design_2008,qi_yu-dong_analysis_2010}. Though most of the work is applied to UML models, the concepts are generally applicable across analysis and design models, i.e.\ they are generally not UML specific. 

Lindland\cite{lindland_understanding_1994} provides a widely used quality categorization into \emph{semantic quality}, which assesses whether a model confers the meaning it was meant to confer, \emph{syntactic quality}, which is a measure of the conformance to the language used to specify the model, and \emph{pragmatic quality}, which is a measure of the extent to which the model can satisfy its intended use. Christiaan Lange  \cite{lange_christiaan_assessing_2007} related model quality back to model purpose and identified a range of pragmatic model quality characteristics including complexity, traceability, modularity, completeness, consistency and communicativeness. Models that exhibit such qualities have been shown to commonly lead to systems with similar quality attributes \cite{podgorelec_estimating_2007}.

In this paper we do not consider communicativeness and social quality as we feel these will be strongly influenced by both, the semantics and the model user interface(s). For example, requirements specialists would typically use a diagrammatic user interface into the model provided by a graphical grammar and appropriate tool support. The client model interface is typically a read-only interface in the form of documentation generated from the model.

The core stakeholders in the model include {\bf requirements engineering} (e.g.\ business analysis), {\bf architecture}, {\bf implementation} (e.g.\ developers), {\bf quality assurance} (e.g.\ testing), {\bf project management} and the {\bf client} (e.g.\ business). In order to address their respective responsibilities, these stakeholders require certain model qualities. All of these stakeholders require \emph{semantic quality} (since they need to extract the correct meaning from the model). The syntactic and pragmatic quality requirements differ across stakeholders, though \emph{simplicity} and \emph{consistency} are qualities which are beneficial to all.

{\bf Requirements engineers} (e.g.\ business analysts) use the model to capture the functional and non-functional requirements of services in the form of services contracts and (business) process designs realizing these services contracts. They also may use the model for model validation and documentation generation, requiring \emph{syntactic quality} to make this feasible.  The \emph{pragmatic qualities} particularly relevant to requirements engineers include \emph{modifiability} (as they would have to modify the model in the context of changing requirements and process improvement), \emph{simplicity} assisting understandability and modifiability, \emph{completeness} for complete requirements and process designs, model \emph{consistency}, \emph{decoupling} of functional requirements from process design, \emph{reuse} to reduce model complexity and increase model consistency, and \emph{traceability} to facilitate model validation and improve \emph{modifiability}. 

For the purpose of this paper, we define architecture as the infrastructure within which processes are deployed and executed. Thus, whilst implementation is concerned with the actual services implementations, the architecture team is concerned with the infrastructure hosting the services. The architecture may span organizational, hardware and software infrastructure. Organizational and systems architects need to be able to design an architecture hosting the services in such a way that the quality requirements (\emph{quality of service}) can be met. They also need to be able to assess whether an existing architecture can host a service as per contract and specify architectural modifications if that is not the case. {\bf Architecture} typically uses the high level services contracts and process specifications. These need to be \emph{complete} and \emph{consistent}. \emph{Simplicity} \emph{cohesion} and \emph{decoupling} make it easier for architects to obtain the required information.

Within in an MDA-based approach, {\bf implementation} (e.g.\ developers or managers who implement and control manual processes) perform the implementation mapping of the services onto a realization within an architecture specified by the architecture team. To this end, they require \emph{syntactic quality} as well as \emph{completeness} and \emph{consistency}. Furthermore, model \emph{simplicity}, \emph{decoupling}, \emph{reuse} and \emph{traceability} typically lead to implementations with those same qualities \cite{podgorelec_estimating_2007}.

{\bf Quality assurance} needs to span across the full development process including quality assurance on the process itself, on the outputs of the different analysis and development activities, and detect and report quality defects in service providers. They require the services contracts specifying both functional and quality requirements for services. \emph{Testability}, \emph{syntactic quality}, \emph{completeness} and \emph{consistency} are required for test generation. Potential service implementations would be tested against the respective service contract. \emph{Traceability} is required for impact reporting.

{\bf Project Management} uses the model measures for reporting and estimation purposes. \emph{Completeness} and \emph{traceability} are required for status reporting on the implementation mapping.

The {\bf client} (business) uses the model to obtain (business) process documentation and services contract documentation. Business process documentation is useful for the general understanding and running of the business as well as for business process optimization. Documentation generation requires \emph{completeness}, \emph{consistency} and \emph{syntactic quality}. \emph{Decoupling} via services contracts is important to the client in the context of sourcing and assessing different concrete service providers (e.g.\ business partners, off-the-shelf systems). More generally, the client commonly requires modifiability in order to have requirements changes cost-effectively realized, benefits from \emph{service reuse} in the context of cost containment and consistency, and requires \emph{traceability} for dependency analysis.

\begin{table}[h]
\caption{Stakeholders and their primary model quality requirements.}
\label{tab:modelQualityRequirements}
\begin{tabular}{|l|cc|cccccccc|} \hline
\multirow{4}{*}{\bf Stakeholders} & \multicolumn{10}{c|}{\bf Model qualities} \\ \cline{2-11}
& & & \multicolumn{8}{c|}{Pragmatic model qualities}\\ \cline{4-11}
    & \begin{sideways}Semantic\end{sideways} & \begin{sideways}Syntactic\end{sideways}  & \begin{sideways}Simplicity\end{sideways}
    & \begin{sideways}Completeness\end{sideways} & \begin{sideways}Modifiability\end{sideways} & \begin{sideways}Consistency\end{sideways}
    & \begin{sideways}Decoupling\end{sideways} & \begin{sideways}Cohesion\end{sideways} & \begin{sideways}Reusability\end{sideways}
    & \begin{sideways}Traceability\end{sideways} \\ \hline
%                         Semantic     Syntax        Simplicity  Completeness   Modifiable  Consistent  Decoupled    Cohesion     Reuse        Traceable
Requirements engineering & \checkmark & \checkmark & \checkmark & \checkmark & \checkmark & \checkmark & \checkmark & \checkmark & \checkmark & \checkmark \\
Architecture             & \checkmark &            & \checkmark & \checkmark &            & \checkmark & \checkmark & \checkmark &            &       \\ 
Implementation           & \checkmark & \checkmark & \checkmark & \checkmark &            & \checkmark & \checkmark & \checkmark & \checkmark & \checkmark \\ 
Quality assurance        & \checkmark & \checkmark & \checkmark & \checkmark &            & \checkmark &            &       &            & \checkmark \\ 
Project management       & \checkmark &            & \checkmark & \checkmark &            & \checkmark &            &       &            & \checkmark \\ 
Client (business)        & \checkmark & \checkmark & \checkmark & \checkmark & \checkmark & \checkmark & \checkmark &            & \checkmark & \checkmark \\ \hline
\end{tabular}
\end{table}

%---------------------------------------------------------------------------------------------------------%

\subsection{Process stakeholders and their quality requirements}

The {\emph Capability Maturity Model} (CMM) \cite{paulk_capability_1993} defines a range of quality attributes for mature processes including \emph{process definition}, \emph{measurability}, \emph{trainability}, \emph{repeatability} and at the higher level the ability to continuously optimize processes for \emph{cost} and other qualities. Berard\cite{berard_what_1995} discusses a related set of general requirements for a methodology including \emph{repeatability}, \emph{trainability}, \emph{wide process applicability}, and quality improvement in the outputs of the methodology. The quality requirements for the outputs of the process have been discussed in \ref{sec:modelStakeholdersAndQualityRequirements}. In this text, the applicability of process has been limited to the services-oriented application domains. Very event-centric projects like game development are not currently targeted service-oriented methodologies. Here we identify the stakeholders in the process itself and their quality requirements for the process. 

{\bf Project management} needs to perform planning, estimation, resource management, monitoring, control and reporting during the analysis and design process. To this end, they require \emph{estimatability}, \emph{repeatability} for accurate estimation and \emph{measurably} and tool support facilitating estimation, monitoring, and reporting.

{\bf Requirements engineers} execute the process itself generating the analysis and design model. They requires a process which is \emph{usable} and \emph{trainable}, and \emph{repeatable}.

The {\bf client} (e.g.\ business) requires \emph{low process cost}, a \emph{trainable} process, enabling the client to easily train up additional requirements engineers, and \emph{isolation}. Process isolation refers here to the ability to grow islands of expertise around the process, extracting value from those teams which apply the process without requiring that the process needs to be rolled out either across an entire project team or across an entire organization. This enables the client to take a risk-averse, incremental approach which allows for proofing, customization and controlled roll-out.

A further quality required of a process is internal consistency. This quality will be discussed in section \ref{sec:urdadConsistency}.

\begin{table}[h]
\caption{Stakeholders and their primary process quality requirements.}
\label{tab:processQualityRequirements}
\begin{tabular}{|l|ccccccc|} \hline
\multirow{2}{*}{\bf Stakeholders} & \multicolumn{7}{c|}{\bf Process qualities} \\ \cline{2-8}
    & \begin{sideways}Low cost\end{sideways}  & \begin{sideways}Repeatability\end{sideways} & \begin{sideways}Estimatability\end{sideways}
    & \begin{sideways}Trainable\end{sideways}
    & \begin{sideways}Measurability\end{sideways} & \begin{sideways}Consistency\end{sideways} & \begin{sideways}Isolation\end{sideways} \\ \hline
%                          Cost         Repatable    Estimatable  Trainable    Measurable  Consistent   Isolated
Project management       &            & \checkmark & \checkmark & \checkmark &            & \checkmark & \checkmark \\
Requirements engineering & \checkmark & \checkmark & \checkmark & \checkmark & \checkmark & \checkmark & \checkmark \\
Client (business)        & \checkmark & \checkmark & \checkmark & \checkmark & \checkmark & \checkmark \\ \hline
\end{tabular}
\end{table}
