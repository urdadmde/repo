\subsection{Quality requirements}

URDAD defines an analysis and design process with defined inputs and outputs for each process step and a metamodel capturing the outputs of the process. The quality of a process may be expected to affect the quality of the outputs of the process. In the case of URDAD this is the case as the methodology defines the semantics and structure of the resultant model through a metamodel.

In this section we aim to identify the stakeholders of both, the process and the resultant model and their respective quality requirements.


\subsection{Process stakeholders and their quality requirements}

In order to be able to assess the quality of the process and of the resultant model, we need to identify the quality requirements of the various stakeholders in the process and model respectively.


\cite{berard_what_1995}

Stake holders:
\begin{itemize}
  \item Project management (measurability, repeatability, estimatability)
  \item Requirements specialists (ease of use, simple process, defined process activities, defined inputs and outputs, tool support)
  \item Business (low cost, trainable, grow in islands)
\end{itemize}


\begin{itemize}
  \item Process measurability
  \item Repeatability
  \item Defined inputs and outputs
  \item Clearly specified tasks with defined activities
  \item Process consistency (URDAD generates itself)
\end{itemize}

Can apply process to a sub-world, decouples from higher and lower level granularities via contracts

Could introduce more abstract qualities and things the process must have to realize these, e.g. \emph{usability} affected by many of these

CMM requires process definition


\subsection{Internal process consistency}

Here show that if you use URDAD to design an analysis and design methodlogy, you will get URDAD. Feed additional concepts into URDAD.

Here identify the stakeholders in both, the process and the outputs of the process and their quality requirements.

\subsection{Model stakeholders and their quality requirements}