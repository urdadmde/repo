\section{Quality requirements}
\label{sec:qualityRequirements}

In order to be able to assess quality and quality-drivers, one needs to identify the stakeholders in both the process and the resultant analysis and design model and their respective quality requirements for each. In this section, we first look at the stakeholders who have an interest in the model itself and their quality requirements for the analysis and design model before looking at the process stakeholders and their quality requirements for the analysis and design process itself. For the purpose of this paper, we take a manufacturing view \cite{garvin_what_1984} to quality, i.e.\ we assess quality from the perspective of conformance to requirements. 

%-------------------------------------------------

\subsection{Model stakeholders and their quality requirements}
\label{sec:modelStakeholdersAndQualityRequirements}

A lot of work has historically been done on defining and measuring code quality (see, for example, \cite{boehm_barry_w._characteristics_1978}), but recently the focus has shifted onto understanding and measuring model quality \cite{lange_managing_2005,lange_improving_2006,shim_design_2008,qi_yu-dong_analysis_2010}. Though most of the work is applied to UML models, the concepts are generally applicable across analysis and design models, i.e.\ they are generally not UML specific. 

Lindland\cite{lindland_understanding_1994} provides a widely used quality categorization into \emph{semantic quality}, which assesses whether a model confers the meaning it was meant to confer, \emph{syntactic quality}, which is a measure of the conformance to the language used to specify the model, and \emph{pragmatic quality}, which is a measure of the extent to which the model can satisfy its intended use. Christiaan Lange  \cite{lange_christiaan_assessing_2007} related model quality back to model purpose and identified a range of pragmatic model quality characteristics including complexity, traceability, modularity, completeness, consistency and communicativeness. Models that exhibit such qualities have been shown to commonly lead to systems with similar quality attributes \cite{podgorelec_estimating_2007}.

In this paper we do not consider communicativeness and social quality, as we feel these will be strongly influenced by semantics as well as the model user interface generated for specific role players. For example, requirements specialists would typically use a diagrammatic user interface into the model provided by a graphical grammar and appropriate tool support. The user interface to the model provided to the client is typically a read-only interface in the form of documentation generated from the model. Implementation and quality assurance can often use the model directly in the context of model transformation in order to generate code and test cases.

We also assume that a services-oriented approach is applicable for the chosen modeling domain and that the intent is to follow a model-driven approach within an MDA framework.

The core stakeholders in the model include {\bf requirements engineering} (e.g.\ business analysis), {\bf architecture}, {\bf implementation} (e.g.\ developers), {\bf quality assurance} (e.g.\ testing), {\bf project management} and the {\bf client} (e.g.\ business). In order to address their respective responsibilities, these stakeholders require certain model qualities. All of these stakeholders require semantic quality (since they need to extract the correct meaning from the model) whilst syntactic quality is required for the automated processing of the model. The latter is particularly important for implementation (code), test and documentation generation - hence for implementation, quality assurance and requirements engineering. The pragmatic quality requirements differ across stakeholders, though \emph{simplicity} and \emph{consistency} is a quality which is beneficial to all.

{\bf Requirements engineers} (e.g.\ business analysts) use the model to capture the functional and non-functional requirements of services in the form of services contracts and (business) process designs realizing these services contracts. They also may use the model for model validation and documentation generation, requiring \emph{syntactic quality} to make this feasible.  The \emph{pragmatic qualities} particularly relevant to requirements engineers include \emph{modifiability} (as they would have to modify the model in the context of changing requirements and process improvement), \emph{simplicity} assisting understandability and completeness, \emph{consistency}, \emph{decoupling} of functional requirements from process design, \emph{reuse} to reduce model complexity and increase model consistency, and \emph{traceability} to facilitate model validation and improve \emph{modifiability}. 

For the purpose of this paper, we define architecture as the infrastructure within which processes are deployed and executed. Thus, whilst implementation is concerned with the actual services implementations, the architecture team is concerned with the infrastructure hosting the services. The architecture may span organizational, hardware and software infrastructure. Organizational and systems architects need to be able to design an architecture hosting the services in such a way that the quality requirements (\emph{quality of service}) can be met. They also need to be able to assess whether an existing architecture can host a service as per contract and specify architectural modifications if that is not the case. {\bf Architecture} typically uses the high level services contracts and process specifications. These need to be \emph{complete} and \emph{consistent}. \emph{Simplicity} \emph{cohesion} and \emph{decoupling} make it easier for architects to obtain the required information.

Within in an MDA-based approach, {\bf implementation} (e.g.\ developers or managers who implement and control manual processes) perform the implementation mapping of the services onto a realization within an architecture specified by the architecture team. To this end, they require syntactic quality as well as \emph{completeness} and \emph{consistency}. Furthermore, model \emph{simplicity}, \emph{decoupling}, \emph{reuse} and \emph{traceability} typically lead to implementations with those same qualities \cite{podgorelec_estimating_2007}.

{\bf Quality assurance} needs to span across the full development process including quality assurance on the process itself, on the outputs of the different analysis and development activities, and detect and report quality defects in service providers. They require the services contracts specifying both functional and quality requirements for services. \emph{Testability}, \emph{syntactic quality}, \emph{completeness} and \emph{consistency} are required for test generation. Potential service implementations would be tested against the respective service contract. \emph{Traceability} can assist with impact reporting.

{\bf Project Management} requires \emph{completeness} assessment and \emph{traceability}, as well, to support estimation and status reporting.

The {\bf client} (business) uses the model to obtain (business) process documentation and services contract documentation. Business process documentation is useful for the general understanding and running of the business as well as for business process optimization. Documentation generation requires \emph{completeness}, \emph{consistency} and \emph{syntactic quality}. \emph{Decoupling} via services contracts is important to the client in the context of sourcing and assessing different concrete service providers (e.g.\ business partners, off-the-shelf systems). More generally, the client commonly requires modifiability in order to have requirements changes cost-effectively realized, benefits from \emph{service reuse} in the context of cost containment and consistency, and requires \emph{traceability} for dependency analysis.

\begin{table}[h]
\caption{Stakeholders and their primary model quality requirements.}
\label{tab:modelQualityRequirements}
\begin{tabular}{|l|cc|cccccccc|} \hline
\multirow{4}{*}{\bf Stakeholders} & \multicolumn{10}{c|}{\bf Model qualities} \\ \cline{2-11}
& & & \multicolumn{8}{c|}{Pragmatic model qualities}\\ \cline{4-11}
    & \begin{sideways}Semantic\end{sideways} & \begin{sideways}Syntactic\end{sideways}  & \begin{sideways}Simplicity\end{sideways}
    & \begin{sideways}Completeness\end{sideways} & \begin{sideways}Modifiability\end{sideways} & \begin{sideways}Consistency\end{sideways}
    & \begin{sideways}Decoupling\end{sideways} & \begin{sideways}Cohesion\end{sideways} & \begin{sideways}Reusability\end{sideways}
    & \begin{sideways}Traceability\end{sideways} \\ \hline
%                         Semantic     Syntax        Simplicity  Completeness   Modifiable  Consistent  Decoupled    Cohesion     Reuse        Traceable
Requirements engineering & \checkmark & \checkmark & \checkmark & \checkmark & \checkmark & \checkmark & \checkmark & \checkmark & \checkmark & \checkmark \\
Architecture             & \checkmark &            & \checkmark & \checkmark &            & \checkmark & \checkmark & \checkmark &            &       \\ 
Implementation           & \checkmark & \checkmark & \checkmark & \checkmark &            & \checkmark & \checkmark & \checkmark & \checkmark & \checkmark \\ 
Quality assurance        & \checkmark & \checkmark & \checkmark & \checkmark &            & \checkmark &            &       &            & \checkmark \\ 
Project management       & \checkmark &            & \checkmark & \checkmark &            & \checkmark &            &       &            & \checkmark \\ 
Client (business)        & \checkmark & \checkmark & \checkmark & \checkmark & \checkmark & \checkmark & \checkmark &            & \checkmark & \checkmark \\ \hline
\end{tabular}
\end{table}

%---------------------------------------------------------------------------------------------------------%

\subsubsection{Model quality-drivers and metrics}
\label{sec:modelQualityDriversAndMetrics}

Model quality-drivers are activities in the process and/or aspects of the metamodel which are included to assist with addressing model quality requirements - they are the basis for the \emph{`quality-driven'} process. One approach which can be used to assess whether a methodology provides a quality-driven process is to assess which of the quality-drivers are supported by the methodology. The list of quality-drivers presented here is not intended to be exhaustive. It is purely a collection of quality-drivers which are widely known and used. In addition to identifying quality-drivers, we also note that quality metrics can be applied to the resultant model in order to assess model quality. This section discusses some commonly used quality metrics for the identified quality requirements suitable for a services-oriented approach. For a more general list of model quality metrics the reader should consult \cite{mohagheghi_existing_2009}.

\paragraph{Semantic Quality} is a measure of the accuracy and completeness of the meaning conveyed with the analysis and design model. Semantic quality-drivers include the specification of an ontology and/or a metamodel for the analysis and design model. One needs to assess the semantic overlap between the modeling domain and the ontology provided by the modeling language. Though a number of studies discuss the relationship between semantics and metamodels \cite{staab_model_2010,veldhuis_tool_2009,henderson-sellers_bridging_2011} and the effect of semantic quality on the quality of analysis and design models \cite{buder_effects_2010,staab_model_2010}, little work has thus far been done on the quantitative assessment of semantic model quality. Domain specific languages aim to explicitly and directly cover the ontological space required by the modeling methodology or/and the modeling domain. The semantic quality assessment will thus be largely confined to the assessment of the semantic support provided by the employed modeling language.

\paragraph{Syntactic Quality} is a measure of correct language usage. Syntactic quality-drivers include the availability of a formal metamodel for the language as well as the specification of concrete text and graphical grammars for the specification of model instances. In this paper we confine the assessment of syntactic quality to the existence of a formal modeling language and concrete grammar for model specification. We will thus assume that model instances will adhere to the grammar of the chosen modeling language as this can easily be verified. Syntactic quality can easily be measured as a ratio of the number of syntax errors relative to the model size.

\paragraph{Simplicity} is a measure of lack of complexity. The complexity of the modeling language is usually assessed by measuring the complexity of the metamodel for that language\cite{mohagheghi_evaluating_2007}. A lot of work has been done on model complexity itself. Common approaches include using information entropy measures\cite{abrahamsson_extreme_2004}, language-theoretic approaches\cite{podgorelec_estimating_2007} and process complexity assessment based on the McCabe complexity measure \cite{mccabe_complexity_1976}. A services-oriented approach already enforces certain drivers for simplicity. In particular the assembling of processes from independent, stateless services, the enforced decoupling through services contracts, the improved reuse through the implied service discovery, as well as an implied adapter layer facilitating reuse across technologies and interfaces mismatches. A metamodel confining the modeling constructs and relationships between these, as well as a convenient grammar, reduce complexity. Another core driver for simplicity is the decomposition across levels of granularity enabling the understanding and processing of one level of granularity before having to concern oneself with the details of the next lower level of granularity. The availability of a services contract enables one to look at a service from a user/service consumer perspective before understanding the service process. Finally, traceability links and in particular enforced satisfaction links (i.e.\ that a process may only use services which address functionality) are drivers for simplicity.

\paragraph{Completeness} is a measure of the lack of missing information in the analysis and design model. Completeness of requirements is difficult to assess, though the discovery of certain stakeholders for which no functional or non-functional requirements exist is an indication of missing requirements. Design completeness is easier to assess as there is a global measure - the degree to which the requirements are realized. Core quality-drivers for completeness include enforced satisfaction links, the enforcement of testable pre- and post-conditions, as well as having a metamodel which enforces certain content, either structurally, or through metamodel constraints. In a service-oriented approach, a completeness metric can be specified as a function of 1) the percentage of functional requirements not addressed in the designed processes (identified through missing satisfaction links)\cite{shim_design_2008}, 2) the fraction of pre- and post-conditions for which the test process has not been specified and 3) the percentage of request and result classes for which the data structures have not been specified. At the implementation level completeness can be assessed as the fraction of services for which service implementations are not yet available.

\paragraph{Consistency} is an important model quality measure. A lot of emphasis has been placed on model consistency for UML models. This is so because UML models are often inconsistent due to the multi-diagram approach and the complexity of the modeling language itself. A commonly used quality-driver for model consistency is to use a much smaller (less general) language with more formally defined semantics and formal textual and/or graphical grammars. 

\paragraph{Cohesion} in a service-oriented approach has been extensively studied by Mikhael Perepletchikov et al.\ \cite{perepletchikov_cohesion_2007,perepletchikov_impact_2010}. Their approach is narrowly related to that of unity criteria \cite{gonzalez_unity_2009} identifying interfacing cohesion, usage cohesion and implementation cohesion. Quality-drivers for cohesion include enforcing the single responsibility principle as well as process localization within a controller service.

\paragraph{Decoupling} is enforced in a services-oriented approach by requiring that services are only consumed via services contracts. This decouples services across levels of granularity. Quality-drivers for decoupling include a metamodel which enforces contracts based decoupling through metamodel structure or constraints and localization of process logic within a controller so that the service providers of the lower level services remain decoupled. In a services-oriented architecture, the number of services which are directly coupled to lower level they consume is a direct measure of coupling\cite{shim_design_2008}.

\paragraph{Modifiability} refers to the efficiency with which model changes can be applied, i.e.\ it is an inverse measure of the cost required to make model changes. Modifiability is difficult to quantitatively measure, but it is supported by other model qualities like \emph{simplicity},  \emph{decoupling} (modifiability through pluggability), and \emph{cohesion} (localized maintenance) and with a further quality-driver in the form of localizing process logic within a controller service, thereby projecting out additional levels of granularity. \cite{shim_design_2008} defines a quality metric for modifiability (flexibility) of services-oriented designs as a weighted sum of coupling, service granularity, and parameter granularity. We feel that simplicity (or complexity) and decoupling should also be included in the quality measure for modifiability.

\paragraph{Reusability} provides a measure of the ability and likelihood that a service can be reused.  \cite{khoshkbarforoushha_metric_2010,choi_quality_2008,feuerlicht_determinants_2007}
\cite{khoshkbarforoushha_metric_2010} point out that service reusability is impeded by contract and requirements mismatch. The former can be addressed via adapters. Core quality-drivers for reusability include \emph{decoupling} via services contracts with the latter also driving discoverability and consumability, \emph{levels of granularity} via process localization within a reusable controller service, and \emph{cohesion} through enforcing the single responsibility principle. Indeed, \cite{shim_design_2008} defines a simple quantitative reusability measure for service-oriented systems as a weighted sum of coupling, cohesion, granularity, and consumability.

\paragraph{Traceability} is required for design validation and estimation. Validation includes assessing sufficiency and necessity. \cite{ramesh_toward_2001} identifies four types of traceability links in models including \emph{satisfaction} links used to assess whether requirements are satisfied, \emph{evolutionary} links to trace along the evolution of an artifact over time, \emph{rationale} links which typically link requirements to higher level business goals, and \emph{dependency} links enabling one to identify dependencies of model elements. Quality-drivers include thus the availability of these traceability links in the modeling language and their enforced usage through the modeling process.

%---------------------------------------------------------------------------------------------------------%

\subsection{Process stakeholders and their quality requirements}

The {\emph Capability Maturity Model} (CMM) \cite{paulk_capability_1993} defines a range of quality attributes for mature processes including \emph{process definition}, \emph{measurability}, \emph{trainability}, \emph{repeatability} and at the higher level the ability to continuously optimize processes for \emph{cost} and other qualities. Berard\cite{berard_what_1995} discusses a related set of general requirements for a methodology including \emph{repeatability}, \emph{trainability}, \emph{wide process applicability}, and quality improvement in the outputs of the methodology. The quality requirements for the outputs of the process have been discussed in \ref{sec:modelStakeholdersAndQualityRequirements}. In this text, the applicability of process has been limited to the services-oriented application domains. Very event-centric projects like game development are not currently targeted service-oriented methodologies. Here we identify the stakeholders in the process itself and their quality requirements for the process. 

{\bf Project management} needs to perform planning, estimation, resource management, monitoring, control and reporting during the analysis and design process. To this end, they require \emph{estimatability}, \emph{repeatability} for accurate estimation and \emph{measurably} and tool support facilitating estimation, monitoring, and reporting.

{\bf Requirements engineers} execute the process itself generating the analysis and design model. They requires a process which is \emph{usable} and \emph{learn able}, and has solid \emph{process definition} with specified inputs, outputs and activities for each process step and tool support for the process.

The {\bf client} (e.g.\ business) requires \emph{low process cost}, a \emph{trainable} process, enabling the client to easily train up additional requirements engineers, and \emph{isolation}. Process isolation refers here to the ability to grow islands of expertise around the process, extracting value from those teams which apply the process without requiring that the process needs to be rolled out either across an entire project team or across an entire organization. This enables the client to take a risk-averse, incremental approach which allows for proofing, customization and controlled roll-out.

A further quality required of a process is internal consistency. This quality will be discussed in section \ref{sec:processConsistence}.

\begin{table}[h]
\caption{Stakeholders and their primary process quality requirements.}
\label{tab:modelQualityRequirements}
\begin{tabular}{plp} \hline
\multirow{2}{*}{\bf Stakeholders} & \multicolumn{7}{cp}{\bf Process qualities} \\ \cline{2-8}
    & \begin{sideways}process definition\end{sideways} & \begin{sideways}Low cost\end{sideways}  & \begin{sideways}Repeatability\end{sideways}
    & \begin{sideways}Estimatability\end{sideways} & \begin{sideways}Measurability\end{sideways} & \begin{sideways}Consistency\end{sideways}
    & \begin{sideways}Isolation\end{sideways} \\ \hline
%                         Proc-Def     Cost         Repatable    Estimatable   Measurable  Consistent   Isolated
Project management       & \checkmark &            & \checkmark & \checkmark &            & \checkmark & \checkmark \\
Requirements engineering & \checkmark & \checkmark & \checkmark & \checkmark & \checkmark & \checkmark & \checkmark \\
Client (business)        & \checkmark & \checkmark & \checkmark & \checkmark & \checkmark & \checkmark & \checkmark \\ \hline
\end{tabular}
\end{table}



