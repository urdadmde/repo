\subsection{Quality requirements}

\todo[inline,color=green!40]{Fritz:In this section we should refrain from referring to URDAD. Lets look purely at the requirements of a services-oriented analysis and design process for MDA}

URDAD defines an service-oriented analysis and design process with defined inputs and outputs for each process step and a metamodel capturing the outputs of the process. In this section we aim to identify the stakeholders of both, the process and the resultant model and their respective quality requirements.

\subsection{Model stake holders and their quality requirements}

A services oriented analysis and design methodology for MDA needs to yield a model which is platform independent and provides 1) the service contract for any required service, 2) the domain of responsibility from which such a service would have to be sourced (facilitating service discovery and responsibility localization), and 3) the service process in the form of an `orchestration' across lower level services\cite{}.

In order to understand the model quality requirements of the different stakeholders, we need to understand the purpose for which they use the model\cite{lange_christiaan_assessing_2007}. {\bf Requirements engineers} (e.g.\ business analysts) use the model to capture the functional and non-functional requirements of a service as well as the (business) process design assembled from lower level services available to the system or organization. They require \emph{modifiability} as they would have to modify the model in the context of changing requirements and process improvement. In order to contain cost and model complexity they would like the model to facilitate reuseIn this context they would like to \emph{reuse} model elements

Organizational and system {\bf Architects}

{\bf Developers} use the model as well as the architecture specification to perform the implementation mapping of services.




\cite{lange_managing_2005,lange_improving_2006}.

\subsection{Model measures}
Emphasize that model a functional model, bot addressing non-functional (quality requirements) of system

\begin{itemize}
  \item \emph{testability} measured as fraction of non-testable requirements.
  \item \emph{traceability} measured as fraction of requirements which are not related to the stake holder who requires them, fraction of activities which are not traceable to the functional requirement they aim to address, 
  \item \emph{complexity} based on number and complexity of services with complexity of service necessarily related to Mcabe (cyclomatic) complexity - not sure that URDAD addresses this.
  \item \emph{cohesion} - will be interesting to try and find a measure for cohesion in the service oriented world.
  \item \emph{reuse} - number of
  \item \emph{coupling} - URDAD enforces decoupling across levels of granularity in both, the process and the metamodel.
  \item \emph{completeness}
  \item \emph{non-ambiguous} = well defined semantics = understandability
\end{itemize}


MDA requires model to be technology neutral.


which is meant to be used to generate the platform independent model for MDA

\subsection{Process stakeholders and their quality requirements}

In order to be able to assess the quality of the process and of the resultant model, we need to identify the quality requirements of the various stakeholders in the process and model respectively.


\cite{berard_what_1995}

Stake holders:
\begin{itemize}
  \item Project management (measurability, repeatability, estimatability)
  \item Requirements specialists (ease of use, simple process, defined process activities, defined inputs and outputs, tool support)
  \item Business (low cost, trainable, grow in islands)
\end{itemize}


\begin{itemize}
  \item Process measurability
  \item Repeatability
  \item Defined inputs and outputs
  \item Clearly specified tasks with defined activities
  \item Process consistency (URDAD generates itself)
\end{itemize}

Can apply process to a sub-world, decouples from higher and lower level granularities via contracts

Could introduce more abstract qualities and things the process must have to realize these, e.g. \emph{usability} affected by many of these

CMM requires process definition


\subsection{Internal process consistency}

\todo[inline,color=blue!40]{Reuse sounds to me like both, a process and a model quality}

Here show that if you use URDAD to design an analysis and design methodlogy, you will get URDAD. Feed additional concepts into URDAD.

Here identify the stakeholders in both, the process and the outputs of the process and their quality requirements.

\subsection{Model stakeholders and their quality requirements}