\subsection{Quality requirements}

In order to be able to measure quality, we need to identify the quality requirements. Requirements only make sense from the perspective of the stakeholder who requires the requirment. Hence to identify quality requirements we need to first identify the stakeholders who have an interest in the process as well as those who have an interest in the resultant analysis and design model. Once we have identified the stakeholders, we can elicit the quality requirements

We differentiate between quality requirements for the process itself from the quality requirements on the outputs (the analysis and design model).

\cite{berard_what_1995}

Stake holders:
\begin{itemize}
  \item Project management (measurability, repeatability, estimatability)
  \item Requirements specialists (ease of use, simple process, defined process activities, defined inputs and outputs, tool support)
  \item Business (low cost, trainable, grow in islands)
\end{itemize}


\begin{itemize}
  \item Process measurability
  \item Repeatability
  \item Defined inputs and outputs
  \item Clearly specified tasks with defined activities
  \item Process consistency (URDAD generates itself)
\end{itemize}

Can apply process to a sub-world, decouples from higher and lower level granularities via contracts

Could introduce more abstract qualities and things the process must have to realize these, e.g. \emph{usability} affected by many of these

CMM requires process definition


\subsection{Internal process consistency}

Here show that if you use URDAD to design an analysis and design methodlogy, you will get URDAD. Feed additional concepts into URDAD.

Here identify the stakeholders in both, the process and the outputs of the process and their quality requirements.