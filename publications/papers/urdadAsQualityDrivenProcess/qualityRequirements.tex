\section{Quality requirements}

\todo[inline,color=green!40]{Fritz:Thanks for your suggestions. Followed them all bar the title change. I hope the introductory paragraph addresses your concerns and that this makes you happy to go along with the current title. Feel free to red-flag.}

In order to be able to assess quality and quality-drivers one needs to identify the stakeholders in both, the process and the resultant analysis and design model and their respective quality requirements. In this section, we first look at at the stakeholders who have an interest in the model itself and their quality requirements for the analysis and design model before looking at the process stake holders and the quality requirements for the analysis and design process itself.

\subsection{Model stake holders and their quality requirements}

A lot of work has historically been done on defining and measuring code quality (see, for example, \cite{boehm_barry_w._characteristics_1978}), but recently the focus has shifted onto understanding and measuring model quality \cite{lange_managing_2005,lange_improving_2006,shim_design_2008,qi_yu-dong_analysis_2010}. Though most of the work is applied to UML models, the concepts are generally applicable across analysis and design models, i.e.\ they are generally not UML specific. Lindland \cite{lindland_understanding_1994} provides a widely used quality categorization into \emph{semantic quality}, \emph{syntactic quality}, \emph{pragmatic quality}. However, the detailed pragmatic model quality requirements vary considerably across authors with model \emph{complexity} being the one measure which is common across most approaches. Christiaan Lange  \cite{lange_christiaan_assessing_2007} related model quality back to model purpose and identified a range of model quality characteristics including complexity, traceability, modularity, completeness, consistency, communicativeness, and others which are largely related to these.

In this paper we do not consider communicativeness and social quality as we feel these will be strongly influenced by semantics as well as the model user interface generated for specific role players. For example, requirements specialists would typically use a diagrammatic user interface into the model provided by a graphical syntax and appropriate tool support. The user interface to the model provided to the client is typically a read-only interface in the form of documentation generated from the model. Implementation and quality assurance can often use the model directly in the context of model transformation in order to generate code and test cases.

We also assume that a services-oriented approach is applicable for the chosen modeling domain and that the intent is to follow a model-driven approach within an MDA framework.

The core stakeholders in the model include {\bf requirements engineering} (e.g.\ business analysis), {\bf architecture}, {\bf implementation} (e.g.\ developers), {\bf quality assurance} (e.g.\ testing), {\bf project management} and the {\bf client} (e.g.\ business). In order to address their respective responsibilities, these stakeholders require certain model qualities. All of these stakeholders require semantic quality whilst syntactic quality is required for the automated processing of the model. The latter is particularly important for implementation (code), test and documentation generation - hence for implementation, quality assurance and requirements engineering. The pragmatic quality requirements differ across stakeholders, though \emph{simplicity} and \emph{consistency} is a quality which is beneficial to all.

{\bf Requirements engineers} (e.g.\ business analysts) use the model to capture the functional and non-functional requirements of  services in the form of services contracts and (business) process designs realizing these services contracts. They also may use the model for model validation and documentation generation, requiring \emph{syntactic quality} to make this feasible.  The \emph{pragmatic qualities} particularly relevant to requirements engineers include \emph{modifiability} (as they would have to modify the model in the context of changing requirements and process improvement), \emph{simplicity} assisting understandability and completeness, \emph{consistency}, \emph{decoupling} of functional requirements from process design, \emph{reuse} to reduce model complexity and increase model consistency, and \emph{traceability} to facilitate model validation and improve \emph{modifiability}. 

For the purpose of this paper we define architecture as the infrastructure within which processes are deployed and executed. Thus, whilst implementation is concerned with the actual services implementations, the architecture team is concerned with the infrastructure hosting the services. The architecture may span organizational, hardware and software infrastructure. Organizational and systems architects need to be able to design an architecture hosting the services in such a way that the quality requirements (\emph{quality of service}) can be met. They also need to be able to assess whether an existing architecture can host a service as per contract and specify architectural modifications if that is not the case. {\bf Architecture} typically uses the high level services contracts and process specifications. These need to be \emph{complete} and \emph{consistent}. \emph{Simplicity} \emph{cohesion} and \emph{decoupling} make it easier for architects to obtain the required information.

Within in an MDA-based approach, {\bf implementation} (e.g.\ developers or managers who implement and control manual processes) perform the implementation mapping of the services onto a realization within in an architecture specified by the architecture team. To this end, they require syntactic quality as well as \emph{completeness} and \emph{consistency}. Furthermore, model \emph{simplicity}, \emph{decoupling}, \emph{reuse} and \emph{traceability} typically lead to implementations with those same qualities \cite{podgorelec_estimating_2007}.

{\bf Quality assurance} needs to span across the full development process including quality assurance on the process itself, on the outputs of the different analysis and development activities detect and report quality defects in in service providers. They require the services contracts specifying both, functional and quality requirements for services. \emph{Testability}, \emph{syntactic quality}, \emph{completeness} and \emph{consistency} are required for test generation. Potential service implementations would be tested against the respective service contract. \emph{Traceability} can assist with impact reporting.

{\bf Project Management} requires \emph{completeness} assessment and \emph{traceability} as well to support estimation and status reporting.

The {\bf client} (business) uses the model to obtain (business) process documentation and services contract documentation. Business process documentation is useful for the general understanding and running of the business as well as for business process optimization. Documentation generation requires \emph{completeness}, \emph{consistency} and \emph{syntactic quality}. \emph{Decoupling} via services contracts is important to the client in the context of sourcing and assessing different concrete service providers (e.g.\ business partners, off-the-shelf systems). More generally, the client commonly requires modifiability in order to have requirements changes cost-effectively realized, benefits from \emph{service reuse} in the context of cost containment and consistency, and requires \emph{traceability} for dependency analysis.

\begin{table}[h]
\caption{Stakeholders and their primary model quality requirements.}
\label{tab:modelQualityRequirements}
\begin{tabular}{|l|cc|cccccccc|} \hline
\multirow{4}{*}{\bf Stakeholders} & \multicolumn{10}{c|}{\bf Model qualities} \\ \cline{2-11}
& & & \multicolumn{8}{c|}{Pragmatic model qualities}\\ \cline{4-11}
    & \begin{sideways}Semantic\end{sideways} & \begin{sideways}Syntactic\end{sideways}  & \begin{sideways}Simplicity\end{sideways}
    & \begin{sideways}Completeness\end{sideways} & \begin{sideways}Modifiability\end{sideways} & \begin{sideways}Consistency\end{sideways}
    & \begin{sideways}Decoupling\end{sideways} & \begin{sideways}Cohesion\end{sideways} & \begin{sideways}Reusability\end{sideways}
    & \begin{sideways}Traceability\end{sideways} \\ \hline
%                         Semantic     Syntax        Simplicity  Completeness   Modifiable  Consistent  Decoupled    Cohesion     Reuse        Traceable
Requirements engineering & \checkmark & \checkmark & \checkmark & \checkmark & \checkmark & \checkmark & \checkmark & \checkmark & \checkmark & \checkmark \\
Architecture             & \checkmark &            & \checkmark & \checkmark &            & \checkmark & \checkmark & \checkmark &            &       \\ 
Implementation           & \checkmark & \checkmark & \checkmark & \checkmark &            & \checkmark & \checkmark & \checkmark & \checkmark & \checkmark \\ 
Quality assurance        & \checkmark & \checkmark & \checkmark & \checkmark &            & \checkmark &            &       &            & \checkmark \\ 
Project management       & \checkmark &            & \checkmark & \checkmark &            & \checkmark &            &       &            & \checkmark \\ 
Client (business)        & \checkmark & \checkmark & \checkmark & \checkmark & \checkmark & \checkmark & \checkmark &            & \checkmark & \checkmark \\ \hline
\end{tabular}
\end{table}

%---------------------------------------------------------------------------------------------------------%

\subsubsection{Model quality drivers and metrics}

In order to analyze to what extend a methodology and its output satisfy their quality requirements we can a) assess which \emph{quality drivers} are built into the methodology and b) specify the quality metrics used to assess the quality of the resultant model. Here we identify quality drivers and some possible simple quality metrics.

\cite{mohagheghi_existing_2009} discusses a number of model quality measures including measures for complexity, reuse.


\paragraph{Semantic Quality} is a measure of the accuracy and completeness of the meaning conveyed with the analysis and design model. Semantic quality drivers include the specification of an ontology and/or a metamodel for the analysis and design model. One needs to assess the semantic overlap between the modeling domain and the ontology provided by the modeling language. Though a number of studies discuss the relationship between semantics and metamodels \cite{staab_model_2010,veldhuis_tool_2009,henderson-sellers_bridging_2011} and the effect of semantic quality on the quality of analysis and design models \cite{buder_effects_2010,staab_model_2010}, little work has thus far been done on the quantitative assessment of semantic model quality. Domain specific languages aim to explicitly and directly cover the ontological space required by the modeling methodology or/and the modeling domain. The semantic quality assessment will thus be largely confined to the assessment of the semantic support provided by the employed modeling language.

\paragraph{Syntactic Quality} is a measure of correct language usage. Syntactic quality drivers include the availability of a formal metamodel for the language as well as the specification of concrete text and graphical grammars for the specification of model instances. In this paper we confine the assessment of syntactic quality to the existence of a formal modeling language and concrete syntax for model specification, assuming that model instances will adhere to the syntax of the chosen modeling language.

\paragraph{Simplicity} is a measure of lack of complexity. The complexity of the modeling language is usually assessed by measuring the complexity of the metmodel for that language\cite{mohagheghi_evaluating_2007}. A lot of work has been done on model complexity itself. Common approaches include using information entropy measures\cite{abrahamsson_extreme_2004}, language-theoretic approaches\cite{podgorelec_estimating_2007} and process complexity assessment based on the Mcabe complexity measure \cite{gill_new_2010}. A services oriented approach already enforces certain drivers  for simplicity. In particular the assembling of processes from independent, stateless services, the enforced decoupling through services contracts and the improved reuse through the implied service discovery an implied adapter facilitating reuse across technologies and interfaces mismatches. In addition the explicit use of services to only address functional requirements can be seen as a quality driver form simplicity.

\paragraph{Completeness} is a measure of the lack of missing information in the analysis and design model. Completeness of requirements is difficult to assess, though the discovery of certain stakeholders for which no functional or nonfunctional requirements exist is an indication of missing requirements. In order to assess design completeness one can identify the number of functional requirements which are not addressed through using lower level services.

\paragraph{Consistency} is an important model quality measure. Since model consistency is difficult to achieve using UML A lot of emphasis has been placed on model consistency for UML models

\paragraph{Cohesion} in a service oriented approach has been extensively studied by Mikhael Perepletchikov et al.\ \cite{perepletchikov_cohesion_2007,perepletchikov_impact_2010}. Their approach is narrowly related to that of unity criteria \cite{gonzalez_unity_2009} identifying interfacing cohesion, usage cohesion and implementation cohesion. Quality drivers for cohesion include enforcing the single responsibility principle as well as process locacalization within a controller service.

\paragraph{Decoupling} is enforced in a services oriented approach by requiring that services are only consumed via services contracts. This decouples services across levels of granularity.

\paragraph{Modifiability} refers to the efficiency with which model changes can be applied, i.e.\ it is an inverse measure of the cost required to make model changes. Modifiability is difficult to quantitatively measure, but it is supported by other model qualities like \emph{simplicity},  \emph{decoupling} (modifiability through pluggability) and \emph{cohesion} (localized maintenance) and further driven by enforcing the localization of process logic within a controller service.

\paragraph{Reusability} metrics are \cite{khoshkbarforoushha_metric_2010,choi_quality_2008,feuerlicht_determinants_2007}
\cite{khoshkbarforoushha_metric_2010} point out thatservice reusability impeded by contract and requirements mismatch. The former can be addressed via adapters.

\paragraph{Traceability} is required for design validation and estimation. Validation includes assessing sufficiency and necessity. \cite{ramesh_toward_2001} identifies four types of traceability links in models including \emph{satisfaction} links used to assess whether requirements are satifies, \emph{evolutional} links to trace along the evolution of an artifact over time, \emph{rationale} links which typically link requirements to higher level business goals, and \emph{dependency} links enabling one to identify dependencies of model elements. Quality drivers include thus the availability of these traceability links in the modeling language and their enforced usage through the modeling process.

%---------------------------------------------------------------------------------------------------------%

\subsection{Process stakeholders and their quality requirements}

In order to be able to assess the quality of the process and of the resultant model, we need to identify the quality requirements of the various stakeholders in the process and model respectively.


\cite{berard_what_1995}

Stake holders:
\begin{itemize}
  \item Project management (measurability, repeatability, estimatability)
  \item Requirements specialists (ease of use, simple process, defined process activities, defined inputs and outputs, tool support)
  \item Business (low cost, trainable, grow in islands)
\end{itemize}


\begin{itemize}
  \item Process measurability
  \item Repeatability
  \item Defined inputs and outputs
  \item Clearly specified tasks with defined activities
  \item Process consistency (URDAD generates itself)
\end{itemize}

Can apply process to a sub-world, decouples from higher and lower level granularities via contracts

Could introduce more abstract qualities and things the process must have to realize these, e.g. \emph{usability} affected by many of these

CMM requires process definition


\subsection{Internal process consistency}

\todo[inline,color=blue!40]{Fritz: Reuse sounds to me like both, a process and a model quality}

Here show that if you use URDAD to design an analysis and design methodlogy, you will get URDAD. Feed additional concepts into URDAD.

Here identify the stakeholders in both, the process and the outputs of the process and their quality requirements.

\subsection{Model stakeholders and their quality requirements}