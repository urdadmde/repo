\subsection{Quality requirements}

\todo[inline,color=green!40]{Fritz:In this section we should refrain from referring to URDAD. Lets look purely at the requirements of a services-oriented analysis and design process for MDA}

In order to be able to assess quality and quality drivers one needs to identify the stakeholders in both, the process and the resultant analysis and design model and their quality requirements regarding these.
with defined inputs and outputs for each process step and a metamodel capturing the outputs of the process. In this section we aim to identify the stakeholders of both, the process and the resultant model and their respective quality requirements.

\subsection{Model stake holders and their quality requirements}

The core stakeholders in the model include \emph{requirements engineering} (e.g.\ business analysis), \emph{architecture}, \emph{implementation} (e.g.\ developers), \emph{quality assurance} (e.g.\ testing), \emph{project management} and the \emph{client} (e.g.\ business). In order to address their respective responsibilities, these stakeholders require certain model qualities. All of these stakeholders require semantic quality. The pragmatic quality requirements of these stakeholders differ, however.

{\bf Requirements engineers} (e.g.\ business analysts) use the model to capture the functional and non-functional requirements of  services in the form of services contracts and (business) process designs realizing these services contracts. They require \emph{semantic quality} in order to have precise and non-ambiguous meaning. On the \emph{pragmatic qualities} they particularly require \emph{modifiability} as they would have to modify the model in the context of changing requirements and process improvement. Modifiability itself is improved through \emph{consistency}, \emph{simplicity} and \emph{reuse}. 

Within in an MDA based approach, {\bf implementation} (e.g.\ developers or managers who implement and control manual processes) perform the implementation mapping of the services onto a realization within in an architecture specified by the architecture team. To this end they primarily require \emph{completeness}, \emph{consistency} and \emph{simplicity}. However, model \emph{reuse} and \emph{traceability} typically lead to implementations with those same qualities.

For the purpose of this paper we define architecture as the infrastructure within which processes are deployed and executed. Thus, whilst implementation is concerned with the actual services implementations, the architecture team is concerned with the infrastructure hosting the services. The architecture may span organizational, hardware and software infrastructure. Organizational and systems architects need to be able to design an architecture hosting the services in such a way that the quality requirements (\emph{quality of service}) can be met. They also need to be able to assess whether an existing architecture can host a service as per contract and specify architectural modifications if that is not the case. {\bf Architecture} typically uses the high level services contracts and process specifications. These need to be \emph{complete} and \emph{consistent}, 

{\bf Quality assurance} needs to span across the full development process including quality assurance on the process itself, on the outputs of the different analysis and development activities detect and report quality defects in in service providers. They require the services contracts specifying both, functional and quality requirements for services. Potential service implementations would be tested against the respective service contract. Service quality assurance requires \emph{complete} and \emph{consistent} service contracts. For model validation relies on \emph{testability} and is improved through \emph{traceable}, \emph{simplicity} and \emph{consistency}. Achieving a high level of \emph{reuse} can also reduce the complexity of model validation.

{\bf Project Management} requires \emph{completeness} assessment and \emph{traceability} as well to support estimation and status reporting.

The {\bf client} uses the model to obtain (business) process documentation and services contract documentation. Business process documentation is useful for the general understanding and running of the business as well as for business process optimization. Services contracts can be used to source concrete service providers. For both of these uses \emph{nderstandability}, \emph{completeness} and \emph{consistency} are important.

\begin{tabular}

\cite{lange_christiaan_assessing_2007}. 



\cite{lange_managing_2005,lange_improving_2006}.


MDA requires model to be technology neutral.


which is meant to be used to generate the platform independent model for MDA

\subsection{Process stakeholders and their quality requirements}

In order to be able to assess the quality of the process and of the resultant model, we need to identify the quality requirements of the various stakeholders in the process and model respectively.


\cite{berard_what_1995}

Stake holders:
\begin{itemize}
  \item Project management (measurability, repeatability, estimatability)
  \item Requirements specialists (ease of use, simple process, defined process activities, defined inputs and outputs, tool support)
  \item Business (low cost, trainable, grow in islands)
\end{itemize}


\begin{itemize}
  \item Process measurability
  \item Repeatability
  \item Defined inputs and outputs
  \item Clearly specified tasks with defined activities
  \item Process consistency (URDAD generates itself)
\end{itemize}

Can apply process to a sub-world, decouples from higher and lower level granularities via contracts

Could introduce more abstract qualities and things the process must have to realize these, e.g. \emph{usability} affected by many of these

CMM requires process definition


\subsection{Internal process consistency}

\todo[inline,color=blue!40]{Fritz: Reuse sounds to me like both, a process and a model quality}

Here show that if you use URDAD to design an analysis and design methodlogy, you will get URDAD. Feed additional concepts into URDAD.

Here identify the stakeholders in both, the process and the outputs of the process and their quality requirements.

\subsection{Model stakeholders and their quality requirements}