\subsection{Quality requirements}

\todo[inline,color=green!40]{Fritz:In this section we should refrain from referring to URDAD. Lets look purely at the requirements of a services-oriented analysis and design process for MDA}

URDAD defines an service-oriented analysis and design process with defined inputs and outputs for each process step and a metamodel capturing the outputs of the process. In this section we aim to identify the stakeholders of both, the process and the resultant model and their respective quality requirements.

\subsection{Model stake holders and their quality requirements}

A services oriented analysis and design methodology for MDA needs to yield a model which is platform independent and provides 1) the service contract for any required service, 2) the domain of responsibility from which such a service would have to be sourced (facilitating service discovery and responsibility localization), and 3) the service process in the form of an `orchestration' across lower level services\cite{}.

The core stakeholders in the model include \emph{requirements engineering} (e.g.\ business analysis), \emph{architecture}, \emph{implementation) (e.g.\ developers), \emph{quality assurance} (e.g.\ testing) and the \emph{client} (e.g.\ business). In order to address their respective responsibilities, these stakeholders require certain model qualities. All of these stakeholders require semantic quality. The pragmatic quality requirements of these stakeholders differ, however.

{\bf Requirements engineers} (e.g.\ business analysts) use the model to capture and validate the functional and non-functional requirements of  services in the form of services contracts and (business) process designs realizing these services contracts. They require \emph{modifiability} as they would have to modify the model in the context of changing requirements and process improvement. Modifiability is improved through \emph{consistency}, \emph{simplicity} and \emph{reuse} whilst model validation is facilitated through \emph{traceable} and \emph{testable}. 

For the purpose of this paper we define architecture as the infrastructure within which processes are deployed and executed. The architecture may span organizational infrastructure, hardware infrastructure and software architecture. Organizational and systems architects need to be able to design an architecture hosting the services in such a way that the quality requirements (\emph{quality of service}) can be met. They also need to be able to assess whether an existing architecture can host a service as per contract and specify architectural modifications if that is not the case. 

{\bf Architects} require 

{\bf Developers} use the model as well as the architecture specification to perform the implementation mapping of services.

Stakeholder model quality requirements of the different stakeholders are linked to the usage purpose of the model \cite{lange_christiaan_assessing_2007}. 



\cite{lange_managing_2005,lange_improving_2006}.


MDA requires model to be technology neutral.


which is meant to be used to generate the platform independent model for MDA

\subsection{Process stakeholders and their quality requirements}

In order to be able to assess the quality of the process and of the resultant model, we need to identify the quality requirements of the various stakeholders in the process and model respectively.


\cite{berard_what_1995}

Stake holders:
\begin{itemize}
  \item Project management (measurability, repeatability, estimatability)
  \item Requirements specialists (ease of use, simple process, defined process activities, defined inputs and outputs, tool support)
  \item Business (low cost, trainable, grow in islands)
\end{itemize}


\begin{itemize}
  \item Process measurability
  \item Repeatability
  \item Defined inputs and outputs
  \item Clearly specified tasks with defined activities
  \item Process consistency (URDAD generates itself)
\end{itemize}

Can apply process to a sub-world, decouples from higher and lower level granularities via contracts

Could introduce more abstract qualities and things the process must have to realize these, e.g. \emph{usability} affected by many of these

CMM requires process definition


\subsection{Internal process consistency}

\todo[inline,color=blue!40]{Reuse sounds to me like both, a process and a model quality}

Here show that if you use URDAD to design an analysis and design methodlogy, you will get URDAD. Feed additional concepts into URDAD.

Here identify the stakeholders in both, the process and the outputs of the process and their quality requirements.

\subsection{Model stakeholders and their quality requirements}