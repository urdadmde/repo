\section{Discussion of quality drivers used by URDAD}

The model and process quality drivers discussed in \ref{sec:modelQualityDriversAndMetrics} are built into the URDAD process. Table \ref{tab:qualityDrivers} lists the employed model quality drivers and the model qualities they are meant to support.

\begin{table}[h]
 \caption{URDAD model quality drivers related to model qualities.}
 \label{tab:qualityDrivers}
\begin{tabular}{|l|cc|cccccccc|} \hline
\multirow{4}{*}{\bf Quality-driver} & \multicolumn{10}{c|}{\bf Model qualities} \\ \cline{2-11}
& & & \multicolumn{8}{c|}{Pragmatic model qualities}\\ \cline{4-11}
    & \begin{sideways}Semantic\end{sideways} & \begin{sideways}Syntactic\end{sideways}  & \begin{sideways}Simplicity\end{sideways}
    & \begin{sideways}Completeness\end{sideways} & \begin{sideways}Modifiability\end{sideways} & \begin{sideways}Consistency\end{sideways}
    & \begin{sideways}Decoupling\end{sideways} & \begin{sideways}Cohesion\end{sideways} & \begin{sideways}Reusability\end{sideways}
    & \begin{sideways}Traceability\end{sideways} \\ \hline
%                                       Semantic     Syntax        Simplicity  Completeness   Modifiable  Consistent  Decoupled    Cohesion     Reuse        Traceable
Define metamodel or ontology                   & \checkmark & \checkmark & \checkmark & \checkmark & \checkmark & \checkmark & \checkmark &            &            & \checkmark \\
Define concrete DSL grammars                   &            & \checkmark & \checkmark &            & \checkmark &            &            &            &            
& \\
Fix levels of granularity                      &            &            & \checkmark &            & \checkmark &            &            &            &
\checkmark & \checkmark \\ 
Decouple services via contracts                &            &            & \checkmark &            & \checkmark &            & \checkmark &            & \checkmark & \checkmark \\ 
Apply single reponsibility principle           &            &            & \checkmark &            & \checkmark &            &            & \checkmark & \checkmark & \checkmark \\ 
Specify testable pre- \& post-conditions       &            &            &            & \checkmark & \checkmark & \checkmark &            &            &            &  \\ 
Localize process within controller             &            &            & \checkmark &            & \checkmark &            & \checkmark & \checkmark & \checkmark & \checkmark \\ 
Include traceability Links                     &            &            & \checkmark & \checkmark & \checkmark & \checkmark &            &            &            & \checkmark \\ \hline 
\end{tabular}
\end{table}

The \emph{semantic quality} has been verified through analyzing the URDAD process and models for the set of statements required by the URDAD methodology and checking that each of the required statements can be made using the URDAD DSL. The URDAD metamodel has also been empirically tested by encoding example models and verifying analytically whether all the information required for code and test generation is available from the model. The consistency of the URDAD metamodel has been verified by transforming the metamodel to an ontology using the \emph{TwoUse} \cite{parreiras_using_2010} Eclipse plugin and analysing the ontology with respect to consistency using a description logic reasoner. The complexity of the modeling language was constrained by including only concepts required by the URDAD methodology. The URDAD metamodel has about an order of magnitude less classes and realationships than the UML.

\emph{Syntactic quality} is enforced by having a concrete text syntax specified for the URDAD DSL. A concrete diagrammatic syntax is currently being developed. We have also generated a syntax-validating compiler for the concrete text syntax which enforces adherence of the model specification to the URDAD DSL text grammar. In addition we use validators to assess adherence to the URDAD metamodel structure and compliance with the URDAD metamodel constraints.

\emph{Reusability Drivers} employed by URDAD include the requirement that all services must realize a service contract and hence be pluggable. Furthermore, the process includes a step checking whether any of the required services can be combined into a single, cohesive, higher level service, resulting in improved reusability across levels of granularity. Cohesion and hence its quality drivers also improve discoverability and reusability. Linkage between service and contract it realizes aids service provider discoverability. The URDAD process is a technology neutral process assembling processes from services specified through services contracts. During the implementation mapping phase an adapter layer is inserted which adapts concrete service providers to the service contract further improving reuse.

The quality drivers used for \emph{simplicity} include the specification of  the URDAD DSL which provides a compact, precise language for URDAD which reduces model size and improves model understandability when compared to using the UML. The process also enforces that all activities specified for the realization of a service are activities which address functional requirements. Enforcing the \emph{single responsibility principle} improves simplicity through separation of concerns. Furthermore, the process includes a step checking whether any of the required services into a single, cohesive, higher level service, resulting in improved reusability across levels of granularity. The URDAD metamodel also does not allow for duplicate process specification, When using the UML a process can be specified through a variety of diagrams including activity and sequence diagrams, state charts, collaboration diagams and even interaction overview diagrams. The extensive notation, potential information duplication and inconsistency risks all add to model complexity.

The quality drivers used for \emph{cohesion} include that the process assigns services contracts to responsibility domains. Enforcing services to be stateless, i.e.\ that no state is maintained within a service across service requests, makes services self-contained, cohesive units of functionality.

Model \emph{consistency} across UML models is generally very weak - different UML models have commonly very different model structures and are often even semantically different. Furthermore, using different diagrams, particularly for process specification, commonly results in inconsistency issues across these diagrams. The model consistency drivers used within the URDAD include that the specification of a metamodel which fixes the model structure of an URDAD model. Furthermore the process specifies a set of process steps with defined inputs, outputs and activities for each process step.

Structural model \emph{completeness} is enforced through the metamodel. Process completeness is partially addressed by requiring that the process must include activities which address all functional requirements. The process does not, however, enforce completeness over levels of granularity. A model is regarded as complete if a level of granularity is complete. This makes the assumption that service providers realizing the services contracts for the required services can be found. URDAD does not verify the latter.

\emph{Modifiability} is addressed through enforced decoupling via services contracts, localization of process definition. The defined levels of granularity also improve modifiability because modifications often have to only ba applied to a particular level of granularity. The enforced usage of stateless services improves modifiability because it leads to the ability to easily assemble new processes and modify existing processes. Simplicity and hence its quality drivers also improve modifiability.

\emph{Traceability} is important for validation and estimation. URDAD facilitates traceability through the process requiring to identify the service to be used to address each functional requirement. The metamodel provides traceability of services across levels of granularity,  from services to the functional requirement realized and from a functional requirement to the stakeholder who requires it.

The process quality drivers employed by URDAD are shown in table \ref{tab:processQualityDrivers}

\begin{table}[h]
\caption{URDAD model quality drivers related to model qualities.}
\label{tab:processQualityDrivers}
\begin{tabular}{|l|ccccccc|} \hline
\multirow{2}{*}{\bf Quality-driver} & \multicolumn{7}{c|}{\bf Process qualities} \\ \cline{2-8}
    & \begin{sideways}Low cost\end{sideways}  & \begin{sideways}Repeatability\end{sideways} & \begin{sideways}Estimatability\end{sideways}
    & \begin{sideways}Trainable\end{sideways}
    & \begin{sideways}Measurability\end{sideways} & \begin{sideways}Consistency\end{sideways} & \begin{sideways}Isolation\end{sideways} \\ \hline
%                    Cost         Repeatable    Estimatable  Trainable    Measurable  Consistent   Isolated
Simple             & \checkmark & \checkmark &            & \checkmark &            &            &            \\
Process definition & \checkmark & \checkmark &            & \checkmark & \checkmark & \checkmark &            \\
Decoupling         &            &            &            &            &            &            & \checkmark \\ 
Metrics            &            &            & \checkmark &            & \checkmark &            &            \\ 
Recursive process  &            & \checkmark &            & \checkmark &            & \checkmark & \checkmark \\ \hline
\end{tabular}
\end{table}

