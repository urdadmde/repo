\section{Outlook}

The core subsequent work I have envisaged for the relatively near future includes
\begin{itemize}
 \item completion of partially completed and envisaged papers,
 \item specification of a graphical syntax and providing guidance for and oversight of the tools construction,
 \item mathematical formalization of URDAD,
 \item guiding model transformation and code generation sub-projects
\end{itemize}

\subsection{Completion of papers}

The ``A domain-specific language for URDAD'' paper is nearing completion. Even though the dead-line has been shifted to 15 April, I would like to have most of my work completed by 7 April so that the remainder of the time can be used for critical reviews and polishing.

After completion of the ``A domain-specific language for URDAD'' paper, I want to focus on the ``URDAD as a quality-driven process'' paper which we need to get out of the way once and for all. One of the things I would like to show in that paper is the recursive nature of URDAD, i.e. that if you use URDAD to design an analysis and design methodology, you get back URDAD. This is important for the internal consistency of the URDAD methodology.

\subsection{Graphical syntax specification and related tools development}
The next priority after that paper is for me the completion of a concrete specification of a graphical syntax for URDAD. It is critical that the graphical sntax is minimal, simple and intuitive so that it can be effectively used by domain experts. The core open aspect is the diagrammatic process specification which I currently envisage to be in a structural form and not in the form of a graph.

I will then provide assistance to the developer (Blessing) to develop tools supporting the graphical syntax (modeling tools similar to UML modeling tools).

\subsection{Mathematical formalization of URDAD}
I need to get more proficient in using formal tools like description logic and to get comfortable suing these formalisms to show, for example,  model consistency, satisfiability and subsumption. This is meant to lead to practical consistency and hopefully completeness tests as well as model complexity assessments which can be used in the QA 

\subsection{Guiding model transformation and code generation sub-projects}

I see myself also as providing guidance for the Java EE and SOA model transformation projects leading ultimately to code generation. I will have to critically assess challanges the respective students (Craig and Juan) are facing and feed any lessons learnt back into the critical assessment and potential improvement of the metamodel,

